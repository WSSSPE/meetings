%%%%%%%%%%%%%%%%%%%%%%%%%%%%%%%%%%%%%%%%%%%%%%%%%%%%%%%%%%%%
\section{Engineering Design Group Discussion}
\label{sec:appendix_eng_design}
%%%%%%%%%%%%%%%%%%%%%%%%%%%%%%%%%%%%%%%%%%%%%%%%%%%%%%%%%%%%

Birgit Penzenstadler\footnote{email: \href{mailto:birgit.penzenstadler@csulb.edu}{birgit.penzenstadler@csulb.edu}} and Colin C. Venters\footnote{email: \href{mailto:c.venters@hud.ac.uk}{c.venters@hud.ac.uk}} will serve as the points of contact for this working group, and be responsible for ensuring timely progress of the planned actions.

\subsection{Group Members}

\begin{itemize}
\item Birgit Penzenstadler -- California State University, CA, USA
\item Colin C. Venters -- University of Huddersfield, Huddersfield, UK
\item Matthias Bussonnier -- UC Berkeley, CA, USA
\item Jeff McWhirter -- Geode Systems 
\item Patrick Nichols -- National Center for Atmospheric Research, CO, USA
\item Ilian Todorov -- Science \& Technology Facilities Council, UK
\item Ian Taylor -- Cardiff University, UK
\item Alexander Vyushkov -- University of Notre Dame, IN, USA
\end{itemize}

\subsection{Summary of Discussion}
Software engineering principles form the basis of methods, techniques, methodologies and tools. This group discussed the principles of software engineering design for sustainable software and their application in various domains including quantum chemistry and epidemiology. 

\subsection{Description of Opportunity, Challenges, and Obstacles}
The opportunity was identified to distill existing software engineering and sustainability design knowledge into ''bite sized'' chunks for the Computational Science and Engineering Community. In addition, two primary challenges were identified:
\begin{itemize}
\item Mapping of the principles to best practice.
\item Demonstrating the return on investment of those best practices.
\end{itemize}

\subsection{Key Next Steps}
In order to achieve (1) the systematic analysis a number of example systems from different scientific domains with regards to the identified principles, (2) the identification of the commonalities and gaps in applying principles to different scientific systems, and (3) the proposal of a set of guidelines on the principles, the following next steps were discussed:

\subsection{Plan for Future Organization}
The following plan for future organization was discussed:
\begin{itemize}
\item Identify suitable undergraduate or post-graduate students.
\item Design and pilot study.
\item Organizing coordinating online calls via Google Hangout.
\end{itemize}

\subsection{What Else is Needed?}
The following list of what else is required :
\begin{itemize}
\item Ethics committee review panel approval required for data collection.
\end{itemize}

\subsection{Key Milestones and Responsible Parties}
The following key milestones were discussed as a roadmap for the set of guidelines on software engineering principles:
\begin{itemize}
\item Oct/Nov 2015: Study design and interview guideline
\item Jan/Feb 2016: Interviews conducted and transcribed
\item Mar/Apr 2016: Analysis complete
\item May 2016: Report written
\end{itemize}

\subsection{Description of Funding Needed}
Specific funding was not discussed in this working group. However, this is a open topic that can be be discussed in relation to emerging funding calls from National agencies or grant proposal initatives.
