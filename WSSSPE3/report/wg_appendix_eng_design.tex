%%%%%%%%%%%%%%%%%%%%%%%%%%%%%%%%%%%%%%%%%%%%%%%%%%%%%%%%%%%%
\section{Engineering Design Group Discussion}
\label{sec:appendix_eng_design}
%%%%%%%%%%%%%%%%%%%%%%%%%%%%%%%%%%%%%%%%%%%%%%%%%%%%%%%%%%%%

Birgit Penzenstadler\footnote{email: \href{mailto:birgit.penzenstadler@csulb.edu}{birgit.penzenstadler@csulb.edu}} and Colin C. Venters\footnote{email: \href{mailto:c.venters@hud.ac.uk}{c.venters@hud.ac.uk}} will serve as the points of contact for this working group, and be responsible for ensuring timely progress of the planned actions.

\subsection{Group Members}

\begin{itemize}
\item Birgit Penzenstadler -- California State University, CA, USA
\item Colin C. Venters -- University of Huddersfield, Huddersfield, UK
\item Matthias Bussonnier -- UC Berkeley, CA, USA
\item Jeff McWhirter -- Geode Systems 
\item Patrick Nichols -- National Center for Atmospheric Research, CO, USA
\item Ilian Todorov -- Science \& Technology Facilities Council, UK
\item Ian Taylor -- Cardiff University, UK
\item Alexander Vyushkov -- University of Notre Dame, IN, USA
\end{itemize}

\subsection{Summary of Discussion}
Software engineering principles form the basis of methods, techniques, methodologies and tools. This group discussed the principles of software engineering design for sustainable software and their application in various domains. The group included members from different backgrounds, including quantum chemistry, epidemiology, computer science, software engineering, and microscopy. Each participant was invited to give their perspective on the topic area and what they thought were the crucial points for discussion. There was a general consensus that there was a need for relating principles to practice for the computational science and engineering community. Furthermore, various members of the group expressed their interest in tools and best practices for facilitating the maintenance and evolution of scientific software systems. It was agreed to identify principles from software engineering and from sustainability design and, based on those lists, discuss what each of those would mean applied to specific example systems from the expert domains of some of the group members. The group identified a number of software engineering principles drawn from the SoftWare Engineering Body of Knowledge (SWEBOK)~\cite{swebokv3}:

Software design principles:
\begin{itemize}
\item Abstraction;
\item Coupling and cohesion;
\item Decomposition and modularization;
\item Encapsulation and information hiding;
\item Separation of interface and implementation;
\item Sufficiency completeness \& primitiveness;
\item Separation of concerns.
\end{itemize}

User interface design principles:
\begin{itemize}
\item Learnability;
\item User familiarity;
\item Consistency;
\item Minimal surprise;
\item Recoverability;
\item User guidance;
\item User diversity
\end{itemize}

The sustainability design principles were drawn from the Karlskrona Manifesto on Sustainability Design~\cite{karlskrona,becker2015}:
\begin{itemize}
\item Sustainability is systemic;
\item ...is multidimensional
\item ...is interdisciplinary;
\item ...transcends the system's purpose;
\item ...applies to both a system and its wider contexts;
\item ...requires action on multiple levels;
\item ...requires multiple timescales;
\item Changing design to take into account long-term effects doesn't automatically imply sacrifices;
\item System visibility is a precondition for and enabler of sustainability design.
\end{itemize}

This congregated list is an initial collection of principles that could be extended by adding from further related work from separate disciplines within the field of software engineering, including requirements engineering, software architecture, and testing. The group identified two example systems to discuss the application of the principles. The first one was a quantum chemistry system that allows the analysis of the characteristics and capabilities of molecules and solids. The second one was a modeling system for malaria that permitted biologists to analyze a range of datasets across geography, biology, and epidemiology, and add their own datasets. The group then examined the principles and took a retrospective analysis of what the developers did in practice against how the principles could have made a difference. 

\subsection{Description of Opportunity, Challenges, and Obstacles}
The opportunity was identified to distill existing software engineering and sustainability design knowledge into ``bite sized'' chunks for the Computational Science and Engineering Community. In addition, two challenges were pointed out: 
\begin{itemize}
\item Mapping of the principles to best practice.
\item Demonstrating the return on investment of those best practices.
\end{itemize}

\subsection{Key Next Steps}
The next steps in this endeavor are to (1) Systematically analyze a number of example systems from different scientific domains with regards to the identified principles, to (2) Identify the commonalities and gaps in applying those principles to different scientific systems, and to (3) Propose a guideline on the principles and how they exemplary apply to scientific software system.

\subsection{Plan for Future Organization}


\subsection{What Else is Needed?}


\subsection{Key Milestones and Responsible Parties}


\subsection{Description of Funding Needed}
