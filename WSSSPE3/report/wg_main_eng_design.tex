\subsection{Principles for Software Engineering Design for Sustainable Software} 

\subsubsection{Why it is important}

Software engineering principles form the basis of methods, techniques, methodologies and tools~\cite{}. However, there is often a mismatch between software engineering theory and practice particulalry in the fields of compuational science and engineering, which can lead to the development of unsustainable software~\cite{}. Understanding and applying software engineering principles is essential in order to create and maintain sustainable software~\cite{}.

\subsubsection{Fit with related activities}
The group discussion focused on identifying existing principles of software engineering design that could be adopted by the computational science and engineering communities.

\subsubsection{Discussion}

The group included members from different backgrounds, including quantum chemistry, epidemiology, computer science, software engineering, and microscopy. Each participant was invited to give their perspective on the topic area and what they thought were the crucial points for discussion. There was a general consensus that there was a need for relating principles to practice for the computational science and engineering community. Furthermore, various members of the group expressed their interest in tools and best practices for facilitating the maintenance and evolution of scientific software systems. It was agreed to identify principles from software engineering and from sustainability design and, based on those lists, discuss what each of those would mean applied to specific example systems from the expert domains of some of the group members. The group identified a number of software engineering principles drawn from the SoftWare Engineering Body of Knowledge (SWEBOK)~\cite{swebokv3}. 

Software design principles included: Abstraction; Coupling and cohesion; Decomposition and modularization; Encapsulation and information hiding; Separation of interface and implementation; Sufficiency completeness \& primitiveness; and Separation of concerns. Similalry, user interface design principles included: Learnability; User familiarity; Consistency; Minimal surprise; Recoverability; User guidance; and User diversity. The sustainability design principles were drawn from the Karlskrona Manifesto on Sustainability Design~\cite{Becker:2014}. The maifestio states that sustainability is systemic; multidimensional; interdisciplinary; transcends the system's purpose; applies to both a system and its wider contexts; requires action on multiple levels; requires multiple timescales; changing design to take into account long-term effects doesn't automatically imply sacrifices; system visibility is a precondition for and enabler of sustainability design.
%\todo{should these sets of bullet points be moved to the appendix? or moved into paragraph form, since they are relatively short items?}
A number of sustainable software engineering principles proposed by Tate~\cite{tate2005} were also considered including: continual refinement of product and project practices; a working product at all times; continual emphasis on design; and value defect prevention over defect detection.

This congregated list is an initial collection of principles that could be extended by adding from further related work form separate disciplines within the field of software engineering, including requirements engineering, software architecture, and testing. The group identified two example systems to discuss the application of the principles. The first one was a quantum chemistry system that allows the analysis of the characteristics and capabilities of molecules and solids. The second one was a modeling system for malaria that permitted biologists to analyze a range of datasets across geography, biology, and epidemiology, and add their own datasets. The group then examined the principles and took a retrospective analysis of what the developers did in practice against how the principles could have made a difference.

\subsubsection{Plans}
The next steps in this endeavor are to (1) Systematically analyze a number of example systems from different scientific domains with regards to the identified principles, to (2) Identify the commonalities and gaps in applying those principles to different scientific systems, and to (3) Propose a set of guidelines on the principles and how they exemplary apply to scientific software system. 

\subsubsection{Landing Page}
In the absence of a landing page, the Principles for Software Engineering Design for Sustainable Software working group requests an email be sent to Birgit Penzenstadler\footnote{email: \href{mailto:birgit.penzenstadler@csulb.edu}{birgit.penzenstadler@csulb.edu}} and Colin C. Venters\ to find out more about the group's efforts and how to participate.
