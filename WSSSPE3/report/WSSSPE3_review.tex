\documentclass[11pt, oneside]{amsart}
\pdfoutput=1

\usepackage{amsmath}
\usepackage{amssymb}

\usepackage{color}
\usepackage{dcolumn}
\usepackage{float}
\usepackage{graphicx}
\usepackage[utf8]{inputenc}
\usepackage[T1]{fontenc}
\usepackage{lmodern}
\usepackage{multirow}
\usepackage{rotating}
\usepackage{subfigure}
\usepackage{psfrag}
\usepackage{tabularx}
\usepackage[hyphens]{url}
\usepackage{wrapfig}
\usepackage{longtable}
\usepackage{verbatim}

% The following three lines are used for displaying footnote in tables.
\usepackage{footnote}
\makesavenoteenv{tabular}
\makesavenoteenv{table}


\usepackage{enumitem}
\setlist{leftmargin=7mm}

%\setcounter{secnumdepth}{3}
%\setcounter{tocdepth}{3}


\usepackage[bookmarks, bookmarksopen, bookmarksnumbered]{hyperref}
\usepackage[all]{hypcap}
\urlstyle{rm}

\definecolor{orange}{rgb}{1.0,0.3,0.0}
\definecolor{violet}{rgb}{0.75,0,1}
\definecolor{darkgreen}{rgb}{0,0.6,0}
\definecolor{cyan}{rgb}{0.2,0.7,0.7}
\definecolor{blueish}{rgb}{0.2,0.2,0.8}

\newcommand{\todo}[1]{{\color{blue}$\blacksquare$~\textsf{[TODO: #1]}}}
\newcommand{\note}[1]{ {\textcolor{blueish}    { ***Note:      #1 }}}
\newcommand{\katznote}[1]{ {\textcolor{magenta}    { ***Dan:      #1 }}}
\newcommand{\clunenote}[1]{ {\textcolor{orange}    { ***Tom:      #1 }}}
\newcommand{\gabnote}[1]{ {\textcolor{cyan}    { ***Gabrielle:     #1 }}}
\newcommand{\nchnote}[1]{  {\textcolor{orange}      { ***Neil: #1 }}}
\newcommand{\manishnote}[1]{  {\textcolor{violet}     { ***Manish: #1 }}}
\newcommand{\davidnote}[1]{  {\textcolor{darkgreen}      { ***David: #1 }}}
\newcommand{\colinnote}[1]{ {\textcolor{red}    {***Colin: #1 }}}
\newcommand{\choinote}[1]{ {\textcolor{orange}    {***Choi: #1 }}}

% Don't use tt font for urls
\urlstyle{rm}

% 15 characters / 2.5 cm => 100 characters / line
% Using 11 pt => 94 characters / line
\setlength{\paperwidth}{216 mm}
% 6 lines / 2.5 cm => 55 lines / page
% Using 11pt => 48 lines / pages
\setlength{\paperheight}{279 mm}
\usepackage[top=2.5cm, bottom=2.5cm, left=2.5cm, right=2.5cm]{geometry}
% You can use a baselinestretch of down to 0.9
\renewcommand{\baselinestretch}{0.96}

\sloppypar

\begin{document}

\title[]{Report on the Third Workshop on Sustainable Software for Science: Practice and Experiences (WSSSPE3)}

\author{tbd by writing and organizing}

%\author{Daniel S. Katz$^{(1)}$, Sou-Cheng T. Choi$^{(2)}$, Nancy Wilkins-Diehr$^{(3)}$, Neil Chue Hong$^{(4)}$,
%\\Colin C. Venters$^{(5)}$, James Howison$^{(6)}$, Frank Seinstra$^{(7)}$, Matthew Jones$^{(8)}$,
%\\Karen Cranston$^{(9)}$, Thomas L. Clune$^{(10)}$, Miguel de Val-Borro$^{(11)}$, Richard Littauer$^{(12)}$}
%%
%\thanks{{}$^{(1)}$ Computation Institute, 
%University of Chicago \& Argonne National Laboratory, Chicago, IL, USA; \url{dsk@uchicago.edu}}
%%
%\thanks{{}$^{(2)}$ NORC at the University of Chicago and Illinois Institute of Technology, Chicago, IL, USA; \url{sctchoi@uchicago.edu}}
%%
%\thanks{{}$^{(3)}$ University of California-San Diego, San Diego, CA, USA; \url{wilkinsn@sdsc.edu}}
%%
%\thanks{{}$^{(4)}$ Software Sustainability Institute, 
%University of Edinburgh, Edinburgh, UK; \url{N.ChueHong@software.ac.uk}}
%%
%\thanks{{}$^{(5)}$ University of Huddersfield, School of Computing and Engineering, Huddersfield, UK; \url{C.Venters@hud.ac.uk}}
%%
%\thanks{{}$^{(6)}$ University of Texas at Austin, Austin, TX, USA; \url{jhowison@ischool.utexas.edu}}
%%
%\thanks{{}$^{(7)}$ Netherlands eScience Centre, Amsterdam, Netherlands; \url{F.Seinstra@esciencecenter.nl}}
%%
%\thanks{{}$^{(8)}$ National Center for Ecological Analysis and Synthesis, Santa Barbara, CA, USA; \url{jones@nceas.ucsb.edu}}
%%
%\thanks{{}$^{(9)}$ National Evolutionary Synthesis Center, Durham, NC, USA; \url{karen.cranston@nescent.org}}
%%
%\thanks{{}$^{(10)}$ NASA Goddard Space Flight Center, Greenbelt, MD, USA; \url{Thomas.L.Clune@nasa.gov}}
%%
%\thanks{{}$^{(11)}$ Department of Astrophysical Sciences, 
%Princeton University, Princeton, NJ, USA; \url{mdevalbo@astro.princeton.edu}}
%%
%\thanks{{}$^{(12)}$ University of Saarland, Germany; \url{richard.littauer@gmail.com}}
%%
 

\begin{abstract}
\todo{need to update for WSSSPE3}
This technical report records and discusses the Second Workshop on Sustainable
Software for Science: Practice and Experiences (WSSSPE2). 
%The workshop used an
%alternative submission and peer-review process, which led to a set of papers
%divided across five topic areas: 
The report includes a description of the alternative, experimental submission
and review process, two workshop keynote presentations, a series of lightning
talks, a discussion on sustainability, and five discussions from the topic areas
of exploring sustainability; software development experiences; credit \&
incentives; reproducibility \& reuse \& sharing; and code testing \& code
review. For each topic, the report includes a list of tangible actions that were
proposed and that would lead to potential change.
%
The workshop recognized that reliance on scientific software
is pervasive in all areas of world-leading research today. The workshop
participants then proceeded to explore different perspectives on the concept of
sustainability. Key enablers and barriers of sustainable scientific software
were identified from their experiences. In addition,
recommendations with new requirements such as software credit files and software
prize frameworks were outlined for improving practices in sustainable software
engineering.
%
There was also broad consensus that formal
training in software development or engineering was rare among the
practitioners. Significant strides need to be made in building a sense of
community via training in software and technical practices, on increasing their
size and scope, and on better integrating them directly into graduate education
programs.
%
Finally, journals can define and publish policies to improve reproducibility, whereas
reviewers can insist that authors provide sufficient information and access to
data and software to allow them reproduce the results in the paper. Hence a list of
criteria is compiled for  journals to provide to reviewers so as to make it easier to
review software submitted for publication as a ``Software Paper.''

\end{abstract}


\maketitle
%\newpage

%%%%%%%%%%%%%%%%%%%%%%%%%%%%%%%%%%%%%%%%%%%%%%%%%%%%%%%%%%%%
\section{Introduction} \label{sec:intro}
%%%%%%%%%%%%%%%%%%%%%%%%%%%%%%%%%%%%%%%%%%%%%%%%%%%%%%%%%%%%

%\katznote{example comment by Dan}
%
%\gabnote{example comment by Gabrielle}
%
%\nchnote{example comment by Neil}
%
%\manishnote{example comment by Manish}
%
%\davidnote{example comment by David}

%\note{google doc of notes for reference: \url{http://tinyurl.com/q6ew45v}}
%https://docs.google.com/document/d/1-BxkYWDQ6nNNBXBStUL0xcKF9qCTlEALwf928J_MemI/edit?usp=sharing

\todo{need to update for WSSSPE3}
The Second Workshop on Sustainable Software for Science: Practice and
Experiences
(WSSSPE2)\footnote{\url{http://wssspe.researchcomputing.org.uk/wssspe2/}} was
held on 16 November, 2014 in the city of New Orleans, Louisiana, USA, in
conjunction with the International Conference for High Performance Computing,
Networking, Storage and Analysis
(SC14)\footnote{\url{http://sc14.supercomputing.org}}. WSSSPE2 followed the
model of a general initial workshop,
WSSSPE1\footnote{\url{http://wssspe.researchcomputing.org.uk/wssspe1/}}~\cite{WSSSPE1-pre-report,WSSSPE1},
which co-occurred with SC13, and a focused workshop,
WSSSPE1.1\footnote{\url{http://wssspe.researchcomputing.org.uk/wssspe1-1/}},
which was organized in July 2014 jointly with the SciPy
conference\footnote{\url{https://conference.scipy.org/scipy2014/participate/wssspe/}}.

Progress in scientific research is dependent on the quality and accessibility of
software at all levels. Hence it is critical to address challenges related to
the development, deployment, maintenance, and overall sustainability of reusable
software as well as education around software practices. These challenges can be
technological, policy based, organizational, and educational, and are of
interest to developers (the software community), users (science disciplines),
software-engineering researchers, and researchers studying the conduct of
science (science of team science, science of organizations, science of science
and innovation policy, and social science communities). The WSSSPE1 workshop
engaged the broad scientific community to identify challenges and best practices
in areas of interest to creating sustainable scientific software. WSSSPE2
invited the community to propose and discuss specific mechanisms to move towards
an imagined future practice for software development and usage in science and
engineering. The workshop included multiple mechanisms for participation,
encouraged team building around solutions, and identified risky solutions with
potentially transformative outcomes. It strongly encouraged participation of
early-career scientists, postdoctoral researchers, and graduate students, with
funds provided to the conference organizers by the Moore Foundation and the
National Science Foundation (NSF), to support the travel of potential
participants who would not otherwise be able to attend, and young participants
and those from underrepresented groups, respectively. These funds allowed 17
additional participants to attend, and each was offered the chance to present a
lightning talk.

This report extends a previous report that discussed the submission,
peer-review, and peer-grouping processes in detail~\cite{WSSSPE2-pre-report}. It
is also based on a collaborative set of notes taken with Google Docs during the
workshop~\cite{WSSSPE2-google-notes}. Overall, the report discusses the
organization work done before the workshop (\S\ref{sec:preworkshop}); the
keynotes (\S\ref{sec:keynotes}); a series of lightning talks
(\S\ref{sec:lightning}), intended to give an opportunity for attendees to
quickly highlight an important issue or a potential solution; a session on
defining sustainability (\S\ref{sec:defining}). The report also gives summaries
of action plans proposed by five breakout sessions, which explored in specific
areas including sustainability (\S\ref{sec:exploring}); software development
experiences (\S\ref{sec:devel}); credit \& incentives (\S\ref{sec:credit});
reproducibility, reuse, \& sharing (\S\ref{sec:reproduce}); code testing \& code
review (\S\ref{sec:code_testing}). Lastly, the report also includes some
conclusions (\S\ref{sec:conclusions}) and an incomplete list of attendees
(Appendix~\ref{sec:attendees}).



%%%%%%%%%%%%%%%%%%%%%%%%%%%%%%%%%%%%%%%%%%%%%%%%%%%%%%%%%%%%
\section{Submissions, Peer-Review, and Peer-Grouping} \label{sec:preworkshop}
%%%%%%%%%%%%%%%%%%%%%%%%%%%%%%%%%%%%%%%%%%%%%%%%%%%%%%%%%%%%

%\note{this section is taken from \cite{WSSSPE2-pre-report}. It could be shortened.}

\todo{need to update for WSSSPE3}
WSSSPE2 began with a call for papers~\cite{WSSSPE2-pre-report}. Based on the
goal of encouraging a wide range of submissions from those involved in software
practice, ranging from initial thoughts and partial studies to mature
deployments, but focusing on papers that are intended to lead to changes, the
organizers wanted to make submission as easy as possible. The call for papers
stated:

\begin{quote} We invite short (4-page) \textbf{actionable} papers that will lead
to improvements for sustainable software science. These papers could be a call
to action, or could provide position or experience reports on sustainable
software activities. The papers will be used by the organizing committee to
design sessions that will be highly interactive and targeted towards
facilitating action. Submitted papers should be archived by a third-party
service that provides DOIs. We encourage submitters to license their papers
under a Creative Commons license that encourages sharing and remixing, as we
will combine ideas (with attribution) into the outcomes of the workshop.
\end{quote}

The call included the following areas of interest:
\begin{quote}
\begin{itemize}
\renewcommand{\labelenumi}{\textbf{\theenumi}.}
\setlength{\rightmargin}{1em}
\item defining software sustainability in the context of science and engineering
software
\begin{itemize}
\item how to evaluate software sustainability
\end{itemize}

\item improving the development process that leads to new software
\begin{itemize}
\item methods to develop sustainable software from the outset
\item effective approaches to reusable software created as a by-product of
research
\item impact of computer science research on the development of scientific
software
\end{itemize}

\item recommendations for the support and maintenance of existing software
\begin{itemize}
\item software engineering best practices
\item governance, business, and sustainability models
\item the role, operation, and
sustainability of community software repositories
\item reproducibility, transparency needs that may be unique to science
\end{itemize}

\item successful open source software implementations
\begin{itemize}
\item incentives for using and contributing to open source software
\item transitioning users into contributing developers
\end{itemize}

\item building large and engaged user communities
\begin{itemize}
\item developing strong advocates
\item measurement of usage and impact
\end{itemize}

\item encouraging industry's role in sustainability
\begin{itemize}
\item engagement of industry with volunteer communities
\item incentives for industry
\item incentives for community to contribute to industry-driven projects
\end{itemize}

\item recommending policy changes
\begin{itemize}
\item software credit, attribution, incentive, and reward
\item issues related to multiple organizations and multiple countries, such as
intellectual property, licensing, etc.
\item mechanisms and venues for publishing software, and the role of publishers
\end{itemize}

\item improving education and training
\begin{itemize}
\item best practices for providing graduate students and postdoctoral
researchers in domain communities with sufficient training in software
development
\item novel uses of sustainable software in education (K-20)
\item case studies from students on issues around software development in the
undergraduate or graduate curricula
\end{itemize}

\item careers and profession
\begin{itemize}
\item successful examples of career paths for developers
\item institutional changes to support sustainable software such as promotion
and tenure metrics, job categories, etc.
\end{itemize}

\end{itemize}

\end{quote}


31 submissions were received; all but one used arXiv\footnote{\url{http://arxiv.org}}
or figshare\footnote{\url{http://figshare.com}} to self-publish their papers.

The review process was fairly standard. First, reviewers bid for papers. Then an
automated system matched the bids to determine assignments. After the reviewers
completed their assigned reviews (with an average of 4.9 reviews per paper and
4.1 reviews per reviewer), they used EasyChair\footnote{\url{http://easychair.org/}} 
to record scores on relevance
and comments. Finally, the organizers accessed the information to decide which
papers to associate with the workshop and provided authors with the comments to
help them improve their papers.

The organizers decided to list 28 of the papers as significantly contributing to
the workshop, a very high acceptance rate, but one that is reasonable, given the
goal of broad participation and the fact that the reports were already
self-published.

The organizers wanted very interactive sessions, with the process of creating
the sessions open to the full program committee, the paper authors, and others
who might attend the workshop. In order to do this, the organizers used 
Well Sorted\footnote{\url{http://www.well-sorted.org}} with the following steps:
\begin{enumerate}
%
\item Authors were asked to create Well Sorted ``cards'' for the papers. These
cards have a title (50 characters maximum) and a body (255 characters maximum).
%
\item Authors, members of the WSSSPE program committee, and mailing list subscribers
were asked to sort the cards. Each person dragged the cards, one by one, into
groups. A group could have as many cards as the person wanted it to have, and it
could have any meaning that made sense to that person.
%
\item Well Sorted produced a set of averages of all the sorts, with
various numbers of card clusters.
%
\end{enumerate}

The organizers then chose a sort that contained five groups that felt most
meaningful. After that, they decided on names for the five groups:
\begin{itemize}
\item Exploring Sustainability
\item Software Development Experiences
\item Credit \& Incentives
\item Reproducibility \& Reuse \& Sharing
\item Code Testing \& Code Review.
\end{itemize}

Finally, since some of the papers were not represented by cards in the process,
they were not placed in groups by the peer-grouping system. The authors of
these papers were asked which groups seemed the best for their papers; these
papers were then placed in those groups. Sections~\ref{sec:exploring}-\ref{sec:code_testing}
discuss the breakout groups, including a list of the papers associated with each
group.


%%%%%%%%%%%%%%%%%%%%%%%%%%%%%%%%%%%%%%%%%%%%%%%%%%%%%%%%%%%%
\section{Keynote} \label{sec:keynote}
%%%%%%%%%%%%%%%%%%%%%%%%%%%%%%%%%%%%%%%%%%%%%%%%%%%%%%%%%%%%
\todo{need to update for WSSSPE3}
The workshop featured two keynote addresses. In the opening keynote
presentation, Kaitlin Thaney of the Mozilla Science Lab talked about her
organization's work and policy to enable and support sustainable and
reproducible scientific research through the open web. The second keynote
speaker was Neil Chue Hong of Software Sustainability Institute. He shed
light on how scientific software is prevalently driving advances in many science
and engineering fields. Both keynote speeches spawned further discussion among
workshop participants on the crucial notion of \emph{software sustainability}
in the theme of our workshop.


Kaitlin Thaney is the Director of the Mozilla Science Lab (hereafter Mozilla), which
is a non-profit organization interested in openness, news, website
creation, and Science, all taking advantage of the open web.

Thaney started noting the unfortunate fact that many current systems suffer the
unintended consequence of creating friction that hinders users, despite
designers' original purposes to do good. An example is the National Cancer
Institute's caBIG. A total of \$350 million was spent, including more than \$60
million for management. More than $70$ tools were created, but caBIG is still
seen as a failure\footnote{Report Blasts Problem-Plagued Cancer Research Grid,
\url{http://tinyurl.com/maf6dz2}}. Those that had the least investment were the
most used; the most invested software were the least utilized.

Thaney emphasized that for efficient reproducible open research, we would need
research tools (e.g., software repositories), social capital (e.g., incentives),
and capacity (e.g., training and mentorship). Our systems would need to
communicate with each other. A point was made by a member of the audience that
as systems become less monolithic, it often becomes harder to sustain the links
between them\footnote{See, for example, \url{http://tinyurl.com/l76tba2}.}.
%http://www.slideshare.net/jameshowison/scientific-software-sustainability-and-ecosystem-complexity
%(but does Anon Grizzly have other cites/info for that?).
%In contrast, works licensed by Creative Commons typically do not
%have dependencies (a book, a photo, an artwork).

Thaney spoke about Mozilla's work around code citation, through a collaboration
and prototype crafted between Mozilla, GitHub, figshare, and Zenodo. This work
was presented at a closed meeting in May 2014 at the National Institutes of
Health (NIH) around these issues, sparking a conversation from that meeting
around what a \emph{Software Discovery
Index}\footnote{Software Discovery Index, \url{http://softwarediscoveryindex.org/report/}} might look
like. The meeting included a number of publishers, researchers, and those behind
major scientific software efforts such as
Bioconductor\footnote{Bioconductor, \url{http://www.bioconductor.org}},
Galaxy\footnote{Galaxy, \url{http://galaxyproject.org}}, and
nanoHUB\footnote{nanoHUB, \url{https://nanohub.org}}.
%facilitates more efficient scientific research. SDI identifies scientific
%software by archiving and standardizing metadata for software and hence help
%connect both developers and users.
Ted Habermann in the audience commented that if the metadata is minimal, it
would be less onerous for data providers, but more burdensome for users---it
could be challenging to keep a balance between what have to be captured and what
would be ideal if we do not want to lose user engagement as in the case of the
old Harvard Dataverse, finding often only the first four fields of three pages
of metadata were filled out.

The speaker concluded her talk urging the audience to design scientific software
with the general community, not an individual, in mind; and to design to unlock
latent potential of our systems. In addition, she encouraged everyone to rethink
how we reward researchers and support roles.
%Lastly, she cautioned the community to be mindful of jargon or semantics traps.



%%%%%%%%%%%%%%%%%%%%%%%%%%%%%%%%%%%%%%%%%%%%%%%%%%%%%%%%%%%%
\section{Lightning Talks} \label{sec:lightning}
%%%%%%%%%%%%%%%%%%%%%%%%%%%%%%%%%%%%%%%%%%%%%%%%%%%%%%%%%%%%
\todo{need to update for WSSSPE3}
\begin{comment}
\note{Lead: Choi. Volunteers welcome.
\href{http://wssspe.researchcomputing.org.uk/wssspe2/lightning-talks/}{Slides.}}

Lighting talks, breakout groups, working groups are all different. Lightning
talks were new in WSSSPE2. They were offered to all paper contributors, and some
other attendees, those who got travel awards. Breakout groups in WSSSPE1 were
just discussions about topics; in WSSSPE2, the breakout groups were supposed to
come up with plans for actions in specific areas. The idea of working groups is
that they will come out of breakout groups as a way for the groups to actually
do the actions over time, not at the workshops.

Email addresses of speakers: gvwilson@software-carpentry.org, Colin Venters
<C.Venters@hud.ac.uk>, j.spencer@imperial.ac.uk, erinrobinson@esipfed.org,
marpierc@iu.edu, jwpeterson@gmail.com, abani@buffalo.edu, clenhardt@renci.org,
dsk@uchicago.edu, Samin.Ishtiaq@microsoft.com, james@howison.name,
s.a.harris@leeds.ac.uk, marcus.hanwell@kitware.com, thabermann@hdfgroup.org,
rdowns@ciesin.columbia.edu, cboettig@gmail.com, jakob.blomer@cern.ch,
editor@ascl.net
\end{comment}

Lightning talks were a new feature in WSSSPE2. Since the workshop program was
mostly dedicated to discussions, the organizers wanted to give the attendees a
chance to also `make a pitch' for an idea, either representing a contributed
paper or something different. Eighteen attendees volunteered to participate in
the lightning talks, each given only two minutes to speak and at most one slide
to show. The talks were presented in reverse alphabetic order of speakers' last
names. In the rest of this section, we highlight the gist of some of the
speakers' messages.

%\item Greg Wilson: Close Enough for Scientific Work

\begin{enumerate}
\item \textbf{Colin C. Venters: The Nebuchadnezzar Effect: Dreaming of Sustainable
Software through Sustainable Software Architectures~\cite{Venters_poster}.}
Venters proposed that sustainable software is a composite, first-class,
non-functional requirement (NFR) that is at a minimum a measure of a system's
maintainability and extensibility, but may also include other NFRs such as
efficiency (e.g., energy, cost), interoperability, portability, reusability,
scalability, and usability. To achieve technically sustainable software, Venters
suggested that software architectures are fundamental as they are the primary
carrier of system NFRs, i.e., pre-system understanding; and influence how
developers are able to understand, analyze, extend, test, and maintain a
software system, i.e., post-deployment system understanding. In addition,
Venters highlighted that sustainability of software architectures needs to be
addressed to endure different types of change and evolution in order to mitigate
architectural drift, erosion, and knowledge vaporization.

%\item James Spencer: Developing New Developers

%\item Erin Robinson:  Sustaining Science Bazaars

\item \textbf{Marlon Pierce: Patching It Up, Pulling It
Forward~\cite{Pierce_poster}.} Pierce discussed how open open source is. Open
software needs a diverse, openly governed community behind it, just as it needs
open licensing and a public code repository. To probe the level of governance
within open source projects, he and his co-authors (Marru and Mattmann)
suggested a contest to encourage individual developers to submit
patches and requests to projects that are important to them. This simple
mechanism shall expose several governance mechanisms, such as how easy it is for
independent developers to communicate with project leadership, how projects
accept and license third-party contributions, and how projects make
decisions such as granting source tree write access.

\item \textbf{John Peterson: Continuous Integration for Concurrent MOOSE Framework
and Application Development on GitHub~\cite{Peterson_poster}.}
Peterson from the Idaho National Laboratory reported
that in March 2014, the MOOSE framework was released under an open source
license on GitHub, significantly expanding and diversifying the pool of current
active and potential future contributors on the project. The MOOSE team employs
an extensive continuous integration test suite to ensure that both the framework
and the applications based on the framework are verified before any code changes
are accepted into the repository. They use a combination of built-in Git
features such as branching and submodules, GitHub API integration capabilities,
and in-house developed testing software to perform this verification and update
the dependent applications in a relatively seamless manner for users.

\item \textbf{Abani Patra: Execute it~\cite{Patra_poster}.}
%Univ at Buffalo, SUNY
Patra discussed the value of an easily accessible platform for
executing scientific software, e.g., HUBzero to access XSEDE or other computing
resources. Such a platform for executing benchmark problems (even at a small
scale) allows the developer community access a reference implementation and
provides an easy way to train the larger user community. A second idea of this
talk was that for true usability, much more attention and support needs
to be given to the integrated use of simulation tools inside complex workflows.

%\item Christopher Lenhardt: Open Science for Synthesis (OSS): Filling the Gapin %Early Career Training

\item \textbf{Daniel S. Katz: Implementing Transitive Credit with
JSON-LD~\cite{Katz_transitive_credit_poster}.} Science and engineering research
increasingly relies on activities that facilitate research but are not rewarded
or recognized, such as: data sharing; developing common data resources, software
and methodologies; and annotating data and publications. To promote and advance
these activities, we must develop mechanisms for assigning credit, facilitate
the appropriate attribution of research outcomes, devise incentives for
activities that facilitate research, and allocate funds to maximize return on
investment. Katz discussed the issue of assigning credit for both direct and
indirect contributions by using JSON-LD to implement a prototype transitive
credit system.

\item \textbf{Samin Ishtiaq: Daemons, Notifications and Sustaining Software. }%~\cite{Ishtiaq_poster}}
%  Microsoft Research Cambridge
The reproduction and replication of novel results has become a major issue in
computer science, systems biology, and other computational disciplines. These
include both the inability to re-implement novel algorithms and approaches, and
lack of an agreement on how and what to benchmark these algorithms on. Ishtiaq
from Microsoft Research Cambridge pointed out these problems and made several
suggestions to address them.

\item \textbf{James Howison: Retract all Bit-Rotten Publications. }%~\cite{Howison_poster}
Howison sought to provoke discussion by proposing that papers whose workflows
are not kept current with the changing software ecosystem should be
automatically retracted. This would create an incentive for authors to keep
their software current and usable, rather than the current situation in which
every potential user has to do this individually. A softer version of the
proposal would identify papers whose software workflow has become bit-rotten and
allow others to keep the code up to date, either adding them as new authors of
the paper or providing credit for their academic service in some other form.

%\item Sarah Harris: Sustainability of Multidisciplinary Software: A Science
%Perspective

%\item Marcus D. Hanwell: We Need Tool Builders for Sustainable Scientific
%Software

%\item Ted Habermann: Communities as the Whole Product

\item \textbf{Robert Downs: Community Recommendations for Improving Sustainable
Scientific Software Practices~\cite{Downs_poster}.} Robert Downs, of the
Columbia University Center for International Earth Science Information Network
(CIESIN), described a focus group study conducted with the Science Software
Cluster (SSC) of the Federation of Earth Science Information Partners (ESIP).
For the study, almost 300 attendees of the 2014 Summer ESIP Meeting were invited
to participate in simultaneous roundtable discussions on sustainability of
science software. Over two-thirds of the roundtable focus groups responded to a
semi-structured survey that contained three sets of questions eliciting
recommendations for near-term actions of the community to improve sustainable
software practices. Initial analysis of the participants' responses to the
questionnaire revealed several suggestions, which included improving community
engagement and collaborative activities, increasing understanding and awareness,
and creating incentives to motivate sustainable science software practices. 
The ESIP SSC plans to engage the community in the recommended activities for
improving sustainable scientific software practices.

\item \textbf{Carl Boettiger: rOpenSci: Building Sustainable Software by
Fostering a Diverse Community~\cite{Boettiger_poster}.} Boettiger described how
the rOpenSci project has been successful by focusing not just on building
software but also on building a community of researchers who learn and adopt
their approaches to reproducible research and sustainable software practice.
Through outreach, mentoring, workshops, and hackathons, they have not only
reached new users, but also turned users into co-developers of robust software
and good practices to support data science research across a growing set of
disciplines.

\item \textbf{Jakob Blomer: The Need for a Versioned Data Analysis
Environment~\cite{Blomer_poster}.} Large-scale scientific endeavors, such as the
discovery of the Higgs boson at the Large Hadron Collider (LHC), often rely on
complex software stacks. Maintaining thousands of dependencies of software
libraries and operating system versions has shown that despite source code
availability, the setup and the validation of a minimal usable analysis
environment can easily become prohibitively expensive. In high-energy physics,
CernVM-FS, a special-purpose, open-source, versioning, and snapshotting file
system used to capture and distribute entire software stacks, proved to be
useful for providing instant access to ready-to-run data analysis environments.
% for the global scientific community as well as for
%interested citizens, for instance through the CERN Open Data Portal
%(http://opendata.cern.ch). \todo{revise}

\item \textbf{Alice Allen: Find it! Cite it! }%~\cite{Allen_poster}}
The Astrophysics Source Code Library (ASCL) is an online registry of
scientist-written software used in astronomy research. Their primary interest is
rendering research more transparent by making this software more discoverable for
examination. The ASCL is treated as a publication by an indexing resource for
astronomy, the Astrophysics Data System (ADS). ADS tracks citations to what it
indexes, including citations to ASCL entries. Increasing rewards for writing
software, whether through citation, transitive credit or other methods, gives
software authors a powerful reason to take the time to build sustainability into
their software and is an excellent way to drive community change.
\end{enumerate}

%%%%%%%%%%%%%%%%%%%%%%%%%%%%%%%%%%%%%%%%%%%%%%%%%%%%%%%%%%%%
\section{Working Groups} \label{sec:WGs}
%%%%%%%%%%%%%%%%%%%%%%%%%%%%%%%%%%%%%%%%%%%%%%%%%%%%%%%%%%%%

\subsection{White paper/journal paper about best practices in developing sustainable software}
\label{sec:best-practices}

\subsubsection{Why it is important}
\todo{short text here}


\subsubsection{Fit with related activities}
\todo{short text here - can include links/cites}

\subsubsection{Discussion}
\todo{short-ish text here}

\subsubsection{Plans}
\todo{short text here - not bullets}

[15 Nov] Introduction and scope finished
[15 Nov] Sections assigned
[31 Jan] Analysing funding possibilities for survey
[31 Jan] First versions of section
[15 Feb] Distribution to WSSSPE community
[31 Mar] Final version of white paper
[30 Apr] Submission of peer-reviewed paper?


\begin{enumerate}
\item Introduction and Scope of White Paper 
\item Related Work
\item Case Studies
\begin{enumerate} 
\item PeTSC
\item NWCHEM
\item CIG
\end{enumerate}
\item Community Related Practices
\begin{enumerate} 
\item Findings
\item Recommendations
\end{enumerate}
\item Governance and management
\begin{enumerate} 
\item Findings
\item Recommendations
\end{enumerate}
\item Funding Related
\begin{enumerate} 
\item Findings
\item Recommendations
\end{enumerate}
\item Metrics for sustainability
\item Tools
\item Conclusions
\end{enumerate}

\subsubsection{Landing Page}
\todo{link to landing page}

\include{wg_main_catalogs}
\subsection{Principles for Software Engineering Design for Sustainable Software} 

\subsubsection{Why it is important}

Software engineering principles form the basis of methods, techniques, methodologies and tools~\cite{}. However, there is often a mismatch between software engineering theory and practice particulalry in the fields of compuational science and engineering, which can lead to the development of unsustainable software~\cite{}. Understanding and applying software engineering principles is essential in order to create and maintain sustainable software~\cite{}.

\subsubsection{Fit with related activities}
The group discussion focused on identifying existing principles of software engineering design that could be adopted by the computational science and engineering communities.

\subsubsection{Discussion}

The group included members from different backgrounds, including quantum chemistry, epidemiology, computer science, software engineering, and microscopy. Each participant was invited to give their perspective on the topic area and what they thought were the crucial points for discussion. There was a general consensus that there was a need for relating principles to practice for the computational science and engineering community. Furthermore, various members of the group expressed their interest in tools and best practices for facilitating the maintenance and evolution of scientific software systems. It was agreed to identify principles from software engineering and from sustainability design and, based on those lists, discuss what each of those would mean applied to specific example systems from the expert domains of some of the group members. The group identified a number of software engineering principles drawn from the SoftWare Engineering Body of Knowledge (SWEBOK)~\cite{swebokv3}. 

Software design principles included: Abstraction; Coupling and cohesion; Decomposition and modularization; Encapsulation and information hiding; Separation of interface and implementation; Sufficiency completeness \& primitiveness; and Separation of concerns. Similalry, user interface design principles included: Learnability; User familiarity; Consistency; Minimal surprise; Recoverability; User guidance; and User diversity. The sustainability design principles were drawn from the Karlskrona Manifesto on Sustainability Design~\cite{Becker:2014}. The maifestio states that sustainability is systemic; multidimensional; interdisciplinary; transcends the system's purpose; applies to both a system and its wider contexts; requires action on multiple levels; requires multiple timescales; changing design to take into account long-term effects doesn't automatically imply sacrifices; system visibility is a precondition for and enabler of sustainability design.
%\todo{should these sets of bullet points be moved to the appendix? or moved into paragraph form, since they are relatively short items?}
A number of sustainable software engineering principles proposed by Tate~\cite{tate2005} were also considered including: continual refinement of product and project practices; a working product at all times; continual emphasis on design; and value defect prevention over defect detection.

This congregated list is an initial collection of principles that could be extended by adding from further related work form separate disciplines within the field of software engineering, including requirements engineering, software architecture, and testing. The group identified two example systems to discuss the application of the principles. The first one was a quantum chemistry system that allows the analysis of the characteristics and capabilities of molecules and solids. The second one was a modeling system for malaria that permitted biologists to analyze a range of datasets across geography, biology, and epidemiology, and add their own datasets. The group then examined the principles and took a retrospective analysis of what the developers did in practice against how the principles could have made a difference.

\subsubsection{Plans}
The next steps in this endeavor are to (1) Systematically analyze a number of example systems from different scientific domains with regards to the identified principles, to (2) Identify the commonalities and gaps in applying those principles to different scientific systems, and to (3) Propose a set of guidelines on the principles and how they exemplary apply to scientific software system. 

\subsubsection{Landing Page}
In the absence of a landing page, the Principles for Software Engineering Design for Sustainable Software working group requests an email be sent to Birgit Penzenstadler\footnote{email: \href{mailto:birgit.penzenstadler@csulb.edu}{birgit.penzenstadler@csulb.edu}} and Colin C. Venters\ to find out more about the group's efforts and how to participate.

\subsection{Funding Research Programmer Expertise}
\label{RSE}

\katznote{I don't think this section title is exactly right - are there some other options?  Also, the title of Appendix~\ref{sec:appendix_funding_spec_expert} should match, whatever is chosen.  Maybe `Funding Research Programmer Expertise'?}

%\subsubsection{Why it is important}

Research Software Engineers -- those who contribute to science and scholarship
through software development -- are an important part of the team
needed to deliver 21st century research. However, existing academic structures
and systems of funding do not effectively fund and sustain these skills.
The resulting high levels of turnover and inappropriate incentives
are a significant contributing factor to low levels of reliability and
readability observed in scientific software. Moreover, the absence of skilled and experienced 
developers retards progress in key projects, and at times causes important projects to fail completely.

Effective development of software for advanced research requires
that researchers work closely with scientific software developers who understand
the research domain sufficiently to build meaningful software at a reasonable pace. 
This requires a collaborative approach -- where developers who are fully engaged/invested 
in the research context are co-developing software with domain academics.

\subsubsection{Fit with related activities}

The solution we envision entails creating an environment where software developers 
are a stable part of the research team. Such an environment mitigates the risk
of losing a key developer at a critical moment in a projects lifetime, and provides
the benefits of building a store of institutional knowledge about specific projects as well as about software 
development for today's research. Our vision is to find a way to promote a University/research institute environment where
software developers are stable components of research project teams. 

One strategy to promote stability is implementing a mechansim for developers to obtain academic
credit for software development work. With such a mechanism in place, traditional academic funding
models and career tracks could properly sustain individuals for whom software development is their
primary contribution to research. A contributing factor to the problem with the current academic reward system is the
devastating effect on an academic
publication record resulting from time in industry; such postings often develop exactly the skills that research software
engineers need, yet returns to university positions following an industry role are penalized by the current structures.
Retention of senior developers is hard, because these people are highly in demand by the economy. However, people who have a
PhD in science and enter industry, may desire to return for diverse reasons, and should be welcomed back.

While development of new mechanisms in the current academic reward system is a worthy aspirational goal, such a dramatic
change in this structure does not seem likely in a time scale relevant to this working group. Accordingly, our working party
sought alternate solutions that may be achievable within the context of existing academic structures. The group felt that
developing dedicated research software engineering roles within the University, and finding stable funding for those individuals is the most promising mechanism for creating a stable software development staff.

Measures of impact and success for research programming groups, as well as for individual research software engineers, will
be required in order to make the case to the University for continued funding. Research software engineers will not be measured by publications, we hope, but by other measures. (what measures?) Middle-author publications are common for RSEs. Most RSEs welcome co-authorship on papers where the PIs consider the contribution deserves it.
(these last two statements seem contradictory, not sure which message is to be used?)

\subsubsection{Discussion}

It is hard for an individual PI in a university or college to support dedicated research software engineering resources, as
the need for, and funding for, these activities is intermittent within the research cycle. To sustain this capacity, therefore, it is necessary to aggregate this work across multiple research groups.

One solution is to fund dedicated software engineering roles for major research software projects at national laboratories
or other non-educational institutions. This solution is in place and working well for many well-used scientific codebases.
However, this strategy has limited application, as much of the body of software is created and maintained in research
universities. Therefore, we argue that research institutions should develop hybrid academic-technical tracks for this
capacity, where employees in this track work with more than one PI, rather than the traditional RA role within a single group
. This could be coordinated centrally, as a core facility, perhaps within research computing organizations which have
traditionally supported university cyberinfrastructure, libraryorganizations, or research offices. Alternatively, these
groups could be organizationally closer to research groups, sitting within academic departments. The most effective model
will vary from institution to institution, but the mandate and ways of working should be similar.

Having convinced ourselves that this would be a positive innovation, we were then faced with the specific question of how to
fund the initiation of this activity. A self-sustaining research software group will support itself through collaborations
with PIs in the normal grant process, with PIs choosing to fund some amount of research software engineering effort through grants in
the usual way. However, to bootstrap such a function to a level where it has sufficent reputation and client base to be self
-sustaining will generally require seed investment.

This might come from universities themselves (this was the model that led to the creation of the group in University College
London), but more likely, seed funding needs to come from research councils (as with the Research Software
Engineering Fellowship provided by the UK Engineering and Physical Sciences Research Council). We therefore recommend that
funding organizations consider how they might provide such seed funding.

Success, appropriately measured, will help make the case to such funding bodies for further investment. One might expect that metrics such as improved productivity, software adoption rates, and grant success rates would be sufficient arguments in favor of such a model. However, useful measurement of code cleanliness, and the resulting productivity gains, is an unsolved problem in empirical software engineering. To measure ``what did not go wrong'' because of an intervention is particularly hard. 

We finally noted that the institutional case for such groups is made easier by having successful examples to point to. In the
UK, a collective effort to identify the research software engineering community, with individuals clearly stating "I'm a research software engineer", has been important to the campaign. It will be useful to the global effort to similarly identify emerging research software organizations, and also, importantly, to identify longer-running research software groups, which have in some cases had a long running \emph{sui-generis} existence, but which now can be identified as part of a wider solution. There remains the problem of how to "sell" the value of this investment to investigators within the university. This is an issue best addressed by the individual organizations that embark on the plan. 

For more detail on the discussion, see Appendix~\ref{sec:appendix_funding_spec_expert}.

\subsubsection{Plans}

The first step in moving this strategy forward is to gather a list of groups that self-identify as research software engineering groups, and to reach out to other organizations to see if there may be a widespread community of RSE's who do not identify themselves as such at this time. We will collect information as to the organizational models under which these groups function, and how they are funded. For example, how many research universities currently fund people in the RSE track, whether they bear the the RSE moniker or not. Are these developers paid by the University or through a program supported by research grants/individual PIs? How did they bootstrap the developer track to get this started? How successful is the university in getting investigators to pay for fractional RSEs?  We will author a report describing our findings, should funding be available to conduct the investigation. 

\subsubsection{Landing Page}

To find more information about this group, or join it, see \url{http://www.rse.ac.uk/groups}.

\subsection{Funding Research Programmer Expertise}
\label{RSE}

\katznote{I don't think this section title is exactly right - are there some other options?  Also, the title of Appendix~\ref{sec:appendix_funding_spec_expert} should match, whatever is chosen.  Maybe `Funding Research Programmer Expertise'?}

%\subsubsection{Why it is important}

Research Software Engineers -- those who contribute to science and scholarship
through software development -- are an important part of the team
needed to deliver 21st century research. However, existing academic structures
and systems of funding do not effectively fund and sustain these skills.
The resulting high levels of turnover and inappropriate incentives
are a significant contributing factor to low levels of reliability and
readability observed in scientific software. Moreover, the absence of skilled and experienced 
developers retards progress in key projects, and at times causes important projects to fail completely.

Effective development of software for advanced research requires
that researchers work closely with scientific software developers who understand
the research domain sufficiently to build meaningful software at a reasonable pace. 
This requires a collaborative approach -- where developers who are fully engaged/invested 
in the research context are co-developing software with domain academics.

\subsubsection{Fit with related activities}

The solution we envision entails creating an environment where software developers 
are a stable part of the research team. Such an environment mitigates the risk
of losing a key developer at a critical moment in a projects lifetime, and provides
the benefits of building a store of institutional knowledge about specific projects as well as about software 
development for today's research. Our vision is to find a way to promote a University/research institute environment where
software developers are stable components of research project teams. 

One strategy to promote stability is implementing a mechansim for developers to obtain academic
credit for software development work. With such a mechanism in place, traditional academic funding
models and career tracks could properly sustain individuals for whom software development is their
primary contribution to research. A contributing factor to the problem with the current academic reward system is the
devastating effect on an academic
publication record resulting from time in industry; such postings often develop exactly the skills that research software
engineers need, yet returns to university positions following an industry role are penalized by the current structures.
Retention of senior developers is hard, because these people are highly in demand by the economy. However, people who have a
PhD in science and enter industry, may desire to return for diverse reasons, and should be welcomed back.

While development of new mechanisms in the current academic reward system is a worthy aspirational goal, such a dramatic
change in this structure does not seem likely in a time scale relevant to this working group. Accordingly, our working party
sought alternate solutions that may be achievable within the context of existing academic structures. The group felt that
developing dedicated research software engineering roles within the University, and finding stable funding for those individuals is the most promising mechanism for creating a stable software development staff.

Measures of impact and success for research programming groups, as well as for individual research software engineers, will
be required in order to make the case to the University for continued funding. Research software engineers will not be measured by publications, we hope, but by other measures. (what measures?) Middle-author publications are common for RSEs. Most RSEs welcome co-authorship on papers where the PIs consider the contribution deserves it.
(these last two statements seem contradictory, not sure which message is to be used?)

\subsubsection{Discussion}

It is hard for an individual PI in a university or college to support dedicated research software engineering resources, as
the need for, and funding for, these activities is intermittent within the research cycle. To sustain this capacity, therefore, it is necessary to aggregate this work across multiple research groups.

One solution is to fund dedicated software engineering roles for major research software projects at national laboratories
or other non-educational institutions. This solution is in place and working well for many well-used scientific codebases.
However, this strategy has limited application, as much of the body of software is created and maintained in research
universities. Therefore, we argue that research institutions should develop hybrid academic-technical tracks for this
capacity, where employees in this track work with more than one PI, rather than the traditional RA role within a single group
. This could be coordinated centrally, as a core facility, perhaps within research computing organizations which have
traditionally supported university cyberinfrastructure, libraryorganizations, or research offices. Alternatively, these
groups could be organizationally closer to research groups, sitting within academic departments. The most effective model
will vary from institution to institution, but the mandate and ways of working should be similar.

Having convinced ourselves that this would be a positive innovation, we were then faced with the specific question of how to
fund the initiation of this activity. A self-sustaining research software group will support itself through collaborations
with PIs in the normal grant process, with PIs choosing to fund some amount of research software engineering effort through grants in
the usual way. However, to bootstrap such a function to a level where it has sufficent reputation and client base to be self
-sustaining will generally require seed investment.

This might come from universities themselves (this was the model that led to the creation of the group in University College
London), but more likely, seed funding needs to come from research councils (as with the Research Software
Engineering Fellowship provided by the UK Engineering and Physical Sciences Research Council). We therefore recommend that
funding organizations consider how they might provide such seed funding.

Success, appropriately measured, will help make the case to such funding bodies for further investment. One might expect that metrics such as improved productivity, software adoption rates, and grant success rates would be sufficient arguments in favor of such a model. However, useful measurement of code cleanliness, and the resulting productivity gains, is an unsolved problem in empirical software engineering. To measure ``what did not go wrong'' because of an intervention is particularly hard. 

We finally noted that the institutional case for such groups is made easier by having successful examples to point to. In the
UK, a collective effort to identify the research software engineering community, with individuals clearly stating "I'm a research software engineer", has been important to the campaign. It will be useful to the global effort to similarly identify emerging research software organizations, and also, importantly, to identify longer-running research software groups, which have in some cases had a long running \emph{sui-generis} existence, but which now can be identified as part of a wider solution. There remains the problem of how to "sell" the value of this investment to investigators within the university. This is an issue best addressed by the individual organizations that embark on the plan. 

For more detail on the discussion, see Appendix~\ref{sec:appendix_funding_spec_expert}.

\subsubsection{Plans}

The first step in moving this strategy forward is to gather a list of groups that self-identify as research software engineering groups, and to reach out to other organizations to see if there may be a widespread community of RSE's who do not identify themselves as such at this time. We will collect information as to the organizational models under which these groups function, and how they are funded. For example, how many research universities currently fund people in the RSE track, whether they bear the the RSE moniker or not. Are these developers paid by the University or through a program supported by research grants/individual PIs? How did they bootstrap the developer track to get this started? How successful is the university in getting investigators to pay for fractional RSEs?  We will author a report describing our findings, should funding be available to conduct the investigation. 

\subsubsection{Landing Page}

To find more information about this group, or join it, see \url{http://www.rse.ac.uk/groups}.

\subsection{Transition Pathways to Sustainable Software: Industry \& Academic Collaboration} 

%\subsubsection{Why it is important}

Most scientific software is produced as a part of grant-funded research projects
sponsored by federal governments. If we are interested in the sustainability of
scientific software, then we need to understand what exactly happens when that
sponsorship ends. More than likely, the project and its resulting software will
need to undergo some kind of transition in funding and consequently management.

There are a number of potential funding transitions that may occur:  
%
\begin{itemize}

\item A project could be \textbf{refunded}, and development or maintenance of
the software continue as planned.

\item A project might locate a \textbf{new source of funding} in which case the
software may be further developed or simply maintained as before.

\item The project could transition to a \textbf{community supported model}
whereby ownership, maintenance, and stewardship of the software become similar
to peer-production models in open-source (e.g., see
Howison~\cite{howison_sustaining_2015}).

\item The project could receive some form of industry sponsorship in which case
ownership of the intellectual property, licensing, maintenance activities,
hosting, etc.\ may change significantly.

\end{itemize}

We characterize each of these potential changes in funding as ``transition
pathways'' to sustainable software (see similar work by Geels and
Schot~\cite{Geels:2007}).

At WSSSPE3 our group was interested in better understanding successful pathways
for scientific software to ``transition'' from government-funded research
projects to industry sponsorship. (This may be an initially awkward
phrase---some software projects will begin their life being sponsored by
industry, or result in collaboration between industry and academia. In such
cases, there is still a need to understand how IP and how maintenance of the
software is sustained over time.)

Although transitions are often studied under the broad umbrella of ``technology
transfer,'' we believe there are likely to be a number of different ways in which
a pathway from initial production to long-term maintenance and secure funding are
achieved. In short, industry sponsorship is an important aspect of sustaining
scientific software, but our current understanding of these transitions focuses
narrowly on commercial successes\slash failures.

\subsubsection{Fit with related activities}

In looking at existing literature that addresses industry transitions, many
reports focus on benefits that accrue to the private sector, or to a government
that originally sponsored the research project. This literature does not address
the impact that these transitions have on the accessibility or usability of the
software, or the impact that these transitions have on the career of the
researchers involved.

Examples of the former scenario (benefits accruing to private sector) are as
follows:
\begin{itemize}

\item REF Impact Case Studies: \url{http://impact.ref.ac.uk/CaseStudies/}

\item Background of projects funded in the UK: \url{http://gtr.rcuk.ac.uk/}

\item Dowling Review from the UK: addresses complexity of work between these two
communities: \url{http://www.raeng.org.uk/policy/dowling-review}

\item Pathway to Impact - UK report: two pages of grant proposals are asked to
forecast what impact they might have (including environmental, academic, economic).

\end{itemize}

\subsubsection{Discussion}

Our work at WSSSPE3 included the following activities (described in detail
below): (1) brainstorming goals for this type of research, (2) imagining
potential outcomes of completing a set of case studies on this topic, and (3)
generating a set of working definitions for some of the broad concepts we are
describing.

\textbf{Goal}

\emph{What is the goal of doing research on transition pathways?} 

Can we identify collaborations that have occurred and try to understand which were successful, which were unsuccessful, and what factors contributed to these successes/failures? 

Can we determine what each partner wants to get out of such a collaboration?
For example, why would industry be interested in collaborating with academia? 
Or why would academia be interested in collaborating with industry?

How could we design a study that focused on the impact of the software in
undergoing this type of transition?

\textbf{Potential outcomes}

A set of case studies that look at successful and unsuccessful
transitions of researchers between academia and industry. This might address 
each of the transition types (described below). Successful transitions are
described as those that lead to either weak or strong sustainability (also
defined below).

Create a generalizable framework that might allow for the study of different
transition pathways (other than academia to industry).

\textbf{General Definitions}

We characterize transitions in the following ways:
\begin{itemize}

\item Handoff model: academia initially writes the software, industry (for-profit 
or nonprofit) then takes over the project.

\item Co-Production Model: industry and academia interact throughout development
of the project.

\item Sponsorship Model: academia writes and maintains the software; 
industry contributes funding for the development\slash maintenance of software.
In this example, industry is also likely a user of the software.

\item Spinoff model: transition to a for-profit or non-profit company owned by or in
collaboration with original developers.

\end{itemize}

We characterize sustainability in the following ways:
\begin{itemize}

\item Weak Sustainability: Software continues to be accessible, useful, and
usable.

\item Strong Sustainability: Software meets criteria above, but is also able to
be reused for further innovation (i.e., issued non-restrictive open-source
license).

\end{itemize}
We refer readers to Becker et al.~\cite{Becker:2014} for an extended discussion of weak versus strong
sustainability. For more detail on the group's discussion, see
Appendix~\ref{sec:appendix_industry_interaction}.

\subsubsection{Plans}

Plans for carrying forward are currently unclear---this project would require
sustained attention and effort from our team, and at least some amount of
funding in order for us to be involved for extended periods of time.

\subsubsection{Landing Page}

To find more information about this group, or join it, see \todo{where should
someone go who want to know more about this and perhaps wants to contribute?}

\subsection{Transition Pathways to Sustainable Software: Industry \& Academic Collaboration} 

%\subsubsection{Why it is important}

Most scientific software is produced as a part of grant-funded research projects
sponsored by federal governments. If we are interested in the sustainability of
scientific software, then we need to understand what exactly happens when that
sponsorship ends. More than likely, the project and its resulting software will
need to undergo some kind of transition in funding and consequently management.

There are a number of potential funding transitions that may occur:  
%
\begin{itemize}

\item A project could be \textbf{refunded}, and development or maintenance of
the software continue as planned.

\item A project might locate a \textbf{new source of funding} in which case the
software may be further developed or simply maintained as before.

\item The project could transition to a \textbf{community supported model}
whereby ownership, maintenance, and stewardship of the software become similar
to peer-production models in open-source (e.g., see
Howison~\cite{howison_sustaining_2015}).

\item The project could receive some form of industry sponsorship in which case
ownership of the intellectual property, licensing, maintenance activities,
hosting, etc.\ may change significantly.

\end{itemize}

We characterize each of these potential changes in funding as ``transition
pathways'' to sustainable software (see similar work by Geels and
Schot~\cite{Geels:2007}).

At WSSSPE3 our group was interested in better understanding successful pathways
for scientific software to ``transition'' from government-funded research
projects to industry sponsorship. (This may be an initially awkward
phrase---some software projects will begin their life being sponsored by
industry, or result in collaboration between industry and academia. In such
cases, there is still a need to understand how IP and how maintenance of the
software is sustained over time.)

Although transitions are often studied under the broad umbrella of ``technology
transfer,'' we believe there are likely to be a number of different ways in which
a pathway from initial production to long-term maintenance and secure funding are
achieved. In short, industry sponsorship is an important aspect of sustaining
scientific software, but our current understanding of these transitions focuses
narrowly on commercial successes\slash failures.

\subsubsection{Fit with related activities}

In looking at existing literature that addresses industry transitions, many
reports focus on benefits that accrue to the private sector, or to a government
that originally sponsored the research project. This literature does not address
the impact that these transitions have on the accessibility or usability of the
software, or the impact that these transitions have on the career of the
researchers involved.

Examples of the former scenario (benefits accruing to private sector) are as
follows:
\begin{itemize}

\item REF Impact Case Studies: \url{http://impact.ref.ac.uk/CaseStudies/}

\item Background of projects funded in the UK: \url{http://gtr.rcuk.ac.uk/}

\item Dowling Review from the UK: addresses complexity of work between these two
communities: \url{http://www.raeng.org.uk/policy/dowling-review}

\item Pathway to Impact - UK report: two pages of grant proposals are asked to
forecast what impact they might have (including environmental, academic, economic).

\end{itemize}

\subsubsection{Discussion}

Our work at WSSSPE3 included the following activities (described in detail
below): (1) brainstorming goals for this type of research, (2) imagining
potential outcomes of completing a set of case studies on this topic, and (3)
generating a set of working definitions for some of the broad concepts we are
describing.

\textbf{Goal}

\emph{What is the goal of doing research on transition pathways?} 

Can we identify collaborations that have occurred and try to understand which were successful, which were unsuccessful, and what factors contributed to these successes/failures? 

Can we determine what each partner wants to get out of such a collaboration?
For example, why would industry be interested in collaborating with academia? 
Or why would academia be interested in collaborating with industry?

How could we design a study that focused on the impact of the software in
undergoing this type of transition?

\textbf{Potential outcomes}

A set of case studies that look at successful and unsuccessful
transitions of researchers between academia and industry. This might address 
each of the transition types (described below). Successful transitions are
described as those that lead to either weak or strong sustainability (also
defined below).

Create a generalizable framework that might allow for the study of different
transition pathways (other than academia to industry).

\textbf{General Definitions}

We characterize transitions in the following ways:
\begin{itemize}

\item Handoff model: academia initially writes the software, industry (for-profit 
or nonprofit) then takes over the project.

\item Co-Production Model: industry and academia interact throughout development
of the project.

\item Sponsorship Model: academia writes and maintains the software; 
industry contributes funding for the development\slash maintenance of software.
In this example, industry is also likely a user of the software.

\item Spinoff model: transition to a for-profit or non-profit company owned by or in
collaboration with original developers.

\end{itemize}

We characterize sustainability in the following ways:
\begin{itemize}

\item Weak Sustainability: Software continues to be accessible, useful, and
usable.

\item Strong Sustainability: Software meets criteria above, but is also able to
be reused for further innovation (i.e., issued non-restrictive open-source
license).

\end{itemize}
We refer readers to Becker et al.~\cite{Becker:2014} for an extended discussion of weak versus strong
sustainability. For more detail on the group's discussion, see
Appendix~\ref{sec:appendix_industry_interaction}.

\subsubsection{Plans}

Plans for carrying forward are currently unclear---this project would require
sustained attention and effort from our team, and at least some amount of
funding in order for us to be involved for extended periods of time.

\subsubsection{Landing Page}

To find more information about this group, or join it, see \todo{where should
someone go who want to know more about this and perhaps wants to contribute?}

\subsection{Legacy Software} \label{sec:legacy} 

This group met only briefly, for one period on the first day. They discussed
that it is difficult to define legacy code because there is so much stigma
associated with the term. At some point there will be more difficulty and
resources wasted trying to keep legacy software supported, but it will
eventually be too expensive compared to how much it would be to just rebuild the
software or kill it. Most of the group members were not able to attend on the
second day, and those who were able to attend joined other groups.


%\subsubsection{Why it is important}
%\todo{short text here}
%
%\subsubsection{Fit with related activities}
%\todo{short text here - can include links/cites}
%
%\subsubsection{Discussion}
%\todo{short-ish text here}
%
%\subsubsection{Plans}
%\todo{short text here - not bullets}
%
%\subsubsection{Landing Page}
%\todo{link to landing page}




%%%%%%%%%%%%%%%%%%%%%%%%%%%%%%%%%%%%%%%%%%%%%%%%%%%%%%%%%%%%
\subsection{Useful Metrics for Scientific Software}
\label{sec:software-metrics}
%%%%%%%%%%%%%%%%%%%%%%%%%%%%%%%%%%%%%%%%%%%%%%%%%%%%%%%%%%%%

%\subsubsection{Why it is important}

Metrics for scientific software are important for tenure and promotion, scientific impact, discovery, reducing duplication, serving as a basis for potential industrial interest in adopting software, prioritizing develop and support towards 
strategic objectives and making a case for new or continued funding.  However, there is no commonly-used 
standard for collecting or presenting metrics, nor is it known if there is a common set of metrics for scientific software. 
 It is imperative that scientific software stakeholders understand that it is useful to collect metrics.

\subsubsection{Fit with related activities}

The group discussion  focused on identifying existing frameworks and activities for scientific software metrics.  
The group identified the following related activities:

\katznote{why are these in parens?  Can the links be added as text, so that readers of this in print can see them?  I guess as citations}
\begin{itemize}

\item
\href{https://geodynamics.org/cig/dev/best-practices/}{(Computational Infrastructure for Geodynamics: Software Development Best Practices)}

\item
\href{https://docs.google.com/document/d/1cgUDH3RxrfsLotWhKKOrXUnaYFhrtjcV1TDRkFtwQKI/edit}{(WSSSPE3 Breakout Session: How can we measure the impact of a code on research, and its value to the community?)}

\item
\href{https://docs.google.com/document/d/10yj7MYEjvrg__t522XR41ogASYMp647-l-BpFTsqEV4/edit#heading=h.5lah0hp73q99}{(2015 NSF SI2 PI Workshop Breakout Session on Framing Success Metrics)}

\item
\href{https://docs.google.com/document/d/1uDim5bw8rBuubmtaUrz5Eh35NxzDgivmmdXhVzDs3tc/edit}{(2015 NSF SI2 PI Workshop Breakout Session on Software Metrics)}

\item
\href{https://docs.google.com/presentation/d/1PPLVL6uoOmisqnHTlwhsVKJBTFFK1IVzvr8FdEEIvAE/edit#slide=id.g5e66ec9f2_027}{(NSF Workshop on Software and Data Citation Breakout Group on Useful Metrics)}

\item
\href{http://www.software.ac.uk/software-evaluation-guide}{(U.K. Software Sustainability Institute Software Evaluation Guide)}

\item
\href{http://www.software.ac.uk/blog/2013-04-09-five-stars-research-software}{(U.K. Software Sustainability Institute Blog post: The five stars of research software)}

\item
\href{http://figshare.com/articles/Minimal_information_for_reusable_scientific_software/1112528}{(Minimal information for reusable scientific software)}

\item
\href{http://equipment.data.ac.uk/}{(EPSRC-funded Equipment Data Search Site)}

\item
\href{http://www.canarie.ca/software/}{(Canarie Research Software: Software to accelerate discovery)}

\item
\href{https://science.canarie.ca/researchmiddleware/platforms/list/main.html}{(Canarie Research Software: Research Software Platform Registry)}

\item
\href{https://collaboration.canarie.ca/elgg/file/view/2471/research-platform-support-for-the-canarie-registry-and-monitoring-system-revision-3}{(collaboration@CANARIE post: platform support for Canarie registry and monitoring system)}

\item
\href{https://collaboration.canarie.ca/elgg/file/view/2453/research-service-support-for-canarie-registry-and-monitoring-system-revision-7}{(collaboration@CANARIE post: service support for Canarie registry and monitoring system)}

\item
\href{https://www.openhub.net/}{(BlackDuck Open HUB)}

\item
\href{https://www.innovationpolicyplatform.org/frontpage}{(Innovation Policy Platform)}


\end{itemize}



\subsubsection{Discussion}

The group discussion began by agreeing on the common purpose of creating a set of guidance giving examples of specific metrics for the success of scientific 
software in use, why they were chosen, what they are useful to measure, and any challenges and pitfalls; then publish this as a white paper. 
 The group discussed many questions related to useful metrics for scientific software including addressing if there is a common set of metrics that 
 can be filtered in some way, can metrics be fit into a common template, which metrics would be the most useful for each stakeholder, 
 which metrics are the most helpful and how would we assess this, how are metrics monitored, and many more.  
 A more complete bulleted list of these questions can be found in Appendix H.  Next, a roadmap for how to proceed was discussed including
 creating a set of milestones and tasks.  The idea was put forth for the group to interact with the organizing committee of the 2016 NSF Software Infrastructure
  for Sustained Innovation (SI2) PI workshop in order to send a software metrics survey to all SI2 and related awardees as a targeted and relevant set of stakeholders.  
 The five solicitations for software elements released  under the NSF SI2 program all included metrics as a required component with submitters requested to include 
 {\it "a list of tangible metrics, with end user involvement, to be used to measure the success of the software element developed, ...'}. These metrics are then reported as part of annual reports to NSF by the projects. Although neither the proposal text describing the metrics nor the reported metric results are publicly available, there is reason to believe that the community will be willing to provide this information through a survey mechanism. 
This survey would be created by one of the student group members.  Similarly, it was suggested that a software metrics survey be sent to the 
UK SFTF and TRDF software projects to ask them what metrics would be useful to report.  The remainder of the discussion focused mainly on the 
creation of a white paper on this topic.  This resulted in a paper outline and writing assignments with the goal of publishing in venues including 
WSSSPE4, IEEE CISE, or JORS. More information about the group discussion is available in Appendix~\ref{sec:appendix_metrics}.

\subsubsection{Plans}

The main plans for the group going forward are the creation of a white paper on the topic of useful metrics for scientific software.  The authoring of this white paper would happen in parallel with the creation of a survey by the group with the survey results to be incorporated in the white paper.  The timeline for completion of the white paper is approximately one year targeting venues discussed in the previous section.

\subsubsection{Landing Page}

The group will use Google Docs to proceed with the authoring of the white paper in lieu of a group project landing page.  Links to the resulting white paper will be provided when completed. \todo{where will it be provided - this isn't very useful to readers as is}

\subsection{Publishing Software Working Group Discussion}

\subsubsection{Why it is important}

This group explored the value of executable papers (papers whose content includes
the code needed to produce their own results), and other forms of publishing which
include dynamic electronic content. Transitioning to this type of publication offers
possibilities of addressing, or partially addressing sustainability concerns 
such as reproducibility (the paper contains all the artifacts needed to verify its
results), transitive credit (modules an executable paper depends on must be explicitly
loaded, making it more feasible to identify them), and improving documentation (an executable
paper must explain what its code does).

Reproducibility: Part of the purpose of these venues is to (at least partially)
address the reproducibility issue by making the paper itself recompute its own
results.

Transitive Credit: Since these forms of publishing must make their sources explicit,
they should be easier to trace even if appropriately worded credit for software
is not provided. In addition, these notebooks make it possible to provide/define
additional metadata to make the tracing of credit clearer. In addition, attributions
could be added to citations to identify whether a paper extends a result, verifies it,
contradicts it, etc.

Best Practices: Because an executable paper showcases the code 

\subsubsection{Fit with related activities}
\todo{short text here - can include links/cites}

\subsubsection{Discussion}
\todo{short-ish text here}

\subsubsection{Plans}
\todo{short text here - not bullets}

\subsubsection{Landing Page}

A page will be made available on the Software Sustainability Institute website.

%%%%%%%%%%%%%%%%%%%%%%%%%%%%%%%%%%%%%%%%%%%%%%%%%%%%%%%%%%%%
\subsection{Software Credit Working Group}
\label{sec:software-credit}
%%%%%%%%%%%%%%%%%%%%%%%%%%%%%%%%%%%%%%%%%%%%%%%%%%%%%%%%%%%%

\subsubsection{Why it is important}
\todo{short text here}

\subsubsection{Fit with related activities}
\todo{short text here - can include links/cites}

Publishing Software Working Group (\S\ref{sec:publishing-software})

\subsubsection{Discussion}
\todo{short-ish text here}

\subsubsection{Plans}
\todo{short text here - not bullets}

\subsubsection{Landing Page}
\todo{link to landing page}


\subsection{Topic} \todo{Training Working Group Discussion}

\subsubsection{Why it is important}

This group explored a rapidly growing array of training which is seen to contribute to sustainable software. Offerings are diverse, providing training more or less directly relevant to sustinable software. While research institutions  support professional development for research staff, the skills taught which might impact on sustainable software are limited at best, often lacking a clear and coherent development pathway. This growing array of training opportunities could usefully be coordinated, by bringing together those involved in leading relevant initiatives, on a regular basis.

\subsubsection{Fit with related activities}
Two existing venues for discussion of related activities are identified:

\begin{itemize}

\item
\item
WSSSPE Workshop on Sustainable Software for Science: Practice and Experiences \footnote{\url{http://wssspe.researchcomputing.org.uk/}}

\item
SEHPCCSE International Workshop on Software Engineering for High Performance Computing in Computational Science and Engineering \footnote{\url{http://wssspe.researchcomputing.org.uk/}}

\end{itemize}

\subsubsection{Discussion}

Next steps have been identified to quickly test whether there is interest in establishing a community committed to increasing the degree of coordination across training projects.

See Appendix~\ref{sec:appendix_training} for more details about the discussion.

\subsubsection{Plans}

The main plans for the group are to convene discussion to explore bringing together a regular meeting of those involved in leading relavent training projects.

\subsubsection{Landing Page}

The Training working group requests an email be sent to Nick Jones \href{mailto:nick.jones@nesi.org.nz}{(nick.jones@nesi.org.nz)} to find out more about the group's efforts and how to participate.

\subsection{Building Sustainable User Communities for Scientific Software}
%\todo{change the title to your topic}

\subsubsection{Why it is important}
%\todo{short text here}

User communities are the lifeblood of sustainable scientific software. The user community includes the developers, 
both internal and external, of the software; direct users of the software; other software projects that depend on
the software; and any other groups that create or consume data that is specific to the software. Together these
groups provide both the reason for sustaining the software and, collectively, the requirements that drive its continued
evolution and improvement.

\subsubsection{Fit with related activities}
%\todo{short text here - can include links/cites}

Mozilla Science maintains a "'Working Open' Project Guide" (http://mozillascience.github.io/leadership-training). From the introduction:
\begin{quote}
Working openly with contributors enables your
    community to learn how to build and collaborate together. This
    document is a guideline on how to work openly and involve others
    in your projects with Mozilla. We want to help you engage your
    community in a way that encourages contributors and builds other
    leaders.
  \end{quote}

Several books have been written about software communities:
\begin{itemize}
\item "Art of Community" by Jono Bacon. We could consider distilling this for scientific software.
\item Iain Larmour, from EPSRC inthe UK [[https://www.epsrc.ac.uk/][EPSRC]] (not sure who from mentioned UK Collaborative Computational Projects ([[http://ccp.ac.uk][CCP]])
\end{itemize}

\subsubsection{Discussion}
%\todo{short-ish text here}

Discussion revolved around a few questions: what is the benefit of having a "community" for software sustainability, what
practices and circumstances lead to having a community, how can funding help or hinder this process, and perhaps most
importantly how can best practices be described and distilled into a document that can help new projects.

The benefits of having a community that were brought up were considered largely obvious. In addition to having advocates for
the software, and a possible source of ``free'' contributions to the codebase, the community becomes a good source for
requirements, feedback, and metrics. The software community can also act as "cheerleaders" who convince funders or other
potential users to fund/use the software, and thus help sustain the software.

Practices and circumstances that lead to a community are first, that the software offers value. But in addition to this, a
community will be much more likely to form if they receive (expert) support when they have questions. Additional contributing
factors are good usability (not always needed), and an open development process such as IPython developer meetings on YouTube.
It was also pointed out that an evangelist for the project, not necessarily but often one of the developers, can often make a
big difference. 

Funding can help the process by encouraging both value to the community and high-quality user support. Only providing funding
for the software development may create good software, but with less likelihood to have a real community. It was discussed
that federal laboratories are a good incubator for software communities, and that a general facility like EarthCube is too
dispersed to really make a community. Also, domain-specific groups within laboratories or universities might provide as an
incubator for software communities.

In describing best practices, the group discussed the different modes for starting a scientific software project: building on
an existing product that needs improving, recognizing an unsatisfied need of an existing community, or creating a new solution
to a need not yet recognized by the community. The group also thought that the existing books on software communities would
need to be evaluated in light of differences between Science Software projets and general OSS projects in terms of scale,
science, acknowledgement and credit, and funding models.

\subsubsection{Plans}
%\todo{short text here - not bullets}

The most important next steps is a ``Best Practice'' document, which would describe what successful projects with engaged
communities look like, how to replicate this type of project, and look at end-of-life on a community project. Inputs to this
document would include a software community survey of high functioning communities such as R Open Science, Python SciPy,
OPeNDAP, and Unidata, with analysis of factors that feed into their success. Also references lik the "Art of Community" could
be adapted and summarized for the science software community.

Another next step would be increasing recognition of need for science software projects to focus on building and supporting
their user community. Good Software Engineering practices are not enough, and popular training like Software Carpentry does
not currently address this issue head on.

\subsubsection{Landing Page}
\todo{link to landing page}


%%%%%%%%%%%%%%%%%%%%%%%%%%%%%%%%%%%%%%%%%%%%%%%%%%%%%%%%%%%%
\section{Conclusions} \label{sec:conclusions}
%%%%%%%%%%%%%%%%%%%%%%%%%%%%%%%%%%%%%%%%%%%%%%%%%%%%%%%%%%%%

\todo{need to update for WSSSPE3}
The WSSSPE2 workshop continued our experiment from WSSSPE1 in how we can
collaboratively build a workshop agenda, and we began a new experiment in
how to build a series of workshops into an ongoing community activity.

The differences in workshop organization in WSSSPE2 from WSSSPE1
are in using an existing service (EasyChair) to handle submissions and reviews,
rather than an ad hoc process, and using an existing service (Well Sorted) to
allow collaborative grouping of papers into themes by all authors, reviewers,
and the community, rather than this being done in an ad hoc manner by the
organizers alone.

The fact remains that contributors also want to get credit for their
participation in the process. And the workshop organizers will want to make
sure that the workshop content and their efforts are recorded. Ideally, there
would be a service that would index the contributions to the
workshop, serving the authors, the organizers, and the larger community. 
Since there still isn't such a service today, the workshop organizers are
writing this initial report and making use of arXiv as a partial solution to
provide a record of the workshop.

\begin{table*}[t]
\centering
\caption{Top tweets tagged \#WSSSPE on Nov 16, 2014. \todo{update this for WSSSPE3?}}\label{tab:tweets}
  \begin{scriptsize}
  \begin{tabular}{l|l|r|r}
 \hline
    Author  &   Tweet  & Retweets &  Favorites
\\ \hline
% Software Carpentry & Nov 17 & You call it ``project planning'' if it hasn't started yet,          & 8 & 1
%\\                          & &  and``software sustainability'' if it has and wasn't planned.     &    &
%
Neil P Chue Hong   &  Here's @SoftwareSaved guidance on Writing and using a software & 7 & 4
\\     &  management plan used by EPSRC software grants    &    &
\\     &  \url{http://www.software.ac.uk/resources/guides/software-management-plans}  &    &
%
\\  Neil P Chue Hong  &  @jameshowison as well as software plans   & 4 & 7
\\   &   \url{http://www.software.ac.uk/resources/guides/software-management-plans} &    &
\\   &   we provide a software evaluation tool:   &    &
\\   &   \url{http://www.software.ac.uk/online-sustainability-evaluation}  &    &
%
\\ Tom Crick  & $56\%$ of UK researchers develop their own software $\rightarrow  140,000$   &  14 & 8
\\ & UK researchers write research software w$/$out any formal training &    &
%
\\ Karthik Ram  &  OH: ``Institutionalize metadata before metadata institutionalizes you'' & 8 & 6
%
\\ Josh Greenberg  &  @jameshowison: ``1. retract any paper with bitrotten dependencies'' *mic drop* & 13 & 8
\\   &   ``2. add anyone who fixes bitrot as an author'' *mic drop*  &    &
%
\\ Ethan White &  ``@rOpenSci is all about community... our measures of success  & 9 & 3
\\ &   [include] how many faces are up on our community page''   &    &
%
\\ Ethan White & Daniel Katz talking about implementing transitive credit for  & 9 & 7
\\ &  software \url{http://arxiv.org/abs/1407.5117}  Work with @arfon  &    &
%
\\ Kaitlin Thaney  & Great point by @tracykteal about planning for ``end of life'' with scientific & 4 & 8
\\ &  software projects and sustainability.  & &
%
%\\ Aleksandra Pawlik  &  ``Tell us how you test your scientific software'' @gvwilson @swcarpentry  & 7 & 2
%
\\ Aleksandra Pawlik  & Lack of training as one of the main barriers for sustainable software & 10 & 4
\\ &    at @Supercomputing. @swcarpentry @datacarpentry can fix that!  & &
%
\\ Kaitlin Thaney  & My slides from this morning's keynote at & 11 & 12
\\ &  WSSSPE on Designing for Truth, Scale $+$ Sustainability:  & &
\\ &  \href{http://www.slideshare.net/kaythaney/designing-for-truth-scale-and-sustainability-wssspe2-keynote}{http://www.slideshare.net/kaythaney/}     & &
\\ &  \href{http://www.slideshare.net/kaythaney/designing-for-truth-scale-and-sustainability-wssspe2-keynote}{designing-for-truth-scale-and-sustainability-wssspe2-keynote} & &
%
\\ Neil P Chue Hong & @kaythaney shout out for @swcarpentry @datacarpentry & 9 & 4
\\ &  @rOpenSci @stilettofiend around open training activities for sustainability  & &
%
%\\ Richard Littauer &  Sweet! @swcarpentry has had 4000+ learners in the past year. & 6 &
%
\\ Neil P Chue Hong & For those interested in Github - Figshare/Zenodo integration, & 5 & 12
\\ & but want SWORD/DSpace/Fedora/ePrints see:  & &
\\ & \url{http://blog.stuartlewis.com/2014/09/09/github-to-repository-deposit/}  & &
%
\\ Hilmar Lapp & Re: adopting the unix philosophy, consider signing the Small Tools in & 7 & 6
\\ & Bioinformatics Manifesto: \url{https://github.com/pjotrp/bioinformatics}   &
%
\\Andre Luckow &  ``Traditions last not because they are excellent, & 12 & 3
\\ & but because influential people are averse to change...''  C. Sunstein     & &
%
\\ Tom Crick &  ``Can I Implement Your Algorithm?'':  A Model for  Reproducible & 9 & 8
\\   &  Research Software \url{http://arxiv.org/abs/1407.5981}  & &
%
\\Mozilla Science Lab & At a loose end this Sunday? Care about reproducibility, software  $+$ & 10 & 5
\\ &  \#openscience? Follow the    \#WSSSPE hashtag for more, live from New Orleans.  &  &
%
\\Kaitlin Thaney & I'm in New Orleans at \#WSSSPE , speaking at 9:50 ET on  scientific software & 9 & 9
\\ &   $+$ sustainability. Tune in! Live stream: \url{http://ustre.am/17ddh} & &
%
\\   Daniel S. Katz & \#WSSSPE Agenda (Sunday):  & 10 & 1
\\ & \url{http://wssspe.researchcomputing.org.uk/wssspe2/agenda/}   &  &
\\ & URL for live stream of keynotes \& lightning talks: \url{http://ustre.am/17ddh}   &  &
\\ \hline
    \end{tabular}
    \end{scriptsize}
\end{table*}

WSSSPE actively used the online social network Twitter, with hashtag
``\#WSSSPE''. There were substantially more tweets (messages) during the days of
the workshops WSSSPE2, WSSSPE1.1, and WSSSPE1. Out of about 670 tweets as of Apr
18, 2015, more than 225 were about WSSSPE2 and about 180 were posted during the
day of the workshop. Some of the main points and highlights in the meeting are
shown in Table~\ref{tab:tweets}, which summarizes the top \#WSSSPE tweets from
the day of workshop, selected by the metrics that number of retweets or
favorites larger than five and the sum of two measures greater than ten.

In terms of building community activities, we wanted to focus primarily on
working groups, which we were able to do, as discussed above, but we
also wanted to make sure that attendees felt they had a chance to get their
ideas across to the whole group, which was the purpose of the lightning talks.
Overall, this seemed to be successful at the time, in terms of both the lightning
talks and the breakout groups, and the discussion of sustainability also led
to interesting and useful results. However, the challenge that we have discovered
since WSSSPE2 is that it is very hard to continue the breakout groups'
activities.  The WSSSPE2 participants were willing to dedicate their time to
the groups while they were at the meeting, but afterwards, they have gone
back to their (paid) jobs.  We need to determine how to tie the WSSSPE
breakout activities to people's jobs, so that they feel that continuing them
is a higher priority than it is now, perhaps through funding the participants,
or through funding coordinators for each activity, or perhaps by getting
the workshop participants to agree to a specific schedule of activities during the
workshop.


%%%%%%%%%%%%%%%%%%%%%%%%%%%%%%%%%%%%%%%%%%%%%%%%%%%%%%%%%%%%
\section*{Acknowledgments} \label{sec:acks}
%%%%%%%%%%%%%%%%%%%%%%%%%%%%%%%%%%%%%%%%%%%%%%%%%%%%%%%%%%%%

Work by Katz was supported by the National Science Foundation while working at
the Foundation. Any opinion, finding, and conclusions or recommendations
expressed in this material are those of the author(s) and do not necessarily
reflect the views of the National Science Foundation.

\todo{feel free to add stuff here}


\appendix
%%%%%%%%%%%%%%%%%%%%%%%%%%%%%%%%%%%%%%%%%%%%%%%%%%%%%%%%%%%%
\section{Attendees}  \label{sec:attendees}
%%%%%%%%%%%%%%%%%%%%%%%%%%%%%%%%%%%%%%%%%%%%%%%%%%%%%%%%%%%%
\todo{need to update for WSSSPE3}
The following is a partial list of workshop attendees who registered on the
collaborative notes document~\cite{WSSSPE2-google-notes} that was used
for shared note-taking at the meeting, or who participated in a breakout groups
and were noted in that group's notes.


{\small
\begin{longtable}{ll}
   Jordan Adams          &  Tulane University
\\ Alice Allen           &  Astrophysics Source Code Library (ASCL)
\\ Gabrielle Allen       & University of Illinois Urbana-Champaign
\\ Pierre-Yves Aquilanti &  TOTAL E\&P R\&T USA
\\ Wolfgang Bangerth & Texas A\&M University
\\ David Bernholdt       &  Oak Ridge National Laboratory
\\ Jakob Blomer
\\ Carl Boettiger        &  University of California Santa Cruz \& rOpenSci
\\ Chris Bogart          &  ISR/CMU
\\ Steven R. Brandt      &  Louisiana State University
\\ Neil Chue Hong        &  Software Sustainability Institute \& University of Edinburgh
\\ Tom Clune             &  NASA GSFC
\\ John W. Cobb
\\ Dirk Colbry           &  Michigan State University
\\ Karen Cranston        &  NESCent
\\ Tom Crick             &  Cardiff Metropolitan University, UK
\\ Ethan Davis           &  UCAR Unidata
\\ Robert R Downs        &  CIESIN, Columbia University
\\ Anshu Dubey           &  Lawrence Berkeley National Laboratory
\\ Nicole Gasparini      &  Tulane University, New Orleans
\\ Yolanda Gil           &  Information Sciences Institute, University of Southern California
\\ Kurt Glaesemann       &  Pacific northwest national lab
\\ Sol Greenspan         &  National Science Foundation
\\ Ted Habermann         &  The HDF Group
\\ Marcus D. Hanwell     &  Kitware
\\ Sarah Harris          &  University of Leeds
\\ David Henty           &  EPCC, The University of Edinburgh
\\ James Howison         &  University of Texas
\\ Maxime Hughes
\\ Eric Hutton           &  University of Colorado
\\ Ray Idaszak           &  RENCI/UNC
\\ Samin Ishtiaq         &  Microsoft Research Cambridge, UK
\\ Matt Jones            &  University of California Santa Barbara
\\ Nick Jones            &  New Zealand eScience Infrastructure, University of Auckland
\\ Daniel S. Katz        &  University of Chicago \& Argonne National Laboratory
\\ Ian Kelley
\\ Hilmar Lapp           &  National Evolutionary Synthesis Center (NESCent) \& Duke University
\\ Christopher Lenhardt
\\ Richard Littauer      &  University of Saarland
\\ Frank L\"{o}ffler     &  Louisiana State University
\\ Andre Luckow          &  Rutgers
\\ Berkin Malkoc         &  Istanbul Technical University
\\ Kyle Marcus           &  University at Buffalo
\\ Bryan Marker          &  The University of Texas at Austin
\\ Suresh Marru          &  Indiana University
\\ Robert H. McDonald    &  Data to Insight Center/Libraries, Indiana University
\\ Rupert Nash
\\ Andy Nutter-Upham     &  Whitehead Institute
\\ Abani Patra           &  University at Buffalo
\\ Aleksandra Pawlik     &  Software Sustainability Institute
\\ Cody J. Permann       &  Idaho National Laboratory
\\ John W. Peterson      &  Idaho National Laboratory
\\ Benjamin Pharr        &  University of Mississippi
\\ Stephen Piccolo       &  Brigham Young University, Utah
\\ Marlon Pierce         &  Indiana University
\\ Ray Plante            &  NCSA, University of Illinois Urbana-Champaign
\\ Sushil Prasad         &  Georgia State University, Atlanta
\\ Karthik Ram           &  Berkeley Institute for Data Science, University of California Berkeley \& rOpenSci
\\ Mike Rilee            &  NASA/GSFC \& Rilee Systems Technologies
\\ Erin Robinson         &  Foundation for Earth Science
\\ Mark Schildhauer      &  NCEAS, Univ. California, Santa Barbara
\\ Jory Schossau         &  Michigan State University
\\ Frank Seinstra        &  Netherlands eScience Center
\\ James Shepherd        &  Rice University
\\ Justin Shi
\\ Ardita Shkurti        &  University of Nottingham
\\ Alan Simpson          &  EPCC, The University of Edinburgh
\\ Carol Song            &  Purdue University
\\ James Spencer         &  Imperial College London
\\ Tracy Teal            &  Data Carpentry
\\ Kaitlin Thaney        &  Mozilla Science Lab
\\ Matt Turk             &  NCSA, University of Illinois Urbana-Champaign
\\ Colin C. Venters      &  University of Huddersfield
\\ Nathan Weeks
\\ Ethan White           &  University of Florida/Utah State University
\\ Nancy Wilkins-Diehr   &  San Diego Supercomputer Center, University of California San Diego
\\ Greg Wilson           &  Software Carpentry
\end{longtable}
}

\bibliographystyle{vancouver}

\bibliography{wssspe}
\end{document}

