\documentclass[11pt, oneside]{amsart}
\pdfoutput=1

\usepackage{amsmath}
\usepackage{amssymb}

\usepackage{color}
\usepackage{dcolumn}
\usepackage{float}
\usepackage{graphicx}
\usepackage[utf8]{inputenc}
\usepackage[T1]{fontenc}
\usepackage{lmodern}
\usepackage{multirow}
\usepackage{rotating}
\usepackage{subfigure}
\usepackage{psfrag}
\usepackage{tabularx}
\usepackage[hyphens]{url}
\usepackage{wrapfig}
\usepackage{longtable}
\usepackage{verbatim}
\usepackage{booktabs,multicol}

% The following three lines are used for displaying footnote in tables.
\usepackage{footnote}
\makesavenoteenv{tabular}
\makesavenoteenv{table}


\usepackage{enumitem}
\setlist{leftmargin=7mm}

%\setcounter{secnumdepth}{3}
%\setcounter{tocdepth}{3}


\usepackage[bookmarks, bookmarksopen, bookmarksnumbered]{hyperref}
\usepackage[all]{hypcap}
\urlstyle{rm}

\definecolor{orange}{rgb}{1.0,0.3,0.0}
\definecolor{violet}{rgb}{0.75,0,1}
\definecolor{darkgreen}{rgb}{0,0.6,0}
\definecolor{cyan}{rgb}{0.2,0.7,0.7}
\definecolor{blueish}{rgb}{0.2,0.2,0.8}

\newcommand{\todo}[1]{{\color{blue}$\blacksquare$~\textsf{[TODO: #1]}}}
\newcommand{\note}[1]{ {\textcolor{blueish}    { ***Note:      #1 }}}
\newcommand{\katznote}[1]{ {\textcolor{magenta}    { ***Dan:      #1 }}}
\newcommand{\clunenote}[1]{ {\textcolor{orange}    { ***Tom:      #1 }}}
\newcommand{\gabnote}[1]{ {\textcolor{cyan}    { ***Gabrielle:     #1 }}}
\newcommand{\nchnote}[1]{  {\textcolor{orange}      { ***Neil: #1 }}}
\newcommand{\manishnote}[1]{  {\textcolor{violet}     { ***Manish: #1 }}}
\newcommand{\davidnote}[1]{  {\textcolor{darkgreen}      { ***David: #1 }}}
\newcommand{\colinnote}[1]{ {\textcolor{red}    {***Colin: #1 }}}
\newcommand{\choinote}[1]{ {\textcolor{orange}    {***Choi: #1 }}}

% Don't use tt font for urls
\urlstyle{rm}

% 15 characters / 2.5 cm => 100 characters / line
% Using 11 pt => 94 characters / line
\setlength{\paperwidth}{216 mm}
% 6 lines / 2.5 cm => 55 lines / page
% Using 11pt => 48 lines / pages
\setlength{\paperheight}{279 mm}
\usepackage[top=2.5cm, bottom=2.5cm, left=2.5cm, right=2.5cm]{geometry}
% You can use a baselinestretch of down to 0.9
\renewcommand{\baselinestretch}{0.96}

\sloppypar

\begin{document}

\title[]{Report on the Third Workshop on Sustainable Software for Science: Practice and Experiences (WSSSPE3)}

\author{tbd by writing and organizing}

%\author{Daniel S. Katz$^{(1)}$, Sou-Cheng T. Choi$^{(2)}$, Nancy Wilkins-Diehr$^{(3)}$, Neil Chue Hong$^{(4)}$,
%\\Colin C. Venters$^{(5)}$, James Howison$^{(6)}$, Frank Seinstra$^{(7)}$, Matthew Jones$^{(8)}$,
%\\Karen Cranston$^{(9)}$, Thomas L. Clune$^{(10)}$, Miguel de Val-Borro$^{(11)}$, Richard Littauer$^{(12)}$}
%%
%\thanks{{}$^{(1)}$ Computation Institute, 
%University of Chicago \& Argonne National Laboratory, Chicago, IL, USA; \url{dsk@uchicago.edu}}
%%
%\thanks{{}$^{(2)}$ NORC at the University of Chicago and Illinois Institute of Technology, Chicago, IL, USA; \url{sctchoi@uchicago.edu}}
%%
%\thanks{{}$^{(3)}$ University of California-San Diego, San Diego, CA, USA; \url{wilkinsn@sdsc.edu}}
%%
%\thanks{{}$^{(4)}$ Software Sustainability Institute, 
%University of Edinburgh, Edinburgh, UK; \url{N.ChueHong@software.ac.uk}}
%%
%\thanks{{}$^{(5)}$ University of Huddersfield, School of Computing and Engineering, Huddersfield, UK; \url{C.Venters@hud.ac.uk}}
%%
%\thanks{{}$^{(6)}$ University of Texas at Austin, Austin, TX, USA; \url{jhowison@ischool.utexas.edu}}
%%
%\thanks{{}$^{(7)}$ Netherlands eScience Centre, Amsterdam, Netherlands; \url{F.Seinstra@esciencecenter.nl}}
%%
%\thanks{{}$^{(8)}$ National Center for Ecological Analysis and Synthesis, Santa Barbara, CA, USA; \url{jones@nceas.ucsb.edu}}
%%
%\thanks{{}$^{(9)}$ National Evolutionary Synthesis Center, Durham, NC, USA; \url{karen.cranston@nescent.org}}
%%
%\thanks{{}$^{(10)}$ NASA Goddard Space Flight Center, Greenbelt, MD, USA; \url{Thomas.L.Clune@nasa.gov}}
%%
%\thanks{{}$^{(11)}$ Department of Astrophysical Sciences, 
%Princeton University, Princeton, NJ, USA; \url{mdevalbo@astro.princeton.edu}}
%%
%\thanks{{}$^{(12)}$ University of Saarland, Germany; \url{richard.littauer@gmail.com}}
%%
 

\begin{abstract}
\todo{need to update this abstract for WSSSPE3}
This technical report records and discusses the Second Workshop on Sustainable
Software for Science: Practice and Experiences (WSSSPE2). 
%The workshop used an
%alternative submission and peer-review process, which led to a set of papers
%divided across five topic areas: 
The report includes a description of the alternative, experimental submission
and review process, two workshop keynote presentations, a series of lightning
talks, a discussion on sustainability, and five discussions from the topic areas
of exploring sustainability; software development experiences; credit \&
incentives; reproducibility \& reuse \& sharing; and code testing \& code
review. For each topic, the report includes a list of tangible actions that were
proposed and that would lead to potential change.
%
The workshop recognized that reliance on scientific software
is pervasive in all areas of world-leading research today. The workshop
participants then proceeded to explore different perspectives on the concept of
sustainability. Key enablers and barriers of sustainable scientific software
were identified from their experiences. In addition,
recommendations with new requirements such as software credit files and software
prize frameworks were outlined for improving practices in sustainable software
engineering.
%
There was also broad consensus that formal
training in software development or engineering was rare among the
practitioners. Significant strides need to be made in building a sense of
community via training in software and technical practices, on increasing their
size and scope, and on better integrating them directly into graduate education
programs.
%
Finally, journals can define and publish policies to improve reproducibility, whereas
reviewers can insist that authors provide sufficient information and access to
data and software to allow them reproduce the results in the paper. Hence a list of
criteria is compiled for  journals to provide to reviewers so as to make it easier to
review software submitted for publication as a ``Software Paper.''

\end{abstract}


\maketitle
%\newpage

%%%%%%%%%%%%%%%%%%%%%%%%%%%%%%%%%%%%%%%%%%%%%%%%%%%%%%%%%%%%
\section{Introduction} \label{sec:intro}
%%%%%%%%%%%%%%%%%%%%%%%%%%%%%%%%%%%%%%%%%%%%%%%%%%%%%%%%%%%%

%\katznote{example comment by Dan}
%
%\gabnote{example comment by Gabrielle}
%
%\nchnote{example comment by Neil}
%
%\manishnote{example comment by Manish}
%
%\davidnote{example comment by David}

The Third Workshop on Sustainable Software for Science: Practice and Experiences
(WSSSPE3)\footnote{\url{http://wssspe.researchcomputing.org.uk/wssspe3/}} was
held on 28--29 September 2015 in Boulder, Colorado, USA. Previous events in the
WSSSPE series are
WSSSPE1\footnote{\url{http://wssspe.researchcomputing.org.uk/wssspe1/}}~\cite{WSSSPE1-pre-report,WSSSPE1},
held in conjunction with SC13,
WSSSPE1.1\footnote{\url{http://wssspe.researchcomputing.org.uk/wssspe1-1/}}, a
focused workshop organized in July 2014 jointly with the SciPy
conference\footnote{\url{https://conference.scipy.org/scipy2014/participate/wssspe/}},
WSSSPE2\footnote{\url{http://wssspe.researchcomputing.org.uk/wssspe2/}}~\cite{WSSSPE2-pre-report,WSSSPE2},
held in conjunction with SC14, and
WSSSPE2.1\footnote{\url{http://wssspe.researchcomputing.org.uk/wssspe2-1/}}, a
focused workshop organized in July 2015 again jointly with
SciPy\footnote{\url{http://scipy2015.scipy.org/ehome/115969/286469/}}.

Progress in scientific research is dependent on the quality and accessibility of
software at all levels. Hence it is critical to address challenges related to
the development, deployment, maintenance, and overall sustainability of reusable
software as well as education around software practices. These challenges can be
technological, policy based, organizational, and educational, and are of
interest to developers (the software community), users (science disciplines),
software-engineering researchers, and researchers studying the conduct of
science (science of team science, science of organizations, science of science
and innovation policy, and social science communities). The WSSSPE1 workshop
engaged the broad scientific community to identify challenges and best practices
in areas of interest to creating sustainable scientific software. WSSSPE2
invited the community to propose and discuss specific mechanisms to move towards
an imagined future practice for software development and usage in science and
engineering, but WSSSPE2 didn't have a good way to enact those mechanisms, or to
encourage the attendees to follow through on their intentions.

The workshop included multiple mechanisms for participation and encouraged team
building around solutions. It strongly encouraged participation of early-career
scientists, postdoctoral researchers, and graduate students, with funds provided
to the conference organizers by the Moore Foundation, the National Science
Foundation (NSF), and the Software Sustainability Institute, to support the
travel of potential participants who would not otherwise be able to attend,
early-stage researchers, and those from underrepresented groups. These funds allowed
16 additional participants to attend.

This report is based on collaborative notes taken with during the workshop,
which were linked from the GitHub issues that represented the potential and
actual working
groups\footnote{\url{https://github.com/issues?q=label\%3A\%22WSSSPE3+activity\%22}}.
Overall, the report discusses the organization work done before the workshop
(\S\ref{sec:preworkshop}); the keynotes (\S\ref{sec:keynote}); a series of
lightning talks (\S\ref{sec:lightning}), intended to give an opportunity for
attendees to quickly highlight an important issue or a potential solution. The
report also gives summaries of action plans proposed by the eleven working
groups (\S\ref{sec:WGs}), and some conclusions (\S\ref{sec:conclusions}).
Lists of the organizing committee (Appendix~\ref{sec:orgcom}), the registered
attendees (Appendix~\ref{sec:attendees}), and the travel award recipients
(Appendix~\ref{sec:awardees}) are provided. Finally, the report includes longer
descriptions of the activities that
occurred in each of the working groups that made substantial progress 
(Appendices~\ref{sec:appendix_best_practices}--\ref{sec:appendix_user_community}).

%%%%%%%%%%%%%%%%%%%%%%%%%%%%%%%%%%%%%%%%%%%%%%%%%%%%%%%%%%%%
\section{Calls for Participation} \label{sec:preworkshop}
%%%%%%%%%%%%%%%%%%%%%%%%%%%%%%%%%%%%%%%%%%%%%%%%%%%%%%%%%%%%

WSSSPE3 was based on the work done in WSSSPE1 and WSSSPE2, but aimed at starting
a process to make progress in sustainable software, as the calls for
participation said:

\begin{quote} The WSSSPE1 workshop engaged the broad scientific community to
identify challenges and best practices in areas relevant to sustainable
scientific software. WSSSPE2 invited the community to propose and discuss
specific mechanisms to move towards an imagined future practice of software
development and usage in science and engineering. WSSSPE3 will organize
self-directed teams that will collaborate prior to and during the workshop to
create vision documents, proposals, papers, and action plans that will help the
scientific software community produce software that is more sustainable,
including developing sustainable career paths for community members. These teams
are intended to lead into working groups that will be active after the workshop,
if appropriate, working collaboratively to achieve their goals, and seeking
funding to do so if needed. \end{quote}

The first call for participation requested lightning talks, where authors could
make a brief statement about work that either had been done or was needed, with
the goal of contributing to the discussion of one or more working groups. There
was 24 lightning talks submitted, and, after a peer-review process, 16 of these
were accepted, as discussed further in Section~\ref{sec:lightning}.

The first call also discussed the potential action topics that came out of
WSSSPE2, and requested additional suggestions. The combination of existing and
new topics led to the following 18 potential topics that were advertised in the
call for participation:


\begin{quote}
\begin{itemize}
\renewcommand{\labelenumi}{\textbf{\theenumi}.}
\setlength{\rightmargin}{1em}

\item Development and Community
\begin{itemize}
\item Writing a white paper/review paper about best practices in developing
sustainable software
\item Documenting successful models for funding specialist expertise in software
collaborations
\item Creating and curating catalogs for software tools that aid sustainability
(perhaps categorized by domain, programming languages, architectures, and/or
functions, e.g., for code testing, documentation)
\item Documenting case studies for academia/industry interaction
\item Determining effective strategies for refactoring/improving legacy
scientific software
\item Determining principles for engineering design for sustainable software
\item Create a set of guidance giving examples of specific metrics for the
success of scientific software in use, why they were chosen, what they are
useful to measure, and any challenges/pitfalls; then publish this as a white
paper
\end{itemize}

\item Training
\begin{itemize}
\item Writing a white paper on training for developing sustainable software, and
coordinating multiple ongoing training-oriented projects
\item Developing curriculum for software sustainability, and ideas about where
such curriculum would be presented, such as a summer training institute
\end{itemize}

\item Credit
\begin{itemize}
\item Hacking the credit and citation ecosystem (making it work, or work better,
for software)
\item Developing a taxonomy of contributorship/guidelines for including software
contributions in tenure review
\item Documenting case studies of receiving credit for software contributions
\item Developing a system of awards and recognitions to encourage sustainable software
\end{itemize}

\item Publishing
\begin{itemize}
\item Developing a categorization of journals that publish software papers
(building on existing work), and case studies of alternative publishing
mechanisms that have been shown to improve software discoverability/reuse e.g.,
popular blogs/websites
\item Determining what journals that publish software paper should provide to
their reviewers (e.g., guidelines, mechanisms, metadata standards, etc.)
\end{itemize}

\item Reproducibility and Testing
\begin{itemize}
\item Building a toolkit that could allow conference organizers to easily add a
reproducibility track
\item Documenting best practices for code testing and code review
\end{itemize}

\item Documentation
\begin{itemize}
\item Develop landing pages on the WSSSPE website (or elsewhere) that enable the
community to easily find up-to-date information on a WSSSPE topic (e.g.,
software credit, scientific software metrics, testing scientific software)
\end{itemize}

\end{itemize}
\end{quote}

%%%%%%%%%%%%%%%%%%%%%%%%%%%%%%%%%%%%%%%%%%%%%%%%%%%%%%%%%%%%
\section{Keynote \note{Lead: S.-C. Choi.  See \href{http://tinyurl.com/qbbqgsj}{slides} \& \href{http://tinyurl.com/q45kfcn}{video}.}} \label{sec:keynote} 
%%%%%%%%%%%%%%%%%%%%%%%%%%%%%%%%%%%%%%%%%%%%%%%%%%%%%%%%%%%%

WSSSPE3 began with a keynote speech delivered by Professor Matthew Turk from the
Department of Astronomy, University of Illinois titled \emph{Why Sustain
Scientific Software?}. Turk is a prolific scientific software practitioner and
has extensive experiences working on large collaborative projects employing
modern computing tools. He also co-organizes and champions WSSSPE events.

In his keynote address, Turk recapped the course of development of WSSSPE
workshops over the past few years, alongside his career development from a
postdoc to an academic. The first WSSSPE workshop was at the Supercomputing
conference (SC13) in 2013, but he observed that the notion of sustainable
scientific software drew in an audience beyond supercomputing. In the following
year, WSSSPE1.1 at SciPy had speakers talking about how software has been
sustained inside the scientific Python community. WSSSPE2 at SC14 had breakout
group discussions coming up with actionable items, and WSSSPE2.1 at SciPy 2015
was similar. Turk noted the different atmosphere of the surrounding large
conferences, despite similar WSSSPE participants.

WSSSPE3 left the traditional Supercomputing Conference environment this year,
and in Turk's words, this spoke to the fact that scientific software comes from
many different types of inquiries, deployment, strategies for maintenance,
users, and ways of measuring the value of a piece of software. It appeared to
Turk that the supercomputing community generally adopt some top-down approaches,
whereas the SciPy community more often than not uses more bottom-up systems. The
essential messages perceived were also often bipolar: the supercomputing
community thinks software is getting harder, with exascale computing and
optimization issues in mind, but the SciPy community thinks software becoming
better, with emerging tools such as Jupiter and productivity packages for
research workflow. Admitting such comparisons are somewhat unfair
generalizations, Turk reminded the audience that the different approaches bring
different types of ideas to the table, and he welcomed WSSSPE3 being conducted
outside existing preconceptions.

Returning to the topic of his talk, Turk invited the audience to picture
scientific software as a flower on a landscape under the Sun, which may
represent a number of measurable factors such as number of citations; growth of
a community and number of contributors; amount of funding; prestigious prizes
awarded; stability of the community in terms of leadership transitions, serving
community needs, not breaking test suites, and performance on new architectures.
But all these metrics are strictly speaking \emph{proxies} for the values and
the impact scientific software bears. What we can measure does not give us
direct insight---it just gives us proxies of insight.
  
Turk then moved onto various different definitions of sustainability. His
favorite one was `keeping up with bug reports,' where even if no new features
were added, the software remains sustainable. Another definition of
sustainability Turk mentioned was `adding of new features', or `maintaining the
software for a long period of time' such as the cases of TeX or LaTeX with
community help. A notion Turk heard often at supercomputing conferences was that
sustainable software `continues to work on new architectures'. Yet another
metric was `people continuing to be able to learn how to use and apply the
software'. A funder Turk heard talked about sustainability as `continuing to get
funded'. Turk also recalled that Greg Wilson, among others, said in WSSSPE1.1
that his view of sustainable software was software that `continued to give the
same results over time'. A last measure of sustainability Turk presented was
`the ability to transition between different people developing and using a piece
of software.'

At WSSSPE1, several models were presented for ensuring sustainability. Turk
considered that a familiar one was a funded piece of software where an external
agency provided funds to a group who are not necessarily exclusively working on
and developing a piece of software, keeps it going, and provides it to the
scientific community. The model of productized software, in which a piece of
software has grown into something that individuals rely on to the point that
research groups or people are willing to support it with some amount of funding,
for instance, a subscription to use cloud services that deploy a piece of
software, or purchase of a piece of software. A final model Turk felt conflicted
about is a volunteer model that is traditional old-school---not modern-day open
source---development.

Turk discussed whether productizing scientific software was synonymous with
being sustainable and self-sufficient. He thought it was not necessarily the
case and furthermore, it could lead to a divergence of interest between users
and developers.

Turk reminded the audience that the volunteer model means unpaid labor. On this
note, he recommended Ashe Dryden's blog post on the
ethics of unpaid labor and the open source software
community\footnote{http://www.ashedryden.com/blog/the-ethics-of-unpaid-labor-and-the-oss-community}.
Often times, a person
funded to work full time on a scientific project can spend a small amount of
time for working on a piece of software necessary for that project. However, all
researchers' ability to participate in that volunteer community are not always
the same and may not always be aligned with their research projects. From Turk's
experience, we cannot always rely on unpaid labor and volunteer time to sustain
a piece of software---this came down to the notions of the top-down and the
bottom-up approaches, i.e., the funded versus the grassroots. However, Turk
pointed out that bottom-up, volunteer-driven projects can be just as large
as a top-down software development project.

Turk said that sustaining scientific software really meant to him conducting
scientific inquiries, often by some specific software, and sustaining the people
we care about, our careers, and the future of our fields. According to Turk, we
all have an invested stake in sustaining scientific software. Hence, having
``sustained'' projects can suffocate new projects, so we need to make sure
we don't cause novel ideas and packages to suffer at the hands of the status quo.
%This is a straw man concept, but it's something to look out for, since organizations will act in their own interest.

Turk talked about possible reasons why we want to sustain scientific
software: devotion to science and interests in pursuing the next stage of
research; fun and creative thrill in writing codes and papers; usefulness with
measurable impacts, for example, LINPACK and HDF groups providing of data
storage to satellites, which goes beyond usefulness to necessity. Lastly, Turk
presented his wishlist of questions to be answered in the future: 
%
\begin{itemize} 

\item How do we ship a product on time when dealing with a mix of funding models
and motivations especially when we rely on volunteers?

\item How do we know when it is time to end some software and move on? For
example, should we stop sustaining Python and switch to Julia and Javascript?
 
\item How can productized software balance its future versus its past, or the new
needs of the customers versus the existing needs of the development community?

\item How can we help avoid burnout and retain the joy in the communities?

\item How can we reduce systemic bias, which goes back to the blog post of
Dryden especially on how ethics of unpaid labor disproportionately affect
underrepresented communities?

\end{itemize}

%%%%%%%%%%%%%%%%%%%%%%%%%%%%%%%%%%%%%%%%%%%%%%%%%%%%%%%%%%%%
\section{Lightning Talks} \label{sec:lightning}
%%%%%%%%%%%%%%%%%%%%%%%%%%%%%%%%%%%%%%%%%%%%%%%%%%%%%%%%%%%%
\todo{need to update for WSSSPE3}
\todo{KEN: Should these be bolded? - dsk: yes, titles can be bolded, but contents under each should not}
\begin{comment}
\note{
\href{http://wssspe.researchcomputing.org.uk/wssspe3/agenda/}{Slides.}}
\end{comment}

\begin{enumerate}
\item \textbf{Benjamin Tovar and Douglas Thain: \textit{Freedom vs. Stability:
Facilitating Research Training While Supporting Scientific Research}}
A case study: ``The Cooperative Computing Lab'' at the University of Notre Dame
is a small group of individuals whose main tasks are collaborating with
people that have large-scale computing problems, operating various parallel
computer systems, conducting computer science research, and developing open
source software. One of the main challenges they face is finding the balance
between flexibility/training and stability/quality. Their current solution for
ensuring the latter was to add a software engineer (the presenter) to the
existing team of faculty and students, who now also serves as ``spring''
between flexibility and stability.

\item \textbf{Birgit Penzenstadler, Colin Venters, Christoph Becker, Stefanie
Betz, Ruzanna Chitchyan, Let\'{i}cia Duboc, Steve Easterbrook, Guillermo
Rodriguez-Navas and Norbert Seyff: \textit{Manifesting the Ghost of the Future:
Sustainability}}
The concept of sustainability has become a topic of interest in the field of
computing, which is evidenced by the increase in the number of events that
focus on the topic. Nevertheless, it isn't well understood yet. Penzenstadler
argued that we often define sustainability too narrowly. Instead,
sustainability at its heart is a systemic concept and must be viewed from a
range of different dimensions including environmental, economic, individual,
social and technical. She introduced the Karlskrona Manifesto on Software
Design~\cite{Becker:2014}, which distills knowledge from a broad range of related work on the
topic of sustainability into a set of (mis-)perceptions and principles. The
manifesto does not proclaim that there is an easy, one-size-fits-all solution
around the corner, but rather points out that sustainability is a ``wicked
problem'' and is often misunderstood. Due to these misperceptions, even though
sustainability's importance is increasingly recognized, many software systems
are unsustainable. Even more alarming is that most software systems' broader
impacts on sustainability are unknown. To change this, the Karlskrona Manifesto
proposes nine principles and commitments. These commitments are not dogmatic
laws, but rather a commitment to rethink, to move beyond the silo mentality,
and analyze in more depth. As such, they do not restrict, but rather open up a
space for discussion.

\item \textbf{Abani Patra, Hossein Aghakhani, Nikolay Simakov, Matthew D. Jones
and Tevfik Kosar: \textit{Integrating New Functionality Using Smart Interfaces to
Improve Productivity of Legacy Tools}}
Abani Patra presented an example how the community using Titan2d, a Geoflow
Simulation Software, increased the productivity of their tools by improving
both code and data layout. Main obstacles in this change were the non-existence
of a common version control system for the source code, coupled with multiple
versions of the same code base, the fixed format of input files, that many
input values used to be set as compilation flags, and that the internal data
layout was not suitable for modern technologies (vectorization, accelerators).
The approach of the Titan2d developers included reinforcing the code structure
using a multiple laters of Python and C++, and a redesign of the data layout to
be better suitable for modern CPUs and accelerators.

\item \textbf{Abigail Cabunoc Mayes, Bill Mills, Arliss Collins and Kaitlin
Thaney: \textit{Collaborative Software Development as Sustainable Software: Lessons
from Open Source}}

\item \textbf{Louise Kellogg and Lorraine Hwang: \textit{Advancing Earth Science
through Best Practices in Open Source Software: Computational Infrastructure
for Geodynamics}}

\item \textbf{Lorraine Hwang, Joe Dumit, Alison Fish, Louise Kellogg, Mackenzie
Smith and Laura Soito: \textit{Software Attribution for Geoscience Applications in the
Computational Infrastructure for Geodynamics}}

\item \textbf{Mike Hildreth, Jarek Nabrzyski, Da Huo, Peter Ivie, Haiyan Meng,
Douglas Thain and Charles Vardeman: \textit{Data And Software Preservation for Open
Science (DASPOS)}}

\item \textbf{James Hetherington, Jonathan Cooper, Robert Haines, Simon
Hettrick, James Spencer, Mark Stillwell, Mike Croucher, Christopher Woods and
Susheel Varma: \textit{Research Software Engineering Groups in Universities: The Story
from the UK}}

\item \textbf{Dan Gunter, Sarah Poon and Lavanya Ramakrishnan: \textit{Bringing the
User into Building Sustainable Software for Science}}

\item \textbf{Dan Gunter, Adam Arkin, Rick Stevens, Robert Cottingham and
Sergei Maslov: \textit{Challenges of a Sustainable Software Platform for Predictive
Biology: Lessons Learned on the KBase Project}}

\item \textbf{Yolanda Gil, Chris Duffy, Chris Mattmann, Erin Robinson and Karan
Venayagamoorthy: \textit{The Geoscience Paper of the Future Initiative: Training
Scientists in Best Practices of Software Sharing}}

\item \textbf{Neil Chue Hong: \textit{Building a Scientific Software Accreditation
Framework}}

\item \textbf{Jeffrey Carver: \textit{On the Need for Software Engineering Support for
Sustainable Scientic Software}}
Jeff argued that for scientific software to be truly sustainable, there is a need
for developers to use appropriate software engineering practices. His experience
interacting with scientific teams indicates that choosing and tailoring these
practices is not a trivial exercise. There is a general culture clash between
software engineering and science that hinders our ability to communicate and
choose appropriate methods. In addition, many experience scientific software
developers appear to be unaware of software engineering practices that may be
beneficial to them. The most appropriate software engineering practices are
those that are lightweight, properly tailored, and focus on the key software
development problems faced by scientists. In order to increase the use of
software engineering in science, we need more documented success stories. These
successes need to be socialized within the scientific community through
workshops like the Software Engineering for Science workshop
series\footnote{\href{http://www.SE4Science.org/workshops}{http://www.SE4Science.org/workshops}}
and the new Software Engineering track in
\textit{Computing in Science and Engineering} magazine.

\item \textbf{Matthias Bussonnier: \textit{User Data Collection in Open Source}}

\item \textbf{Alice Allen: \textit{We're giving away the store! (Merchandise not
included)}}

\item \textbf{Stan Ahalt, Bruce Berriman, Maxine Brown, Jeffrey Carver, Neil
Chue Hong, Allison Fish, Ray Idaszak, Greg Newman, Dhabaleswar Panda, Abani
Patra, Elbridge Gerry Puckett, Chris Roland, Douglas Thain, Selcuk Uluagac, and
Bo Zhang: \textit{Scientific Software Success: Developing Metrics While Developing
Community}}

\end{enumerate}

%%%%%%%%%%%%%%%%%%%%%%%%%%%%%%%%%%%%%%%%%%%%%%%%%%%%%%%%%%%%
\section{Working Groups} \label{sec:WGs}
%%%%%%%%%%%%%%%%%%%%%%%%%%%%%%%%%%%%%%%%%%%%%%%%%%%%%%%%%%%%

\todo{go through subsections that have both content here and appendix, and add pointer from text in the subsection here to the corresponding appendix - done in line for Software Credit group already}

\subsection{White paper/journal paper about best practices in developing sustainable software}
\label{sec:best-practices}

\subsubsection{Why it is important}
\todo{short text here}


\subsubsection{Fit with related activities}
\todo{short text here - can include links/cites}

\subsubsection{Discussion}
\todo{short-ish text here}

\subsubsection{Plans}
\todo{short text here - not bullets}

[15 Nov] Introduction and scope finished
[15 Nov] Sections assigned
[31 Jan] Analysing funding possibilities for survey
[31 Jan] First versions of section
[15 Feb] Distribution to WSSSPE community
[31 Mar] Final version of white paper
[30 Apr] Submission of peer-reviewed paper?


\begin{enumerate}
\item Introduction and Scope of White Paper 
\item Related Work
\item Case Studies
\begin{enumerate} 
\item PeTSC
\item NWCHEM
\item CIG
\end{enumerate}
\item Community Related Practices
\begin{enumerate} 
\item Findings
\item Recommendations
\end{enumerate}
\item Governance and management
\begin{enumerate} 
\item Findings
\item Recommendations
\end{enumerate}
\item Funding Related
\begin{enumerate} 
\item Findings
\item Recommendations
\end{enumerate}
\item Metrics for sustainability
\item Tools
\item Conclusions
\end{enumerate}

\subsubsection{Landing Page}
\todo{link to landing page}

\subsection{Funding Research Programmer Expertise}
\label{RSE}

\katznote{I don't think this section title is exactly right - are there some other options?  Also, the title of Appendix~\ref{sec:appendix_funding_spec_expert} should match, whatever is chosen.  Maybe `Funding Research Programmer Expertise'?}

%\subsubsection{Why it is important}

Research Software Engineers -- those who contribute to science and scholarship
through software development -- are an important part of the team
needed to deliver 21st century research. However, existing academic structures
and systems of funding do not effectively fund and sustain these skills.
The resulting high levels of turnover and inappropriate incentives
are a significant contributing factor to low levels of reliability and
readability observed in scientific software. Moreover, the absence of skilled and experienced 
developers retards progress in key projects, and at times causes important projects to fail completely.

Effective development of software for advanced research requires
that researchers work closely with scientific software developers who understand
the research domain sufficiently to build meaningful software at a reasonable pace. 
This requires a collaborative approach -- where developers who are fully engaged/invested 
in the research context are co-developing software with domain academics.

\subsubsection{Fit with related activities}

The solution we envision entails creating an environment where software developers 
are a stable part of the research team. Such an environment mitigates the risk
of losing a key developer at a critical moment in a projects lifetime, and provides
the benefits of building a store of institutional knowledge about specific projects as well as about software 
development for today's research. Our vision is to find a way to promote a University/research institute environment where
software developers are stable components of research project teams. 

One strategy to promote stability is implementing a mechansim for developers to obtain academic
credit for software development work. With such a mechanism in place, traditional academic funding
models and career tracks could properly sustain individuals for whom software development is their
primary contribution to research. A contributing factor to the problem with the current academic reward system is the
devastating effect on an academic
publication record resulting from time in industry; such postings often develop exactly the skills that research software
engineers need, yet returns to university positions following an industry role are penalized by the current structures.
Retention of senior developers is hard, because these people are highly in demand by the economy. However, people who have a
PhD in science and enter industry, may desire to return for diverse reasons, and should be welcomed back.

While development of new mechanisms in the current academic reward system is a worthy aspirational goal, such a dramatic
change in this structure does not seem likely in a time scale relevant to this working group. Accordingly, our working party
sought alternate solutions that may be achievable within the context of existing academic structures. The group felt that
developing dedicated research software engineering roles within the University, and finding stable funding for those individuals is the most promising mechanism for creating a stable software development staff.

Measures of impact and success for research programming groups, as well as for individual research software engineers, will
be required in order to make the case to the University for continued funding. Research software engineers will not be measured by publications, we hope, but by other measures. (what measures?) Middle-author publications are common for RSEs. Most RSEs welcome co-authorship on papers where the PIs consider the contribution deserves it.
(these last two statements seem contradictory, not sure which message is to be used?)

\subsubsection{Discussion}

It is hard for an individual PI in a university or college to support dedicated research software engineering resources, as
the need for, and funding for, these activities is intermittent within the research cycle. To sustain this capacity, therefore, it is necessary to aggregate this work across multiple research groups.

One solution is to fund dedicated software engineering roles for major research software projects at national laboratories
or other non-educational institutions. This solution is in place and working well for many well-used scientific codebases.
However, this strategy has limited application, as much of the body of software is created and maintained in research
universities. Therefore, we argue that research institutions should develop hybrid academic-technical tracks for this
capacity, where employees in this track work with more than one PI, rather than the traditional RA role within a single group
. This could be coordinated centrally, as a core facility, perhaps within research computing organizations which have
traditionally supported university cyberinfrastructure, libraryorganizations, or research offices. Alternatively, these
groups could be organizationally closer to research groups, sitting within academic departments. The most effective model
will vary from institution to institution, but the mandate and ways of working should be similar.

Having convinced ourselves that this would be a positive innovation, we were then faced with the specific question of how to
fund the initiation of this activity. A self-sustaining research software group will support itself through collaborations
with PIs in the normal grant process, with PIs choosing to fund some amount of research software engineering effort through grants in
the usual way. However, to bootstrap such a function to a level where it has sufficent reputation and client base to be self
-sustaining will generally require seed investment.

This might come from universities themselves (this was the model that led to the creation of the group in University College
London), but more likely, seed funding needs to come from research councils (as with the Research Software
Engineering Fellowship provided by the UK Engineering and Physical Sciences Research Council). We therefore recommend that
funding organizations consider how they might provide such seed funding.

Success, appropriately measured, will help make the case to such funding bodies for further investment. One might expect that metrics such as improved productivity, software adoption rates, and grant success rates would be sufficient arguments in favor of such a model. However, useful measurement of code cleanliness, and the resulting productivity gains, is an unsolved problem in empirical software engineering. To measure ``what did not go wrong'' because of an intervention is particularly hard. 

We finally noted that the institutional case for such groups is made easier by having successful examples to point to. In the
UK, a collective effort to identify the research software engineering community, with individuals clearly stating "I'm a research software engineer", has been important to the campaign. It will be useful to the global effort to similarly identify emerging research software organizations, and also, importantly, to identify longer-running research software groups, which have in some cases had a long running \emph{sui-generis} existence, but which now can be identified as part of a wider solution. There remains the problem of how to "sell" the value of this investment to investigators within the university. This is an issue best addressed by the individual organizations that embark on the plan. 

For more detail on the discussion, see Appendix~\ref{sec:appendix_funding_spec_expert}.

\subsubsection{Plans}

The first step in moving this strategy forward is to gather a list of groups that self-identify as research software engineering groups, and to reach out to other organizations to see if there may be a widespread community of RSE's who do not identify themselves as such at this time. We will collect information as to the organizational models under which these groups function, and how they are funded. For example, how many research universities currently fund people in the RSE track, whether they bear the the RSE moniker or not. Are these developers paid by the University or through a program supported by research grants/individual PIs? How did they bootstrap the developer track to get this started? How successful is the university in getting investigators to pay for fractional RSEs?  We will author a report describing our findings, should funding be available to conduct the investigation. 

\subsubsection{Landing Page}

To find more information about this group, or join it, see \url{http://www.rse.ac.uk/groups}.

%\input{wg_main_catalogs}
\subsection{Transition Pathways to Sustainable Software: Industry \& Academic Collaboration} 

%\subsubsection{Why it is important}

Most scientific software is produced as a part of grant-funded research projects
sponsored by federal governments. If we are interested in the sustainability of
scientific software, then we need to understand what exactly happens when that
sponsorship ends. More than likely, the project and its resulting software will
need to undergo some kind of transition in funding and consequently management.

There are a number of potential funding transitions that may occur:  
%
\begin{itemize}

\item A project could be \textbf{refunded}, and development or maintenance of
the software continue as planned.

\item A project might locate a \textbf{new source of funding} in which case the
software may be further developed or simply maintained as before.

\item The project could transition to a \textbf{community supported model}
whereby ownership, maintenance, and stewardship of the software become similar
to peer-production models in open-source (e.g., see
Howison~\cite{howison_sustaining_2015}).

\item The project could receive some form of industry sponsorship in which case
ownership of the intellectual property, licensing, maintenance activities,
hosting, etc.\ may change significantly.

\end{itemize}

We characterize each of these potential changes in funding as ``transition
pathways'' to sustainable software (see similar work by Geels and
Schot~\cite{Geels:2007}).

At WSSSPE3 our group was interested in better understanding successful pathways
for scientific software to ``transition'' from government-funded research
projects to industry sponsorship. (This may be an initially awkward
phrase---some software projects will begin their life being sponsored by
industry, or result in collaboration between industry and academia. In such
cases, there is still a need to understand how IP and how maintenance of the
software is sustained over time.)

Although transitions are often studied under the broad umbrella of ``technology
transfer,'' we believe there are likely to be a number of different ways in which
a pathway from initial production to long-term maintenance and secure funding are
achieved. In short, industry sponsorship is an important aspect of sustaining
scientific software, but our current understanding of these transitions focuses
narrowly on commercial successes\slash failures.

\subsubsection{Fit with related activities}

In looking at existing literature that addresses industry transitions, many
reports focus on benefits that accrue to the private sector, or to a government
that originally sponsored the research project. This literature does not address
the impact that these transitions have on the accessibility or usability of the
software, or the impact that these transitions have on the career of the
researchers involved.

Examples of the former scenario (benefits accruing to private sector) are as
follows:
\begin{itemize}

\item REF Impact Case Studies: \url{http://impact.ref.ac.uk/CaseStudies/}

\item Background of projects funded in the UK: \url{http://gtr.rcuk.ac.uk/}

\item Dowling Review from the UK: addresses complexity of work between these two
communities: \url{http://www.raeng.org.uk/policy/dowling-review}

\item Pathway to Impact - UK report: two pages of grant proposals are asked to
forecast what impact they might have (including environmental, academic, economic).

\end{itemize}

\subsubsection{Discussion}

Our work at WSSSPE3 included the following activities (described in detail
below): (1) brainstorming goals for this type of research, (2) imagining
potential outcomes of completing a set of case studies on this topic, and (3)
generating a set of working definitions for some of the broad concepts we are
describing.

\textbf{Goal}

\emph{What is the goal of doing research on transition pathways?} 

Can we identify collaborations that have occurred and try to understand which were successful, which were unsuccessful, and what factors contributed to these successes/failures? 

Can we determine what each partner wants to get out of such a collaboration?
For example, why would industry be interested in collaborating with academia? 
Or why would academia be interested in collaborating with industry?

How could we design a study that focused on the impact of the software in
undergoing this type of transition?

\textbf{Potential outcomes}

A set of case studies that look at successful and unsuccessful
transitions of researchers between academia and industry. This might address 
each of the transition types (described below). Successful transitions are
described as those that lead to either weak or strong sustainability (also
defined below).

Create a generalizable framework that might allow for the study of different
transition pathways (other than academia to industry).

\textbf{General Definitions}

We characterize transitions in the following ways:
\begin{itemize}

\item Handoff model: academia initially writes the software, industry (for-profit 
or nonprofit) then takes over the project.

\item Co-Production Model: industry and academia interact throughout development
of the project.

\item Sponsorship Model: academia writes and maintains the software; 
industry contributes funding for the development\slash maintenance of software.
In this example, industry is also likely a user of the software.

\item Spinoff model: transition to a for-profit or non-profit company owned by or in
collaboration with original developers.

\end{itemize}

We characterize sustainability in the following ways:
\begin{itemize}

\item Weak Sustainability: Software continues to be accessible, useful, and
usable.

\item Strong Sustainability: Software meets criteria above, but is also able to
be reused for further innovation (i.e., issued non-restrictive open-source
license).

\end{itemize}
We refer readers to Becker et al.~\cite{Becker:2014} for an extended discussion of weak versus strong
sustainability. For more detail on the group's discussion, see
Appendix~\ref{sec:appendix_industry_interaction}.

\subsubsection{Plans}

Plans for carrying forward are currently unclear---this project would require
sustained attention and effort from our team, and at least some amount of
funding in order for us to be involved for extended periods of time.

\subsubsection{Landing Page}

To find more information about this group, or join it, see \todo{where should
someone go who want to know more about this and perhaps wants to contribute?}

\subsection{Legacy Software} \label{sec:legacy} 

This group met only briefly, for one period on the first day. They discussed
that it is difficult to define legacy code because there is so much stigma
associated with the term. At some point there will be more difficulty and
resources wasted trying to keep legacy software supported, but it will
eventually be too expensive compared to how much it would be to just rebuild the
software or kill it. Most of the group members were not able to attend on the
second day, and those who were able to attend joined other groups.


%\subsubsection{Why it is important}
%\todo{short text here}
%
%\subsubsection{Fit with related activities}
%\todo{short text here - can include links/cites}
%
%\subsubsection{Discussion}
%\todo{short-ish text here}
%
%\subsubsection{Plans}
%\todo{short text here - not bullets}
%
%\subsubsection{Landing Page}
%\todo{link to landing page}




\subsection{Principles for Software Engineering Design for Sustainable Software} 

\subsubsection{Why it is important}

Software engineering principles form the basis of methods, techniques, methodologies and tools~\cite{}. However, there is often a mismatch between software engineering theory and practice particulalry in the fields of compuational science and engineering, which can lead to the development of unsustainable software~\cite{}. Understanding and applying software engineering principles is essential in order to create and maintain sustainable software~\cite{}.

\subsubsection{Fit with related activities}
The group discussion focused on identifying existing principles of software engineering design that could be adopted by the computational science and engineering communities.

\subsubsection{Discussion}

The group included members from different backgrounds, including quantum chemistry, epidemiology, computer science, software engineering, and microscopy. Each participant was invited to give their perspective on the topic area and what they thought were the crucial points for discussion. There was a general consensus that there was a need for relating principles to practice for the computational science and engineering community. Furthermore, various members of the group expressed their interest in tools and best practices for facilitating the maintenance and evolution of scientific software systems. It was agreed to identify principles from software engineering and from sustainability design and, based on those lists, discuss what each of those would mean applied to specific example systems from the expert domains of some of the group members. The group identified a number of software engineering principles drawn from the SoftWare Engineering Body of Knowledge (SWEBOK)~\cite{swebokv3}. 

Software design principles included: Abstraction; Coupling and cohesion; Decomposition and modularization; Encapsulation and information hiding; Separation of interface and implementation; Sufficiency completeness \& primitiveness; and Separation of concerns. Similalry, user interface design principles included: Learnability; User familiarity; Consistency; Minimal surprise; Recoverability; User guidance; and User diversity. The sustainability design principles were drawn from the Karlskrona Manifesto on Sustainability Design~\cite{Becker:2014}. The maifestio states that sustainability is systemic; multidimensional; interdisciplinary; transcends the system's purpose; applies to both a system and its wider contexts; requires action on multiple levels; requires multiple timescales; changing design to take into account long-term effects doesn't automatically imply sacrifices; system visibility is a precondition for and enabler of sustainability design.
%\todo{should these sets of bullet points be moved to the appendix? or moved into paragraph form, since they are relatively short items?}
A number of sustainable software engineering principles proposed by Tate~\cite{tate2005} were also considered including: continual refinement of product and project practices; a working product at all times; continual emphasis on design; and value defect prevention over defect detection.

This congregated list is an initial collection of principles that could be extended by adding from further related work form separate disciplines within the field of software engineering, including requirements engineering, software architecture, and testing. The group identified two example systems to discuss the application of the principles. The first one was a quantum chemistry system that allows the analysis of the characteristics and capabilities of molecules and solids. The second one was a modeling system for malaria that permitted biologists to analyze a range of datasets across geography, biology, and epidemiology, and add their own datasets. The group then examined the principles and took a retrospective analysis of what the developers did in practice against how the principles could have made a difference.

\subsubsection{Plans}
The next steps in this endeavor are to (1) Systematically analyze a number of example systems from different scientific domains with regards to the identified principles, to (2) Identify the commonalities and gaps in applying those principles to different scientific systems, and to (3) Propose a set of guidelines on the principles and how they exemplary apply to scientific software system. 

\subsubsection{Landing Page}
In the absence of a landing page, the Principles for Software Engineering Design for Sustainable Software working group requests an email be sent to Birgit Penzenstadler\footnote{email: \href{mailto:birgit.penzenstadler@csulb.edu}{birgit.penzenstadler@csulb.edu}} and Colin C. Venters\ to find out more about the group's efforts and how to participate.

%%%%%%%%%%%%%%%%%%%%%%%%%%%%%%%%%%%%%%%%%%%%%%%%%%%%%%%%%%%%
\subsection{Useful Metrics for Scientific Software}
\label{sec:software-metrics}
%%%%%%%%%%%%%%%%%%%%%%%%%%%%%%%%%%%%%%%%%%%%%%%%%%%%%%%%%%%%

%\subsubsection{Why it is important}

Metrics for scientific software are important for tenure and promotion, scientific impact, discovery, reducing duplication, serving as a basis for potential industrial interest in adopting software, prioritizing develop and support towards 
strategic objectives and making a case for new or continued funding.  However, there is no commonly-used 
standard for collecting or presenting metrics, nor is it known if there is a common set of metrics for scientific software. 
 It is imperative that scientific software stakeholders understand that it is useful to collect metrics.

\subsubsection{Fit with related activities}

The group discussion  focused on identifying existing frameworks and activities for scientific software metrics.  
The group identified the following related activities:

\katznote{why are these in parens?  Can the links be added as text, so that readers of this in print can see them?  I guess as citations}
\begin{itemize}

\item
\href{https://geodynamics.org/cig/dev/best-practices/}{(Computational Infrastructure for Geodynamics: Software Development Best Practices)}

\item
\href{https://docs.google.com/document/d/1cgUDH3RxrfsLotWhKKOrXUnaYFhrtjcV1TDRkFtwQKI/edit}{(WSSSPE3 Breakout Session: How can we measure the impact of a code on research, and its value to the community?)}

\item
\href{https://docs.google.com/document/d/10yj7MYEjvrg__t522XR41ogASYMp647-l-BpFTsqEV4/edit#heading=h.5lah0hp73q99}{(2015 NSF SI2 PI Workshop Breakout Session on Framing Success Metrics)}

\item
\href{https://docs.google.com/document/d/1uDim5bw8rBuubmtaUrz5Eh35NxzDgivmmdXhVzDs3tc/edit}{(2015 NSF SI2 PI Workshop Breakout Session on Software Metrics)}

\item
\href{https://docs.google.com/presentation/d/1PPLVL6uoOmisqnHTlwhsVKJBTFFK1IVzvr8FdEEIvAE/edit#slide=id.g5e66ec9f2_027}{(NSF Workshop on Software and Data Citation Breakout Group on Useful Metrics)}

\item
\href{http://www.software.ac.uk/software-evaluation-guide}{(U.K. Software Sustainability Institute Software Evaluation Guide)}

\item
\href{http://www.software.ac.uk/blog/2013-04-09-five-stars-research-software}{(U.K. Software Sustainability Institute Blog post: The five stars of research software)}

\item
\href{http://figshare.com/articles/Minimal_information_for_reusable_scientific_software/1112528}{(Minimal information for reusable scientific software)}

\item
\href{http://equipment.data.ac.uk/}{(EPSRC-funded Equipment Data Search Site)}

\item
\href{http://www.canarie.ca/software/}{(Canarie Research Software: Software to accelerate discovery)}

\item
\href{https://science.canarie.ca/researchmiddleware/platforms/list/main.html}{(Canarie Research Software: Research Software Platform Registry)}

\item
\href{https://collaboration.canarie.ca/elgg/file/view/2471/research-platform-support-for-the-canarie-registry-and-monitoring-system-revision-3}{(collaboration@CANARIE post: platform support for Canarie registry and monitoring system)}

\item
\href{https://collaboration.canarie.ca/elgg/file/view/2453/research-service-support-for-canarie-registry-and-monitoring-system-revision-7}{(collaboration@CANARIE post: service support for Canarie registry and monitoring system)}

\item
\href{https://www.openhub.net/}{(BlackDuck Open HUB)}

\item
\href{https://www.innovationpolicyplatform.org/frontpage}{(Innovation Policy Platform)}


\end{itemize}



\subsubsection{Discussion}

The group discussion began by agreeing on the common purpose of creating a set of guidance giving examples of specific metrics for the success of scientific 
software in use, why they were chosen, what they are useful to measure, and any challenges and pitfalls; then publish this as a white paper. 
 The group discussed many questions related to useful metrics for scientific software including addressing if there is a common set of metrics that 
 can be filtered in some way, can metrics be fit into a common template, which metrics would be the most useful for each stakeholder, 
 which metrics are the most helpful and how would we assess this, how are metrics monitored, and many more.  
 A more complete bulleted list of these questions can be found in Appendix H.  Next, a roadmap for how to proceed was discussed including
 creating a set of milestones and tasks.  The idea was put forth for the group to interact with the organizing committee of the 2016 NSF Software Infrastructure
  for Sustained Innovation (SI2) PI workshop in order to send a software metrics survey to all SI2 and related awardees as a targeted and relevant set of stakeholders.  
 The five solicitations for software elements released  under the NSF SI2 program all included metrics as a required component with submitters requested to include 
 {\it "a list of tangible metrics, with end user involvement, to be used to measure the success of the software element developed, ...'}. These metrics are then reported as part of annual reports to NSF by the projects. Although neither the proposal text describing the metrics nor the reported metric results are publicly available, there is reason to believe that the community will be willing to provide this information through a survey mechanism. 
This survey would be created by one of the student group members.  Similarly, it was suggested that a software metrics survey be sent to the 
UK SFTF and TRDF software projects to ask them what metrics would be useful to report.  The remainder of the discussion focused mainly on the 
creation of a white paper on this topic.  This resulted in a paper outline and writing assignments with the goal of publishing in venues including 
WSSSPE4, IEEE CISE, or JORS. More information about the group discussion is available in Appendix~\ref{sec:appendix_metrics}.

\subsubsection{Plans}

The main plans for the group going forward are the creation of a white paper on the topic of useful metrics for scientific software.  The authoring of this white paper would happen in parallel with the creation of a survey by the group with the survey results to be incorporated in the white paper.  The timeline for completion of the white paper is approximately one year targeting venues discussed in the previous section.

\subsubsection{Landing Page}

The group will use Google Docs to proceed with the authoring of the white paper in lieu of a group project landing page.  Links to the resulting white paper will be provided when completed. \todo{where will it be provided - this isn't very useful to readers as is}

\subsection{Topic} \todo{Training Working Group Discussion}

\subsubsection{Why it is important}

This group explored a rapidly growing array of training which is seen to contribute to sustainable software. Offerings are diverse, providing training more or less directly relevant to sustinable software. While research institutions  support professional development for research staff, the skills taught which might impact on sustainable software are limited at best, often lacking a clear and coherent development pathway. This growing array of training opportunities could usefully be coordinated, by bringing together those involved in leading relevant initiatives, on a regular basis.

\subsubsection{Fit with related activities}
Two existing venues for discussion of related activities are identified:

\begin{itemize}

\item
\item
WSSSPE Workshop on Sustainable Software for Science: Practice and Experiences \footnote{\url{http://wssspe.researchcomputing.org.uk/}}

\item
SEHPCCSE International Workshop on Software Engineering for High Performance Computing in Computational Science and Engineering \footnote{\url{http://wssspe.researchcomputing.org.uk/}}

\end{itemize}

\subsubsection{Discussion}

Next steps have been identified to quickly test whether there is interest in establishing a community committed to increasing the degree of coordination across training projects.

See Appendix~\ref{sec:appendix_training} for more details about the discussion.

\subsubsection{Plans}

The main plans for the group are to convene discussion to explore bringing together a regular meeting of those involved in leading relavent training projects.

\subsubsection{Landing Page}

The Training working group requests an email be sent to Nick Jones \href{mailto:nick.jones@nesi.org.nz}{(nick.jones@nesi.org.nz)} to find out more about the group's efforts and how to participate.

%%%%%%%%%%%%%%%%%%%%%%%%%%%%%%%%%%%%%%%%%%%%%%%%%%%%%%%%%%%%
\subsection{Software Credit Working Group}
\label{sec:software-credit}
%%%%%%%%%%%%%%%%%%%%%%%%%%%%%%%%%%%%%%%%%%%%%%%%%%%%%%%%%%%%

\subsubsection{Why it is important}
\todo{short text here}

\subsubsection{Fit with related activities}
\todo{short text here - can include links/cites}

Publishing Software Working Group (\S\ref{sec:publishing-software})

\subsubsection{Discussion}
\todo{short-ish text here}

\subsubsection{Plans}
\todo{short text here - not bullets}

\subsubsection{Landing Page}
\todo{link to landing page}


\subsection{Publishing Software Working Group Discussion}

\subsubsection{Why it is important}

This group explored the value of executable papers (papers whose content includes
the code needed to produce their own results), and other forms of publishing which
include dynamic electronic content. Transitioning to this type of publication offers
possibilities of addressing, or partially addressing sustainability concerns 
such as reproducibility (the paper contains all the artifacts needed to verify its
results), transitive credit (modules an executable paper depends on must be explicitly
loaded, making it more feasible to identify them), and improving documentation (an executable
paper must explain what its code does).

Reproducibility: Part of the purpose of these venues is to (at least partially)
address the reproducibility issue by making the paper itself recompute its own
results.

Transitive Credit: Since these forms of publishing must make their sources explicit,
they should be easier to trace even if appropriately worded credit for software
is not provided. In addition, these notebooks make it possible to provide/define
additional metadata to make the tracing of credit clearer. In addition, attributions
could be added to citations to identify whether a paper extends a result, verifies it,
contradicts it, etc.

Best Practices: Because an executable paper showcases the code 

\subsubsection{Fit with related activities}
\todo{short text here - can include links/cites}

\subsubsection{Discussion}
\todo{short-ish text here}

\subsubsection{Plans}
\todo{short text here - not bullets}

\subsubsection{Landing Page}

A page will be made available on the Software Sustainability Institute website.

\subsection{Building Sustainable User Communities for Scientific Software}
%\todo{change the title to your topic}

\subsubsection{Why it is important}
%\todo{short text here}

User communities are the lifeblood of sustainable scientific software. The user community includes the developers, 
both internal and external, of the software; direct users of the software; other software projects that depend on
the software; and any other groups that create or consume data that is specific to the software. Together these
groups provide both the reason for sustaining the software and, collectively, the requirements that drive its continued
evolution and improvement.

\subsubsection{Fit with related activities}
%\todo{short text here - can include links/cites}

Mozilla Science maintains a "'Working Open' Project Guide" (http://mozillascience.github.io/leadership-training). From the introduction:
\begin{quote}
Working openly with contributors enables your
    community to learn how to build and collaborate together. This
    document is a guideline on how to work openly and involve others
    in your projects with Mozilla. We want to help you engage your
    community in a way that encourages contributors and builds other
    leaders.
  \end{quote}

Several books have been written about software communities:
\begin{itemize}
\item "Art of Community" by Jono Bacon. We could consider distilling this for scientific software.
\item Iain Larmour, from EPSRC inthe UK [[https://www.epsrc.ac.uk/][EPSRC]] (not sure who from mentioned UK Collaborative Computational Projects ([[http://ccp.ac.uk][CCP]])
\end{itemize}

\subsubsection{Discussion}
%\todo{short-ish text here}

Discussion revolved around a few questions: what is the benefit of having a "community" for software sustainability, what
practices and circumstances lead to having a community, how can funding help or hinder this process, and perhaps most
importantly how can best practices be described and distilled into a document that can help new projects.

The benefits of having a community that were brought up were considered largely obvious. In addition to having advocates for
the software, and a possible source of ``free'' contributions to the codebase, the community becomes a good source for
requirements, feedback, and metrics. The software community can also act as "cheerleaders" who convince funders or other
potential users to fund/use the software, and thus help sustain the software.

Practices and circumstances that lead to a community are first, that the software offers value. But in addition to this, a
community will be much more likely to form if they receive (expert) support when they have questions. Additional contributing
factors are good usability (not always needed), and an open development process such as IPython developer meetings on YouTube.
It was also pointed out that an evangelist for the project, not necessarily but often one of the developers, can often make a
big difference. 

Funding can help the process by encouraging both value to the community and high-quality user support. Only providing funding
for the software development may create good software, but with less likelihood to have a real community. It was discussed
that federal laboratories are a good incubator for software communities, and that a general facility like EarthCube is too
dispersed to really make a community. Also, domain-specific groups within laboratories or universities might provide as an
incubator for software communities.

In describing best practices, the group discussed the different modes for starting a scientific software project: building on
an existing product that needs improving, recognizing an unsatisfied need of an existing community, or creating a new solution
to a need not yet recognized by the community. The group also thought that the existing books on software communities would
need to be evaluated in light of differences between Science Software projets and general OSS projects in terms of scale,
science, acknowledgement and credit, and funding models.

\subsubsection{Plans}
%\todo{short text here - not bullets}

The most important next steps is a ``Best Practice'' document, which would describe what successful projects with engaged
communities look like, how to replicate this type of project, and look at end-of-life on a community project. Inputs to this
document would include a software community survey of high functioning communities such as R Open Science, Python SciPy,
OPeNDAP, and Unidata, with analysis of factors that feed into their success. Also references lik the "Art of Community" could
be adapted and summarized for the science software community.

Another next step would be increasing recognition of need for science software projects to focus on building and supporting
their user community. Good Software Engineering practices are not enough, and popular training like Software Carpentry does
not currently address this issue head on.

\subsubsection{Landing Page}
\todo{link to landing page}


%%%%%%%%%%%%%%%%%%%%%%%%%%%%%%%%%%%%%%%%%%%%%%%%%%%%%%%%%%%%
\section{Conclusions} \label{sec:conclusions}
%%%%%%%%%%%%%%%%%%%%%%%%%%%%%%%%%%%%%%%%%%%%%%%%%%%%%%%%%%%%

In WSSSPE3, we attempted to take what we learned from WSSSPE1 and WSSSPE2
in how we can collaboratively build a workshop agenda and turn that into
an ongoing community activity.  The success or failure of this effort
of this will only become apparent over time.

The workshop had two components, presentations and working groups.  The presentations,
in the first half day of the workshop, including a inspirational keynote and a set of lightning
talks.
We used lightning talks for two reasons: first, the need of some participants to have a slot
on the agenda to justify their attendance, and second, as a way to get new ideas across
to all the attendees.  We broke with the tradition of requiring the lightning talk submitters
to self-publish their papers, and instead used a common peer-review platform\footnote{\url{http://easychair.org}}, choosing to publish the speaker slides on the workshop web site
instead.

The working groups met for a small part of the first half day and all of the second day,
excepting some times for the groups to report back to the collected workshop attendees.
\todo{need to say some more about what came of this}

\todo{this paragraph is from the wssspe2 report - to be updated...}
However, the challenge that we have discovered
since WSSSPE2 is that it is very hard to continue the breakout groups'
activities.  The WSSSPE2 participants were willing to dedicate their time to
the groups while they were at the meeting, but afterwards, they have gone
back to their (paid) jobs.  We need to determine how to tie the WSSSPE
breakout activities to people's jobs, so that they feel that continuing them
is a higher priority than it is now, perhaps through funding the participants,
or through funding coordinators for each activity, or perhaps by getting
the workshop participants to agree to a specific schedule of activities during the
workshop.

%\begin{table*}[t]
%\centering
%\caption{Top tweets tagged \#WSSSPE or \#WSSSPE3 on Sep.\ 28--29, 2015.}\label{tab:tweets}
%  \begin{scriptsize}
%  \begin{tabular}{@{}l l l l@{}}
% \toprule
%    Author  &   Tweet  & Retweets &  Favorites
%\\ \midrule
%Neil P Chue Hong  &  Getting ready for the start of the \#WSSSPE workshop & 2 & 1
%\\ & \url{http://wssspe.researchcomputing.org.uk/wssspe3/agenda/}  & &
%\\ & Video stream: \url{http://ucarconnect.ucar.edu/live?room=cg1aud} & &
%%
%\\ August Muench & Astronomy \& Software for a totally distracting Monday (2/2) => & 1 & 3
%\\ & Workshop on Sustainable SW for Science \#WSSSPE3 & & 
%\\ & \url{http://wssspe.researchcomputing.org.uk/wssspe3/agenda/} & &
%%
%\\ Neil P Chue Hong & My \#WSSSPE lightning talk on Building a Scientific Software & 6 & 3
%\\ &  Accreditation Framework: \url{http://dx.doi.org/10.6084/m9.figshare.1555925} & & 
%%
%\\ Mozilla Science Lab & At \#wssspe3 and want to learn more about our Working Open guide & 2 & 1 
%\\ &  or Contributorship badges pilot? Keep an eye out for @abbycabs! & & 
%%
%\\ Daniel S.\ Katz & \#wssspe starting! & 1 & 3
%%
%\\ Daniel S.\ Katz & @powersoffour starting \#wssspe keynote & 1 & 2
%%
%\\ Neil P Chue Hong   &  The difference between software presented at SC \& SciPy & 1 & 4
%\\  &  according to @powersoffour straw person \#wssspe & &
%%
%\\ Daniel S.\ Katz & sustainability may mean: keeps up with bug reports, adds new features, & 3 & 1
%\\ &  works on new architectures (1/2)  -- @powersoffour at \#wssspe (1/2) & & 
%%
%\\ Daniel S.\ Katz & sustainability may mean: people use it, keeps getting funded, produces & 2 & 
%\\ &  same results, transits over people (2/2)  -- @powersoffour at \#wssspe & &
%%
%\\ Nic Weber & Rec, by Matt Turk: Ash Dryden - Ethics of Unpaid Labor & 3 & 2 
%\\ & \url{http://t.co/KfMwS4VG4N} & &
%%
%\\ Nic Weber & My .02 - many more models (than 3) for sustaining sci. software. & 3 & 2
%\\ & See this pub on open-source:  & & 
%\\ & \url{http://papers.ssrn.com/sol3/papers.cfm?abstract_id=2568185} & &
%%
%\\ Erin Robinson & \#WSSSPE - Software is like a vampire. It's only dead if you really kill it. & 2 & 2
%%
%\\ Neil P Chue Hong & Absolutely great \#wssspe keynote from @powersoffour to inspire & 2 & 1 
%\\ & the workshop. Should be archived via & & 
%\\ & \url{http://ucarconnect.ucar.edu/live?room=cg1aud#.VgmpvCdIrCQ} soon & & 
%%
%\\ Daniel S.\ Katz & \#wssspe lightning talks starting - streaming at  & 5  &  1
%\\ &  \url{http://ucarconnect.ucar.edu/live?room=cg1aud} & & 
%%
%\\ Kyle Niemeyer & Participatory open source means better documentation  & 2 & 1 
%\\ & (readme, roadmap, contributing) ? @abbycabs at \#wssspe & &
%%
%\\ Erin Robinson & Interesting resource: \url{http://mozillascience.github.io/leadership-training/} & 2 & 
%\\ &  \#WSSSPE cc: @abbycabs & & 
%%
%\\ Nic Weber & @MozillaScience + @abbycabs giving look at cool resources & 2 & 3
%\\ & for leading an open source community & & 
%\\ & \url{http://mozillascience.github.io/leadership-training/} & & 
%%
%\\ Nic Weber & Some interesting sepeartion of best practices for Geosoftware dev & 2 & 
%\\ & into categories \url{https://geodynamics.org/cig/dev/best-practices} & &
%%
%\\ Neil P Chue Hong & Some statistics of software citation in geodynamics & 2 & 
%%
%\\ Daniel S.\ Katz & @jamespjh goal: make research software less rubbish; solution is RSEs & 3 & 
%%
%\\ Nic Weber & Highly relevant to \#wssspe crowd - @Impactstory team is launching & 7 & 3
%\\ & a impact of research software project \url{http://depsy.org/}  \#verycool & & 
%%
%\\ Neil P Chue Hong & @jamespjh talking about Research Software Engineer groups in the UK & 2 & 1
%\\ &  \#wssspe @ResearchSoftEng @SoftwareSaved & & 
%%
%\\ Abby Cabunoc Mayes & Wonderful time @ \#WSSSPE 3! Encouraging 2 see a diverse community & 1 & 2 
%\\ & working 2gether (\& making real plans!) \url{http://gif.co/s6v4.gif} & & 
%%
%\\ \bottomrule
%    \end{tabular}
%    \end{scriptsize}
%\end{table*}
%
%WSSSPE actively used the online social network Twitter, with hashtag
%``\#WSSSPE''. There were substantially more tweets (messages) during the days of
%the workshops WSSSPE2, WSSSPE1.1, and WSSSPE1. Out of about 670 tweets as of Apr
%18, 2015, more than 225 were about WSSSPE2 and about 180 were posted during the
%day of the workshop. Some of the main points and highlights in the meeting are
%shown in Table~\ref{tab:tweets}, which summarizes the top \#WSSSPE tweets from
%the day of workshop, selected by the metrics that number of retweets or
%favorites larger than five and the sum of two measures greater than ten.



%%%%%%%%%%%%%%%%%%%%%%%%%%%%%%%%%%%%%%%%%%%%%%%%%%%%%%%%%%%%
\section*{Acknowledgments} \label{sec:acks}
%%%%%%%%%%%%%%%%%%%%%%%%%%%%%%%%%%%%%%%%%%%%%%%%%%%%%%%%%%%%

Work by Katz was supported by the National Science Foundation while working at
the Foundation. Any opinion, finding, and conclusions or recommendations
expressed in this material are those of the author(s) and do not necessarily
reflect the views of the National Science Foundation. Choi's work is supported
in part by the National Science Foundation research grant DMS-1522687 and
a WSSSPE3 travel award. 

\todo{feel free to add stuff here}


\appendix
%%%%%%%%%%%%%%%%%%%%%%%%%%%%%%%%%%%%%%%%%%%%%%%%%%%%%%%%%%%%
\section{Organizing Committee}  \label{sec:orgcom}
%%%%%%%%%%%%%%%%%%%%%%%%%%%%%%%%%%%%%%%%%%%%%%%%%%%%%%%%%%%%
%\todo{Do we want email addresses here?}
The following is the list of organizers of WSSSPE3.

{\scriptsize
\begin{longtable}{lll}
Daniel S. Katz &  University of Chicago \& Argonne National Laboratory, USA\\
Gabrielle Allen &  University of Illinois Urbana-Champaign, USA\\
Neil Chue Hong &  Software Sustainability Institute.  University of Edinburgh, UK\\
Sou-Cheng (Terrya) Choi &  NORC at the University of Chicago \& Illinois Institute of Technology, USA\\
Sandra Gesing &  University of Notre Dame,  USA\\
Lorraine Hwang &   University of California, Davis, USA\\
Manish Parashar &  Rutgers University, USA\\
Erin Robinson &  Foundation for Earth Science, USA (local organizer)\\
Matthew Turk &  University of Illinois Urbana-Champaign, USA\\
Colin C. Venters &  University of Huddersfield, UK

\end{longtable}
}
 

%%%%%%%%%%%%%%%%%%%%%%%%%%%%%%%%%%%%%%%%%%%%%%%%%%%%%%%%%%%%
\section{Attendees}  \label{sec:attendees}
%%%%%%%%%%%%%%%%%%%%%%%%%%%%%%%%%%%%%%%%%%%%%%%%%%%%%%%%%%%%
%\todo{Do we want email addresses here?}
The following is list of participants registered for the WSSSPE3 workshop.

{\scriptsize
\begin{longtable}{lll}
Alice Allen & \href{mailto:aallen@ascl.net}{aallen@ascl.net}&Astrophysics Source Code Library\\
Gabrielle Allen & \href{mailto:gdallen@illinois.edu}{gdallen@illinois.edu}&NCSA\\
Janine Aquino & \href{mailto:janine@ucar.edu}{janine@ucar.edu}&UCAR/NCAR Earth Observing Laboratory\\
Steven Brandt & \href{mailto:sbrandt@cct.lsu.edu}{sbrandt@cct.lsu.edu}&Louisiana State University\\
Jed Brown & \href{mailto:jed@jedbrown.org}{jed@jedbrown.org}&CU Boulder\\
Matthias Bussonnier & \href{mailto:bussonniermatthias@gmail.com}{bussonniermatthias@gmail.com}&UC Berkeley\\
Jeffrey Carver & \href{mailto:carver@cs.ua.edu}{carver@cs.ua.edu}&University of Alabama\\
Emily Chen & \href{mailto:echen35@illinois.edu}{echen35@illinois.edu}&NCSA\\
Sou Cheng Choi & \href{mailto:sctchoi@uchicago.edu}{sctchoi@uchicago.edu}&NORC at UChicago / IIT\\
Nancy Collins & \href{mailto:nancy@ucar.edu}{nancy@ucar.edu}&NCAR\\
Ethan Davis & \href{mailto:edavis@ucar.edu}{edavis@ucar.edu}&UCAR Unidata\\
Davide DelVento & \href{mailto:ddvento@ucar.edu}{ddvento@ucar.edu}&NCAR/CISL\\
Yuhan Ding & \href{mailto:yding2@hawk.iit.edu}{yding2@hawk.iit.edu}&Illinois Institute of Technology\\
Tim Dunne & \href{mailto:tim.dunne@knowinnovation.com}{tim.dunne@knowinnovation.com}&KnowInnovation \\
Ward Fisher & \href{mailto:wfisher@ucar.edu}{wfisher@ucar.edu}&UCAR/Unidata\\
Sandra Gesing & \href{mailto:sandra.gesing@nd.edu}{sandra.gesing@nd.edu}&University of Notre Dame\\
Josh Greenberg & \href{mailto:greenberg@sloan.org}{greenberg@sloan.org}&Sloan Foundation\\
Dan Gunter & \href{mailto:dkgunter@lbl.gov}{dkgunter@lbl.gov}&LBNL\\
Ted Habermann & \href{mailto:thabermann@hdfgroup.org}{thabermann@hdfgroup.org}&The HDF Group\\
James Hetherington & \href{mailto:jamespjh@gmail.com}{jamespjh@gmail.com}&University College London\\
Neil Chue Hong & \href{mailto:N.ChueHong@software.ac.uk}{N.ChueHong@software.ac.uk}&Software Sustainability Institute\\
Elisabeth Huffer & \href{mailto:beth@lingualogica.net}{beth@lingualogica.net}&Lingua Logica/NASA \\
Lorraine Hwang & \href{mailto:ljhwang@ucdavis.edu}{ljhwang@ucdavis.edu}&UC Davis - CIG\\
Raymond Idaszak & \href{mailto:rayi@renci.org}{rayi@renci.org}&RENCI; University of North Carolina at Chapel Hill\\
Elizabeth Jessup & \href{mailto:jessup@cs.colorado.edu}{jessup@cs.colorado.edu}&University of Colorado Boulder\\
Nick Jones & \href{mailto:nick.jones@nesi.org.nz}{nick.jones@nesi.org.nz}&New Zealand eScience Infrastructure (NeSI)\\
Daniel Katz & \href{mailto:dsk@uchicago.edu}{dsk@uchicago.edu}&U Chicago \& Argonne\\
Iain Larmour & \href{mailto:Iain.Larmour@epsrc.ac.uk}{Iain.Larmour@epsrc.ac.uk}&EPSRC (UK)\\
Frank Löffler & \href{mailto:knarf@cct.lsu.edu}{knarf@cct.lsu.edu}&Louisiana State University\\
Suresh Marru & \href{mailto:smarru@iu.edu}{smarru@iu.edu}&Indiana University\\
Ryan May & \href{mailto:rmay@ucar.edu}{rmay@ucar.edu}&UCAR/Unidata\\
Abigail Cabunoc Mayes & \href{mailto:abigail@mozillafoundation.org}{abigail@mozillafoundation.org}&Mozilla Foundation\\
Jeff McWhirter & \href{mailto:jeff.mcwhirter@gmail.com}{jeff.mcwhirter@gmail.com}&Geode Systems\\
Constantinos Michailidis & \href{mailto:costa.michailidis@knowinnovation.com}{costa.michailidis@knowinnovation.com}&Knowinnovation\\
Don Middleton & \href{mailto:don@ucar.edu}{don@ucar.edu}&NCAR\\
Mark Miller & \href{mailto:mark.alan.miller2@gmail.com}{mark.alan.miller2@gmail.com}&SDSC\\
Pate Motter & \href{mailto:pate.motter@colorado.edu}{pate.motter@colorado.edu}&University of Colorado\\
Jaroslaw Nabrzyski & \href{mailto:naber@nd.edu}{naber@nd.edu}&University of Notre Dame\\
Patrick Nichols & \href{mailto:pnichols@ucar.edu}{pnichols@ucar.edu}&National Center for Atmospheric Research\\
Kyle Niemeyer & \href{mailto:Kyle.Niemeyer@oregonstate.edu}{Kyle.Niemeyer@oregonstate.edu}&Oregon State University\\
Laura Owen & \href{mailto:lowen@illinois.edu}{lowen@illinois.edu}&NCSA\\
Abani Patra & \href{mailto:abani@buffalo.edu}{abani@buffalo.edu}&Univ at Buffalo\\
Grace Peng & \href{mailto:grace@ucar.edu}{grace@ucar.edu}&National Center for Atmospheric Research\\
Birgit Penzenstadler & \href{mailto:Birgit.Penzenstadler@csulb.edu}{Birgit.Penzenstadler@csulb.edu}&California State University Long Beach\\
Lindsay Powers & \href{mailto:lpowers@hdfgroup.org}{lpowers@hdfgroup.org}&The HDF Group\\
Bernie Randles & \href{mailto:randles@ucla.edu}{randles@ucla.edu}&UCLA\\
Erin Robinson & \href{mailto:erinrobinson@esipfed.org}{erinrobinson@esipfed.org}&Foundation for Earth Science\\
Daniel Sellars & \href{mailto:dan.sellars@canarie.ca}{dan.sellars@canarie.ca}&CANARIE Inc\\
Nikolay Simakov & \href{mailto:nikolays@buffalo.edu}{nikolays@buffalo.edu}&SUNY University at Buffalo\\
Ian Taylor & \href{mailto:ian.j.taylor@gmail.com}{ian.j.taylor@gmail.com}&Cardiff University\\
Ilian Todorov & \href{mailto:ilian.todorov@stfc.ac.uk}{ilian.todorov@stfc.ac.uk}&Science \& Technology Facilities Council, UK\\
Benjamin Tovar & \href{mailto:btovar@nd.edu}{btovar@nd.edu}&University of Notre Dame\\
Gregory Tucker & \href{mailto:gtucker@colorado.edu}{gtucker@colorado.edu}&University of Colorado at Boulder\\
Matthew Turk & \href{mailto:mjturk@illinois.edu}{mjturk@illinois.edu}&NCSA\\
Colin Venters & \href{mailto:colin.venters@googlemail.com}{colin.venters@googlemail.com}&University of Huddersfield\\
Alexander Vyushkov & \href{mailto:avyushko@nd.edu}{avyushko@nd.edu}&University of Notre Dame\\
Fraser Watson & \href{mailto:fwatson@nso.edu}{fwatson@nso.edu}&National Solar Observatory\\
Nic Weber & \href{mailto:nmweber@uw.edu}{nmweber@uw.edu}&University of Washington\\
Daniel Ziskin & \href{mailto:ziskin@ucar.edu}{ziskin@ucar.edu}&NCAR - ACOM\\

\end{longtable}
}

%%%%%%%%%%%%%%%%%%%%%%%%%%%%%%%%%%%%%%%%%%%%%%%%%%%%%%%%%%%%
\section{Travel Award Recipients}  \label{sec:awardees}
%%%%%%%%%%%%%%%%%%%%%%%%%%%%%%%%%%%%%%%%%%%%%%%%%%%%%%%%%%%%
%\todo{Do we want email addresses here?}
The following is list of travel award recipients for the WSSSPE3 workshop.

{\scriptsize
\begin{longtable}{lll}
Alice Allen & Astrophysics Source Code Library\\
Steven Brandt &  Louisiana State University\\
Jeffrey Carver &  University of Alabama\\
Emily Chen & NCSA, University of Illinois\\
Sou-Cheng Choi &  Illinois Institute of Technology\\
Yuhan Ding &  Illinois Institute of Technology\\
Lorraine Hwang &  CIG,  UC Davis\\
Ray Idaszak &  RENCI, University of North Carolina at Chapel Hill\\
%Frank L\"{o}ffler &  Louisiana State University\\
%Abigail Cabunoc Mayes &  Mozilla Science Lab\\
Pate Motter &  University of Colorado\\
Kyle Niemeyer &  Oregon State University\\
Birgit Penzenstadler &  California State University Long Beach\\
Bernadette Randles &  UCLA\\
%Ilian Todorov &  STFC Daresbury Laboratory\\
Nic Weber &  University of Washington iSchool

\end{longtable}
}


%%%%%%%%%%%%%%%%%%%%%%%%%%%%%%%%%%%%%%%%%%%%%%%%%%%%%%%%%%%%
\section{Best Practices Group Discussion}
\label{sec:appendix_best_practices}
%%%%%%%%%%%%%%%%%%%%%%%%%%%%%%%%%%%%%%%%%%%%%%%%%%%%%%%%%%%%

\todo{add POC here}

\subsection{Group Members}

\begin{itemize}
\item Abani Patra 
\item Sandra Gesing -- University of Notre Dame
\item Neil Chue Hong 
\item Greg Tucker 
\item Birgit Penzens 
\item Abigail Cabunoc Mayes 
\item Jeff Carver 
\item Frank L\"offler 
\item Colin Venter 
\item Lorraine Hwang 
\item Sou-Cheng Choi
\item Suresh Marru 
\item Don Middleton 
\item Daniel Katz  
\item Kyle Niemeyer 
\item Jeff Carver 
\item Dan Gunter 
\item Alexander Konovalov 
\item Tom Crick 

\end{itemize}

\subsection{Summary of Discussion}
Core questions that will need to be explored are in knowledge management, 
(transitions between people), reliability (reproducibility), usability and how a software tool becomes part of the core workflow of well identified users (stakeholders)
relating to tool success and hence sustainability. Ideas 
that may need to be explored include:
\begin{itemize}
\item Requirements engineering to create tools with immediate uptake;
\item When should software "die"?
\item Catering to disruptive developments in environment e.g.(new hardware, new methodology) ;
\item Dimensions of sustainability - economic, technical, environmental, 
declining interest in primary application area), social.
\item 
\end{itemize}

Sustainability requires community participation in code development and/or a wide adoption of software. The larger the community base is using a software, the better are the funding possibilities and thus also the sustainability options. Additionally developer commitment to an application is essential and experience shows that especially software packages with an evangelist imposing strong inspiration and discipline is required to achieve sustainability.
While a single person can push sustainability to a certain level, open source software also needs sustained commitment from the developer community. Such sustained commitments include diverse tasks and roles, which can be fulfilled by diverse developers with different knowledge levels. Besides developing software and appropriate software management with measures for extensibility and scalability of the software, active (expertise) support for users via a user forum with a quick turnaround is crucial. The barrier to entry the community as user as well as developer has to be as low as possible.

\subsection{Description of Opportunity, Challenges, and Obstacles}

The opportunity lays in the collaboration on a white paper, which will be revisited regularly for further improvements and enhances the state on best practices resulting in a peer-reviewed paper. This way we would like to reach a wide community. But these are also the challenges and obstacles - to get everyone to write on the paper and reach the community.

\subsection{Key Next Steps}

The key next steps are to write the introduction, reach out to the co-authors and agree on a scope of the white paper.

\subsection{Plan for Future Organization}

Sandra Gesing and Abani Patra are the main editors and will organize the overall communication and paper. Sections will be assigned to diverse co-authors.

\subsection{What Else is Needed?}

At the moment we don't see any further requirements.

\subsection{Key Milestones and Responsible Parties}
\item [15 Nov] Introduction and scope finished (Abani Patra/Sandra Gesing)
\item [15 Nov] Sections assigned (Abani Patra/Sandra Gesing)
\item [31 Jan] Analysing funding possibilities for survey
\item [31 Jan] First versions of section
\item [15 Feb] Distribution to WSSSPE community
\item [31 Mar] Final version of white paper
\item [30 Apr] Submission of peer-reviewed paper?

\subsection{Description of Funding Needed}
We might need funding for a journal publication (open-access options).

%%%%%%%%%%%%%%%%%%%%%%%%%%%%%%%%%%%%%%%%%%%%%%%%%%%%%%%%%%%%
\section{Funding Specialist Expertise Group Discussion}
\label{sec:appendix_funding_spec_expert}
%%%%%%%%%%%%%%%%%%%%%%%%%%%%%%%%%%%%%%%%%%%%%%%%%%%%%%%%%%%%

James Hetherington\footnote{email: \href{mailto:j.hetherington@ucl.ac.uk}{j.hetherington@ucl.ac.uk}}
will serve as the point of contact for this working group, and be responsible for ensuring timely progress of the planned actions.

\subsection{Group Members}

The group at WSSPE:

\begin{itemize}
\item Don Middleton -- National Center for Atmospheric Research
\item Joshua Greenberg -- Alfred P. Sloan Foundation
\item James Hetherington -- University College London
\item Lindsay Powers -- The HDF Group
\item Mark A. Miller -- San Diego Supercomputer Center
\item Dan Sellars -- CANARIE
\end{itemize}

This was further enhanced by additional discussions at the following
GCE15 conference:
  
\begin{itemize}
\item Lorraine Hwang -- UC Davis
\item Simon Trigger
\item Nancy Wilkins-Diehr -- San Diego Supercomputer Center
\item Alexander Vyushkov -- Notre Dame
\item Sandra Gesing - Notre Dame
\item Ali Swanson -- University of Oxford

\end{itemize}

\subsection{Summary of Discussion}

In addition to the points noted in the main discussion~\ref{RSE}, we also
discussed the following:

``Are you an RSE or a RA?'' -- this is not properly a binary question. Most of
us sit at different points on that spectrum, and move along it during our
careers. (Usually from RA to RSE -- examples of movement in the other direction
from readers would be welcomed.)
Either way, the label ``Research Software Engineer'' is now starting to
have some power. Many scientists do not want to be writing code; some do, to
varying degrees. These groups can usefully support each other.

The power of the label? Getting the word out about RSE support using the label.

Will research science developers be required, in the long run?
One issue that came up was whether the need for developers was a time bounded one; is it the case that the new generation of computer and software savvy scientists will be so comfortable in developing their own code that the professional developer will not be needed. And this brings up the flip side question, “Do scientists really want to be writing code?”

Career Path.
We also had a little discussion about how to make a career path for research developers. It need not be solely an academic enterprise, but in the past tenure has often been problematic for people of this class.

Skills and resources may vary between teams. To help resolve this, maintaining
high levels of communication between groups will be valuable. In the UK, there
are plans to permit resource sharing between institutional RSE groups. Perhaps 
there are circumstances under which an RSE skill exchange could be arranged, either
formally or informally. 

Collaborative funding can be important for RSE groups, to ensure that research
leadership remains with the domain scientists. At NCAR, University partnerships
are required for submission of proposals, so collaboration is an essential part
of grant submission, and this will tend to bring developers and
scientists together. The UCL group also follows this approach, with all bids
requiring an academic collaborator.

Domain scientists and developers are funded together in a single proposal.
Another example of a success is the development of semantics and linked data in
support of Ocean Sciences. An EarthCube-funded project pairs domain scientists
with RSEs and has been successful; the semantics attached have increased data
use and discovery significantly.

An alternative approach has been the provision of programming expertise as
part of national compute services. The US XSEDE project's extended collaboration
service (ECSS) is a set of developers who are paid with XSEDE funding, and are on 
“permanent” staff.  When PIs request allocations on XSEDE resources, there is a 
finite pool of developer time that can be awarded, typically for one year only, 
and at partial effort, typically 20 percent or so. The finite time allowed provides
motivation for the scientist and their group to work closely the developer and to 
become educated in what the developer is doing, so they can sustain the effort, once 
the ECSS period is over. This funding mechanism can be highly efficient for scientific
problems, because the developer pool assembled by the research providers are, by definition, 
expert in the characteristics of their specific resource, and can very quickly assess the
scientists needs, and what it will take to implement software that meets the user’s needs. 
However, it does not develop capacity within institutions, and since XSEDE is a time-bounded
program, it should not be relied upon as a long term solution to acquiring this type of capacity.

The UK allows this kind of collaboration to support the creation of scientific
software for the large supercomputing resource called ARCHER.
However, as well as resourcing this from the staff of the Edinburgh Parallel
Computing Centre, who host the computer, this ``embedded CSE'' resource also
allows the programming to come from local groups. This has been very helpful
in providing funding to establish local groups. These work best when they
develop good collaborations with national cyberinfrastructure pools.
When an organization assembles a developer pool, diversity is developed and 
skills can be transferred.

We would like to see these models applied outside High Performance Computing.
Most scientific software is not destined to run on national cyberinfrastructure,
but needs similar support. The argument regarding making better use of expensive
hardware through software improvements has been useful politically, (and many
RSE groups are cited in organizations which host clusters for this reason), but
the time has come to make the case that software itself is a critical
cyberinfrastructure, and, with a much longer shelf-life than hardware, is itself
a capital investment.

The Canarie group (Canada) accepts proposals for providing services to broad communities, 
integrate people who are doing things that are complementary, the goal is to make the available 
stack more robust and richer for everyone. They offer short rounds of funding that can
have as a key metric creating some useful functionality that shows a diversity
of input and draws from across disciplines, then more funding could follow. This
could apply within or across institutions.

There can be problems communicating across cultural barriers, with domain scientists seeing developers 
as “other”. Collaboration, and tools to fund, encourage or motivate collaboration are
extremely important.

We think support from non-governmental organizations will be important if RSE groups will become established.
The Sloan Foundation is currently funding data science engineers, who work in the context of other SW developers
at University of Washington. These scientists work in the e-Science Studio/Data Science Studio, and they
help a group of graduate students in solving their problems in data science and data management. During Fall and
Spring, a 10-week incubator program allows students to work two days a week to work on a data-intensive science
project. Some fraction of the developer time is dedicated to the developers' personal interests as well as instruction.

The goal for Sloan is to provide success stories, to provide demonstrable value
in the presence of data scientists on university staff. These stories are the
basis for arguments to the host organization. This is an effort to create
awareness of the value of research scientist developers. Embedding with
scientists, and adding spare capacity is critical to making the innovation
possible. This model is essentially to argue for permanent budget lines to
support data scientists as part of university staff hires, just as with core
facilities. This could become a fee-for-service model requested by grant
funding, just as DNA sequencing is for core facilities, if it becomes apparent
that this gives competitive advantage to the University’s research effort.

One model that has been helpful in finding funding for RSE groups is the use of
funds left over on research grants when RAs have left prematurely -- PIs like
this arrangement as it is hard to find good staff for short-term positions, so
having a pool of research programming staff on hand resolves this problem.
We recommend that funders give explicit guidance to grant holders and
institutions that such arrangement are favorable. Framework agreements
permitting this to go ahead without
checking back every time with funders and/or grant panels would further
smooth this.
(This is also provides a more stable job for those who hold these skills, but
arguments about making life nicer for postdocs will not help persuade funders
or PIs!)

There is some question about the most effective duration and percent of full time for a 
programmer's work on a project. At least three months is necessary for the programmer to 
read into the science (RSEs must not become so disengaged from research that they don't have
time to read a few papers -- this will result in code which doesn't meet scientific needs), 
but too long could result in an RSE losing their flexibility, becoming so engaged in one project
that when that project ends, they find it hard to transfer. For this reason, we also recommend
that 40 percent is ideal; two projects per developer, with some time for training and infrastructure work.
Two developers per project seems to be ideal, in the sense that software development is enhanced by 
two pairs of eyes.

There is as yet no clear answer as to the scale of aggregation needed to
make such a program work. A university wide program allows enough scale to
be robust to fluctuations of funding within one field.
But a specialization focus on developers to support, for example,
physical or biological
sciences may be preferable, if the customer base is large enough.
The desire to aggregate enough work to make it sustainable, and the need to
have domain-relevant research programming skills, are in tension.

In the UK, another source of funding for research software has been the
Collaborative computational projects (CCPs): domain specific communities put
forward proposals that are a priority of the community as a whole, for example,
biosimulation or plasma physics. These bodies act as custodians of community codes,
and a central team also provides software engineering support.

However this area develops, the need for funding to secure software as a
cyberinfrastructure component is clear. Funding which permits code to be
refactored, tidied and optimized  is rare; this is often done ``on the sly''
in a scientifically focused grant. The UK EPSRC's ``software for the future''
call, which really permits explicit investment in software as an infrastructure,
is so oversubscribed as to have a FOUR PERCENT success rate; the demand is clear!

One opportunity is the idea of co-design; where infrastructural libraries are
developed alongside the scientific codes that will call them. However,
collaboration is hard to foster here; as incentive structures are still focused
on short-term papers. This can cause infrastructure developers to
focus more on publications
in their areas of mathematics and CS, the domain developers on the shorter-term
needs of their own fields. Genuine collaborative co-construction
is harder to foster.

It can be a more difficult
to help leading domain scientists see the value of engineering effort than those
in their teams who are forced to work with difficult-to-use or unreliable
software tools, as they do not see the pain. Perhaps a version of
``software carpentry'' targeted at those PIs
who are awarded or apply for software heavy grants could be valuable here.

RSEs provide a useful contribution to their Universities' *teaching* missions,
as well as for research, as they are well placed to deliver the research programming
training that many scientists now need. In the
longer term, with programming skills taught to all through their career,
we hope specialist
scientific developers will be less needed.

\subsection{Key Next Steps}

We will seek to identify and approach existing research programming organizations,
to get their permission to list them on a list of research software groups.
Casual conversation during the meeting made it clear that although the title is not
widely used in the US, the this position is not rare. We spoke with several individuals
who, at distinct Universities, had RSEs (in effect if not in name) and these were funded
and under differing models. 

We will also look for examples of groups which have successfully become self-
sustaining following. 

In this respect, information gathering via a survey subsequent analysis could be very useful.
We would need to assemble a list of targeted individuals (what position and rank is
likely to know and care enough to respond?) Perhaps the Science Gateway Institute has already acquire
information that could be helpful to advance this issue, and/or craft the proper survey and target
individuals. 

\subsection{Plan for Future Organization and Future Needs}

The UKRSE community will provide initial facilities to host this list, and
continue to work to spread the initiative, but local leadership in the US
is needed if this campaign is to succeed. This will require an initial gathering
of identified research software organizations in the US to this end.

\subsection{Description of Funding Needed}

Financial support for an initial conference to bring together research software
groups to form an organization and create a resource sharing structure would
help to further this campaign. Funding to conduct and analyze a survey could also be quite useful
as knowing where we stand today, and what models are in use could fuel the ideas for further
development of developers in this category.

In the longer term, funding organizations, especially non-governmental organizations
with the capability to effect innovation through seed funding, could provide
support to nucleate the creation of research software groups. As noted above, Sloan has already
initiated one such program, and collaboration with Sloan or at least study of their methods and 
success or failure could be extremely useful in approaching Universities and other institutions 
in funding this development track. It seems clear that if the value proposition can be made to
University administrators, this track could flourish with buy-in at the administrative level.  


%%%%%%%%%%%%%%%%%%%%%%%%%%%%%%%%%%%%%%%%%%%%%%%%%%%%%%%%%%%%%
\section{Catalogs Working Group Discussion}
\label{sec:appendix_catalogs}
%%%%%%%%%%%%%%%%%%%%%%%%%%%%%%%%%%%%%%%%%%%%%%%%%%%%%%%%%%%%

\todo{add POC here}

\subsection{Group Members}

\begin{itemize}
\item name -- affiliation
\item name2 -- affiliation2
\end{itemize}

\subsection{Summary of Discussion}

\subsection{Description of Opportunity, Challenges, and Obstacles}


\subsection{Key Next Steps}


\subsection{Plan for Future Organization}


\subsection{What Else is Needed?}


\subsection{Key Milestones and Responsible Parties}


\subsection{Description of Funding Needed}

%%%%%%%%%%%%%%%%%%%%%%%%%%%%%%%%%%%%%%%%%%%%%%%%%%%%%%%%%%%%
\section{Transition Pathways to Sustainable Software: Industry \& Academic Collaboration Working Group Discussion}
\label{sec:appendix_industry_interaction}
%%%%%%%%%%%%%%%%%%%%%%%%%%%%%%%%%%%%%%%%%%%%%%%%%%%%%%%%%%%%

Nic Weber\footnote{email: \href{mailto:nmweber@uw.edu}{nmweber@uw.edu}} will serve as the point of contact for this working group.

\subsection{Group Members}

\begin{itemize}
\item Nic Weber -- University of Washington
\item Suresh Marru -- Indiana University
\item Jeffrey Carver -- University of Alabama
\item Davide DelVento -- NCAR/CISL
\item Steven Brandt -- Louisiana State University
\end{itemize} 

\subsection{Summary of Discussion}

Our initial broad question was, "What makes for successful transitions of scientific software from academia to industry?" There are a number of potential funding transitions that may occur:  
%
\begin{itemize}

\item A project could be \textbf{refunded}, and development or maintenance of
the software continue as planned.

\item A project might locate a \textbf{new source of funding} in which case the
software may be further developed or simply maintained as before.

\item The project could transition to a \textbf{community supported model}
whereby ownership, maintenance, and stewardship of the software become similar
to peer-production models in open-source (e.g., see Howison~\cite{howison_sustaining_2015}).

\item The project could receive some form of industry sponsorship in which case
ownership of the intellectual property, licensing, maintenance activities,
hosting, etc.\ may change significantly.

\item The project could gain attention from a industry use case who would potentially make in-kind
contributions by having paid staff contribute to the software.

\end{itemize}

We characterized each of these potential changes in funding as ``transition
pathways'' to sustainable software (see similar work by Geels and
Schot~\cite{Geels:2007}).

Our work at WSSSPE3 included the following activities (described in detail
below): (1) brainstorming goals for this type of research, (2) imagining
potential outcomes of completing a set of case studies on this topic, and (3)
generating a set of working definitions for some of the broad concepts we are
describing.

First, we discussed the \textbf{goals} of this research, attempting to answer the 
question \emph{What is the goal of doing research on transition pathways?}
A number of research questions arose:  Can we identify collaborations that have 
occurred and try to understand which were successful, which were unsuccessful, 
and what factors contributed to these successes/failures? Can we determine what 
each partner wants to get out of such a collaboration? For example, why would 
industry be interested in collaborating with academia? Or why would academia 
be interested in collaborating with industry? How could we design a study that 
focused on the impact of the software in undergoing this type of transition?

Next, we imagined \textbf{potential outcomes} of research on this topic, involving 
a set of case studies that look at successful and unsuccessful
transitions of researchers between academia and industry. This might address 
each of the transition types (described below). Successful transitions are
described as those that lead to either weak or strong sustainability (also
defined below). In addition, the results from this research might help create a 
generalizable framework that might allow for the study of different transition 
pathways (other than academia to industry).

Finally, we created some \textbf{general definitions} for these concepts; we 
characterize transitions in the following ways:
\begin{itemize}

\item Handoff model: academia initially writes the software, industry (for-profit 
or nonprofit) then takes over the project.

\item Co-Production Model: industry and academia interact throughout development
of the project.

\item Sponsorship Model: academia writes and maintains the software; 
industry contributes funding for the development\slash maintenance of software.
In this example, industry is also likely a user of the software.

\item Spinoff model: transition to a for-profit or non-profit company owned by or in
collaboration with original developers.

\end{itemize}

We characterized sustainability in the following ways:
\begin{itemize}

\item Weak Sustainability: Software continues to be accessible, useful, and
usable.

\item Strong Sustainability: Software meets criteria above, but is also able to
be reused for further innovation (i.e., issued non-restrictive open-source
license).

\end{itemize}
We refer readers to Becker et al.~\cite{Becker:2014} for an extended discussion of weak versus strong
sustainability. 

\subsection{Description of Opportunity, Challenges, and Obstacles}
The opportunity is to create a catalog of success/failure for current and future software projects to be prepared for transitions and achieve sustainability of the software. 

The obstacle is more superficial, in finding a champion to gather such information. It will be a challenge to keep this information and surveys updated. With changing rapidly changing industry landscapes, an obsolete survey could be of less or no use. 

\subsection{Key Next Steps}

Identify projects that are collaborative, perhaps by reviewing funded projects from programs specifically geared towards industry academic collaborations.

Develop a systematic process for conducting case studies (what kind of data are being gathered about each case)

\subsection{Plan for Future Organization}
No concrete plans were made at this point. If community can rally behind, some momentum could be built. If you are interested post at \url{https://github.com/WSSSPE/meetings/issues/46} 

\subsection{What Else is Needed?}
Nothing at the moment. 

\subsection{Key Milestones and Responsible Parties}
A key portion of this effort will require focused surveys of projects which have succeeded and failed in transition. Both these categories will yield good learning on what works and what does not work. The group has identified what needs to be studied further, but has not identified responsible parties to conduct them.

Community could help in gathering data by means of Interviews; Historical documents / documentation and Surveys

An example data collection will be: 
\begin{itemize}
\item Origin: Where did project start? 
\item People involved: How many people in original project were involved in transition/collaboration? 
\item Specs on software
\item Language
\item Size
\item Hardness (age)
\item Lead-up to Transition: How long was project in development before it began transition?
\item Motivation for Transition: Why was transition initiated? By whom? 
\end{itemize}

\subsection{Description of Funding Needed}
Concrete funding needs were not discussed in this working group but a general impression was a seed funding will motivate members of this group or others in community to launch a survey effort. 
%\input{wg_appendix_legacy_SW}
%%%%%%%%%%%%%%%%%%%%%%%%%%%%%%%%%%%%%%%%%%%%%%%%%%%%%%%%%%%%
\section{Engineering Design Group Discussion}
\label{sec:appendix_eng_design}
%%%%%%%%%%%%%%%%%%%%%%%%%%%%%%%%%%%%%%%%%%%%%%%%%%%%%%%%%%%%

Birgit Penzenstadler\footnote{email: \href{mailto:birgit.penzenstadler@csulb.edu}{birgit.penzenstadler@csulb.edu}} and Colin C. Venters\footnote{email: \href{mailto:c.venters@hud.ac.uk}{c.venters@hud.ac.uk}} will serve as the points of contact for this working group, and be responsible for ensuring timely progress of the planned actions.

\subsection{Group Members}

\begin{itemize}
\item Birgit Penzenstadler -- California State University, CA, USA
\item Colin C. Venters -- University of Huddersfield, Huddersfield, UK
\item Matthias Bussonnier -- UC Berkeley, CA, USA
\item Jeff McWhirter -- Geode Systems 
\item Patrick Nichols -- National Center for Atmospheric Research, CO, USA
\item Ilian Todorov -- Science \& Technology Facilities Council, UK
\item Ian Taylor -- Cardiff University, UK
\item Alexander Vyushkov -- University of Notre Dame, IN, USA
\end{itemize}

\subsection{Summary of Discussion}
Software engineering principles form the basis of methods, techniques, methodologies and tools. This group discussed the principles of software engineering design for sustainable software and their application in various domains including quantum chemistry and epidemiology. 

\subsection{Description of Opportunity, Challenges, and Obstacles}
The opportunity was identified to distill existing software engineering and sustainability design knowledge into ''bite sized'' chunks for the Computational Science and Engineering Community. In addition, two primary challenges were identified:
\begin{itemize}
\item Mapping of the principles to best practice.
\item Demonstrating the return on investment of those best practices.
\end{itemize}

\subsection{Key Next Steps}
In order to achieve (1) the systematic analysis a number of example systems from different scientific domains with regards to the identified principles, (2) the identification of the commonalities and gaps in applying principles to different scientific systems, and (3) the proposal of a set of guidelines on the principles, the following next steps were discussed:

\subsection{Plan for Future Organization}
The following plan for future organization was discussed:
\begin{itemize}
\item Identify suitable undergraduate or post-graduate students.
\item Design and pilot study.
\item Organizing coordinating online calls via Google Hangout.
\end{itemize}

\subsection{What Else is Needed?}
The following list of what else is required :
\begin{itemize}
\item Ethics committee review panel approval required for data collection.
\end{itemize}

\subsection{Key Milestones and Responsible Parties}
The following key milestones were discussed as a roadmap for the set of guidelines on software engineering principles:
\begin{itemize}
\item Oct/Nov 2015: Study design and interview guideline
\item Jan/Feb 2016: Interviews conducted and transcribed
\item Mar/Apr 2016: Analysis complete
\item May 2016: Report written
\end{itemize}

\subsection{Description of Funding Needed}
Specific funding was not discussed in this working group. However, this is a open topic that can be be discussed in relation to emerging funding calls from National agencies or grant proposal initatives.

%%%%%%%%%%%%%%%%%%%%%%%%%%%%%%%%%%%%%%%%%%%%%%%%%%%%%%%%%%%%
\section{Metrics Working Group Discussion}
\label{sec:appendix_metrics}
%%%%%%%%%%%%%%%%%%%%%%%%%%%%%%%%%%%%%%%%%%%%%%%%%%%%%%%%%%%%

Gabrielle Allen \footnote{email: \href{mailto:gdallen@illinois.edu}{gdallen@illinois.edu}} will serve as the point of contact for this working group.


%%%%%%%%%%%%%%%%%%%%%%%%%%%%%%%%%%%%%%%%%%%%%%%%%%%%%%%%%%%%
\subsection{Group Members}
%%%%%%%%%%%%%%%%%%%%%%%%%%%%%%%%%%%%%%%%%%%%%%%%%%%%%%%%%%%%

\begin{itemize}
\item Gabrielle Allen -- National Center for Supercomputing Applications
\item Emily Chen -- University of Illinois at Urbana-Champaign
\item Neil Chue Hong -- U.K. Software Sustainability Institute
\item Ray Idaszak -- RENCI, University of North Carolina at Chapel Hill
\item Iain Larmou -- Engineering and Physical Sciences Research Council
\item Bernie Randles -- University of California, Los Angeles
\item Dan Sellars -- Canarie
\item Fraser Watson -- National Solar Observatory
\end{itemize}

\subsection{Summary of Discussion}

The following summary of the group’s discussion represents the Useful Metrics for Scientific Software working group’s discussion during the WSSSPE3 workshop. The group discussion began by agreeing on the common purpose of creating a set of guidance giving examples of specific metrics for the success of scientific software in use, why they were chosen, what they are useful to measure, and any challenges and pitfalls; then publish this as a white paper.  The group discussed many questions related to useful metrics for scientific software as follows: 

\begin{itemize}

\item
Is there a common set of metrics, that can be filtered in some way

\begin{itemize}
\item
        Does this create a large cost
\end{itemize}

\item
Can we fit metrics into a common template (i.e. for collection, for description)

\item
Which would be the most useful ones

\begin{itemize}
\item
        Which ones would be most useful for each stakeholder
\end{itemize}

\item
Which ones are the most helpful, and how would we assess this

\item
How do you monitor

\begin{itemize}
\item
        Self-checking - if monitoring is done in the open, then people will call out cheats
\end{itemize}

\item
Should this be published with the software metadata

\begin{itemize}
\item
        This would make it easier for public to see the metadata

\item
        However, there is no commonly used standard (DOAP is a good standard but not widely adopted) 

\item
        The Open Directory Project (ODP) metadata is available for UK infrastructure
\end{itemize}

\item
Intersection of most useful and easiest to collect should be explored

\item
How can students/curricula be used as part of a solution

\item
Number of users could be affected by other metrics e.g. by accessibility

\item
Assume metrics are collected properly, but guidance should be provided none-the-less

\item
Continuum for each metric

\begin{itemize}
\item
        Ideal situation is the absolute minimum, so that people can decide on their own what the cost versus usefulness tipping point is
\end{itemize}

\item
Maturity plays a part

\begin{itemize}
\item
        Consider different metrics brackets for different maturity levels
\end{itemize}

\item
What are we using metrics for

\begin{itemize}
\item
        What software should I use if I have a choice

\item
        Where should funders place funding for best impact (e.g. funding two-star software versus three-star) and where there are gaps

\item
        How to promote reduction of code proliferation

\item
        Metrics used for software panels to provide information

\item
        Metrics used for finding problems in their systems

\end{itemize}

\item
Can we use metrics to help people identify the best codes as part of a community effort

\end{itemize}

\smallskip
\noindent
Next, a roadmap for how to proceed was discussed including creating a set of milestones and tasks as follows:

\begin{itemize}
\item
Can we create a roadmap and milestones for this activity

\item
Need to come up with a set of tasks

\item
Go to NSF Software Infrastructure for Sustained Innovation (SI2) projects asking them what metrics they defined, and how useful they were

\begin{itemize}
\item
        Milestone: Create report which assesses the metrics that SI2 projects used

\begin{itemize}
\item
                Ask SI2 PIs to say what metrics they said they would use (copied from proposal)

\item
                Ask SI2 PIs what numbers they reported

\item
                Ask SI2 PIs what they would have changed

\item
                A UIUC student on the project will work on this
\end{itemize}

\item
        Tentatively aim for March 2016
\end{itemize}

\item
Do something similar for UK SFTF and TRDF software projects to ask them what would be useful metrics to report; also eCSE projects

\begin{itemize}
\item
        Compare these to understand if there were any implications for including metrics
\end{itemize}

\item
Collaboratively create plan and documentation for doing this

\begin{itemize}
\item
        Give some examples from group members projects, and aim to build out some of the measurement continuum

\item
        Road-test at the WSSSPE4 meeting
\end{itemize}

\item
Collect the various frameworks together and do a comparison summary

\end{itemize}

\smallskip
\noindent
The idea was put forth for the group to interact with the organizing committee of the 2016 NSF Software Infrastructure for Sustained Innovation (SI2) PI workshop in order to email out a software metrics survey to all SI2 and related awardees as a targeted and relevant set of stakeholders.  This survey would be created by one of the student group members.  Similarly, it was suggested that a software metrics survey be sent to the UK SFTF and TRDF software projects to ask them what metrics would be useful to report.  The remainder of the discussion focused mainly on the creation of a white paper on this topic.  This resulted in a paper outline and writing assignments with the goal of publishing in venues including WSSSPE4, IEEE CISE, or JORS.



\subsection{Description of Opportunity, Challenges, and Obstacles}

The following opportunies, challenges, and obstacles were discussed:

\begin{itemize}
\item
Metrics are important for:

\begin{itemize}
\item
        Tenure and promotion

\item
        Scientific impact

\item
        Discovery

\item
        Reducing duplication

\item
        Basis for potential industrial interest in adopting software

\item
        Make case for funding
\end{itemize}

\item
No commonly-used standard for collecting or presenting metrics

\item
We don’t know if there is a common set of metrics

\item
We have to persuade projects that it is useful to collect metrics

\end{itemize}



\subsection{Key Next Steps}

The following next steps were discussed:

\begin{itemize}
\item
Skype phone call to coordinate shortly after the conclusion of the WSSSPE3 workshop

\item
Get started on IRB at University of Illinois Urbana-Champaign in anticipation of SI2 project survey (may need more thought into survey)

\item
Get started on white paper and associated survey

\end{itemize}



\subsection{Plan for Future Organization}

The following plan for future organization was discussed:

\begin{itemize}
\item
Our group has created a white paper outline with sections assigned to the above individuals, plus see Section 2 response above for timeline

\item
Organizing coordinating phone calls

\end{itemize}




\subsection{What Else is Needed?}

The following list of what else is needed was discussed:

\begin{itemize}
\item
IRB approval/exemption needed for surveys, collecting data

\item
Coordination with 2016 NSF SI2 PI workshop organizing committee to possibly piggyback on this event to offer survey to attendees in advance

\item
Coordination (mail communication, info page etc), via WSSSPE github or?

\end{itemize}



\subsection{Key Milestones and Responsible Parties}

The following items were discussed as a roadmap for the production of a white paper:

\begin{enumerate}
\item
October – November 2015: IRB paperwork as appropriate completed (Gabrielle Allen and Emily Chen)

\item
October – December 2015: Draft white paper sections 1-3 (the paper outline has initial writing assignments)

\item
October – December 2015: Run surveys and collect information

\begin{enumerate}
\item
        Piggy back on planning for 2016 NSF SI2 PIs meeting to be held Feb 16-17, 2016
\end{enumerate}

\item
January – February 2016: Analyze results of data collection from projects

\item
March – April 2016: Draft sections 4-7 of the white paper

\item
May 2016: Draft section 8-9 of the white paper

\item
May – June 2016: Get initial feedback from members of the community and revise

\item
Est. July 2016: By time of next CFP for WSSSPE have complete draft of white paper

\item
Est. Sept – Oct 2016: Responses to white paper submitted to WSSSPE4

\end{enumerate}


\subsection{Description of Funding Needed}

Funding needs were not discussed in this working group and it was thought that this could potentially be revisited down the road.




%%%%%%%%%%%%%%%%%%%%%%%%%%%%%%%%%%%%%%%%%%%%%%%%%%%%%%%%%%%%
\section{Training Working Group Discussion}
\label{sec:appendix_training}
%%%%%%%%%%%%%%%%%%%%%%%%%%%%%%%%%%%%%%%%%%%%%%%%%%%%%%%%%%%%

Nick Jones\footnote{email:
\href{mailto:nick.jones@nesi.org.nz}{nick.jones@nesi.org.nz}} will serve as the
point of contact for this working group, and be responsible for ensuring timely
progress of the planned actions.

\subsection{Group Members}
\begin{itemize}
\item Nick Jones -- New Zealand eScience Infrastructure
\item Iain Larmour -- Engineering \& Physical Sciences Research Council, UK
\item Erin Robinson -- Foundation for Earth Science
\end{itemize}

\subsection{Summary of Discussion}


While little training focuses specifically on this outcome \katznote{sustainable
software?}, a variety of training activities increase researcher awareness of
and engagement with software professionals and software engineering practices.
Research Software Engineers are being recognized as critical contributors to
high quality research; the pathway to acquire and master the relevant skills
is not yet clear; equally those skills required by researchers in general are
also not commonly understood nor routinely developed.

The group's discussion explored a rapidly growing array of training that is seen
to contribute to sustainable software. The offerings are diverse, including:
self-paced online modules focused around specific tools; single and multiple day
training workshops that raise awareness of a tool chain to support collaborative
and shared software development within a research workflow; block courses
specializing on particular methods, technologies, and applications; academic
programs at undergraduate and masters levels; doctoral training programs that in
part contain requisite skills training activities.

While some of this training focuses on applying software engineering practices
within the context of research, meeting the values and goals of research are
less often incorporated as explicit learning outcomes. With software (and
similarly, data) often the only tangible artifact of a research method or
protocol, the dependency between software applications and the quality of
research adds complexity to the learner's journey. In recognition of the longer
term investment required by researchers to integrate such skills into their
research practices, many activities are focusing on emotionally engaging
researchers and cohorts, to build a sense of shared purpose beyond the obvious
goal of technical skill acquisition.

In reviewing current training activities, the group identified a variety of
perspectives seen as useful in positioning activities in ways to better
communicate why and when best to apply each activity. Training can be
categorized on a variety of spectra, with content and delivery ranging across them, for example:
programming to research; basic to advanced; technical to emotional; informal to
formal; and self-paced to participative. A few attempts have been made to situate a
cross section of training activities within such dimensions, creating easier
means of communicating the value of any specific opportunity and the pathways
across opportunities over time.

Evaluation of training delivery and outcomes is seen as a weakness common to
most non-academic training activities. Opportunities for measuring success in
delivering training start simply with collecting a Net Promoter Score, which
lets those delivering training know whether attendees are likely to recommend
the training to others. In looking at the longer term outcomes for the learner,
frameworks such as Bloom's taxonomy and Kirkpatrick's evaluation model offer possible
approaches.

In this latter case of formal evaluation, ownership of evaluation as a component
of career development for any researcher appears mostly absent. While academic
research institutions have professional development centers to support research
staff, the skills taught which might impact on sustainable software are limited
at best, and lack a clear and coherent development pathway.

Coordination of these training projects will depend on buy-in from a broad range
of training program and activity leaders, suggesting a key opportunity lies in
identifying and bringing together these people on a regular basis.

\subsection{Description of Opportunity, Challenges, and Obstacles}

Software skills are needed by an increasing array of researchers and fields. The
training arc is not well-defined, with a sometimes baffling array of training
opportunities responding to various facets of skill deficit and need. Given this
current complexity, coordination across training projects would create common
frames of reference, communicating and integrating activities to better serve
the needs of researchers.

Building this community could lift the maturity of training projects and
capabilities, enabling more advanced approaches to address key gaps in
evaluation, career development, and a lift in the standard of research
practices.

In aiming at these opportunities, it will be necessary to find the means to
support those involved in leading training activities to allocate time to
coordination activities, which will often sit beyond their current scope of
responsibility.

These activities are also distributed globally, with no single country or region
offering a comprehensive set of capabilities and initiatives. Any coordination
activity will therefore need to raise the profile of the opportunity gap with
relevant research funders and policy makers.

\subsection{Key Next Steps}

The goal of the following next steps is to quickly test whether there is
interest in establishing a community committed to increasing the degree of
coordination across training projects.

\begin{enumerate}

\item Hold a virtual meeting by December 2015, to bring together a broader group
of interest in this topic, with specific goals to:

	\begin{enumerate}
	    
	\item Identify programs with existing activities aimed at integrating across
	training projects.
	        
	\item Identify training projects with an interest in participating in
	coordination efforts.
	        
	\item Identify funding opportunities to bring together training program and
	project leaders to identify shared goals for future coordination of activities.
	        
	\item Agree on a communications plan to qualify whether programs, projects, and
	funders are interested in engaging and committing to ongoing activities.
	        
	\end{enumerate}
    
\item Review progress within 3 months, to establish next steps, if any.

\end{enumerate}

\subsection{Plan for Future Organization}

Continue to track progress by posting comments to WSSSPE3 issue.

\subsection{What Else is Needed?}

If the group moves from early-stage formation into working towards shared goals,
expertise will likely be required in pedagogy and training evaluation.

\subsection{Key Milestones and Responsible Parties}
\begin{enumerate}

\item During October, Nick Jones and Erin Robinson to draft WSSSPE3 report back.

\item Before December 2015, Nick Jones and Erin Robinson to call a meeting of
the broader group, to review key next steps.
    
\item First quarter 2016 -- if willing parties are identified, draft workshop proposal
and identify a relevant forum, including future WSSSPE events.
    
\end{enumerate}

\subsection{Description of Funding Needed}

Workshop/RCN travel funding to bring together key program, project, and funder
representatives from across North America, EU, UK, Australasia. In addition,
funding to support work on better defining the landscape of training activities,
the useful perspectives in communicating the value of the varied training
projects, and the possible pathways through training activities over time.

%%%%%%%%%%%%%%%%%%%%%%%%%%%%%%%%%%%%%%%%%%%%%%%%%%%%%%%%%%%%
\section{Software Credit Working Group Discussion}
\label{sec:appendix_SW_credit}
%%%%%%%%%%%%%%%%%%%%%%%%%%%%%%%%%%%%%%%%%%%%%%%%%%%%%%%%%%%%

Kyle Niemeyer\footnote{email: \href{mailto:kyle.niemeyer@oregonstate.edu}{kyle.niemeyer@oregonstate.edu}} will serve as the point of contact for this working group, and be responsible for ensuring timely progress of the planned actions.

%%%%%%%%%%%%%%%%%%%%%%%%%%%%%%%%%%%%%%%%%%%%%%%%%%%%%%%%%%%%
\subsection{Group Members}
%%%%%%%%%%%%%%%%%%%%%%%%%%%%%%%%%%%%%%%%%%%%%%%%%%%%%%%%%%%%

\begin{itemize}
\item Alice Allen -- Astrophysics Source Code Library
\item Sou-Cheng Choi -- NORC at University of Chicago, Illinois Institute of Technology
\item James Hetherington -- University College London
\item Lorraine Hwang -- University of California, Davis
\item Daniel S.\ Katz -- University of Chicago, Argonne National Laboratory
\item Frank Löffler -- Louisiana State University
\item Abby Cabunoc Mayes -- Mozilla Science Lab
\item Kyle E.\ Niemeyer -- Oregon State University
\item Grace Peng -- National Center for Atmospheric Research
\item Ilian Todorov -- Science \& Technology Facilities Council, UK
\end{itemize}


%%%%%%%%%%%%%%%%%%%%%%%%%%%%%%%%%%%%%%%%%%%%%%%%%%%%%%%%%%%%
\subsection{Summary of Discussion}
%%%%%%%%%%%%%%%%%%%%%%%%%%%%%%%%%%%%%%%%%%%%%%%%%%%%%%%%%%%%

The following section summarizes the working group's discussion based on contributions prior to the meeting~\cite{WSSSPE3-SC-github-issues} and the collaborative notes taken during the meeting~\cite{WSSSPE3-SC-google-notes}.
Please refer to the original sources for the unedited discussions.

Initial discussions focused on both various mechanisms for, and the philosophical approach behind, crediting software in scientific papers.
These began with proposals for various ways to credit software (or other research products including data) that contributed more significantly than a generic citation, including
\begin{itemize}
    \item A hierarchy of citations, with a ``substantial'' citation category to indicate software or data that played a more significant role in the research;
    \item Transitive credit~\cite{wssspe2_katz,Katz:2014_tc}, which assigns contriponents (contributors and components) various weights depending on their level of importance; and
    \item Project CRediT~\cite{projectcredit}, which assigns roles to paper authors based on their specific contributions; and
    \item Mozilla Science Lab's recently introduced Contributorship Badges for Science~\cite{Mozilla_badges}, which provide a badge---associated with ORCID~\cite{orcid}---based on author contributions in a similar manner to Project CRediT.
\end{itemize}
However, as of this writing, only the Project CRediT roles~\cite{McCall2015_credit,Lin2015_credit} and Contributorship Badges~\cite{Mozilla_badges} have been implemented in published papers, though both of these only provide a single ``Software'' or ``Computation'' category associated with software.
In addition, neither of these options allow for the citation of software itself, but only provide an author contribution related to software.
The discussion quickly focused on transtitive credit as a more quantitative measure of allocating credit to both authors and software, although there were some concerns about authors overestimating their own contributions compared to prior work.

The discussion then evolved into philosophical questions about the importantance or reliance of a particular work on prior science, materials, or software---in other words, is there a difference between depending on prior scientific advances and depending on particular software (or experimental equipment)?
Multiple contributors converged on the conclusion that unique capabilities require some additional credit.
The---albeit limited---consensus was that if a particular study relied on the unique capabilities of software, data, or an experimental apparatus, then the authors or developers that created this capability should be credited somehow.

The group also agreed on the fact that additional data was required to support the assertion that software was not being sufficiently cited in the literature.
In particular, this issue seemed to be field-dependent.
For example, as shown by study of Howison and Bullard~\cite{Howison2015}, in the field of biology the most-cited papers appear to be those describing scientific software.
However, this may not---and likely is not---the case in other fields, nor is it clear whether developers of scientific software, even in the case of the biology field, are receiving sufficient credit for their efforts.

During the first day of WSSSPE3 breakout sessions, the group discussed the Entertainment Identifier Registry (EIDR)\footnote{Entertainment Identifier Registry: \url{http://eidr.org}} as a potential model for scientific software.
That system assigns unique Digital Object Identifiers (DOIs)---the same system used for scientific publications---to all content (e.g., movies, television shows) and contributors, along with relevant metadata.
One important use of this system is to track rights and credit for contributors to entertainment works in order to distrubute revenues---similar to the proposed transitive credit concept.

The group also discussed separating quantitative measures (e.g., number of citations) from the value of a work in order to give credit, moving towards qualitative or anecdotal evidence of value.
Other topics brought up included a form of PageRank~\cite{Brin1998} for citations, based on number of mentions, and using market penetration or adoption rate in a community as a metric, although it was not clear how this would be measured.
Finally, the concept a software tool's uniqueness or indispensability to a community was mentioned, with value being characterized by a particular piece of software either offering unique capabilities or doing something better, faster, or with less computational requirements than other offerings.

On the second day of WSSSPE3, the working group decided to put aside the taxonomy of contributions and focus on software citations to ensure developers receive credit (regardless of contribution).
Eventually, once standardizing software citations, the goal would be to return to establishing different roles\slash contributions for this credit.




%%%%%%%%%%%%%%%%%%%%%%%%%%%%%%%%%%%%%%%%%%%%%%%%%%%%%%%%%%%%
\subsection{Description of Opportunity, Challenges, and Obstacles}
%%%%%%%%%%%%%%%%%%%%%%%%%%%%%%%%%%%%%%%%%%%%%%%%%%%%%%%%%%%%

There currently is no standard/accepted mechanism for citing software or receiving credit for software (akin to publications/citations).
Software is eligible for DOI assignment, but DOI metadata fields are not well tuned or standardized for software (vs. publications).
Some software providers apply for DOIs but it is still not widely adopted.
Also, there is no mechanism to cite software dependencies within software.

Obstacle: indexers (e.g., Scopus, Web of Science, Google Scholar) don't currently support software/data document types or DataCite DOIs.

Obstacle: no standard mechanism for software to cite other software.

Obstacle: for tenure, there is no standard for software products to be included in policy,
sometimes not even within one university: hard to change globally.

%%%%%%%%%%%%%%%%%%%%%%%%%%%%%%%%%%%%%%%%%%%%%%%%%%%%%%%%%%%%
\subsection{Key Next Steps}
%%%%%%%%%%%%%%%%%%%%%%%%%%%%%%%%%%%%%%%%%%%%%%%%%%%%%%%%%%%%

\begin{enumerate}
\item Hold virtual meeting to determine group members responsible\slash willing to work on the following tasks, to be organized within one month of the workshop
\item Compile best practices of software citation across multiple disciplines, including journals and communities of interest\slash practice in the research world, to begin by December 2015
\item Compile examples of including other products in promotion and tenure dossier, to begin by December 2015
\item Draft the Software Citation Principles document (including citation metadata file), by April 2016
\item Publish\slash release the Software Citation Principles document, by August 2016
\item Reach out to journals, publishers, teachers\slash educators, indexers, and professional societies---likely through meetings with key groups, to begin by September 2016

\end{enumerate}

%%%%%%%%%%%%%%%%%%%%%%%%%%%%%%%%%%%%%%%%%%%%%%%%%%%%%%%%%%%%
\subsection{Plan for Future Organization}
%%%%%%%%%%%%%%%%%%%%%%%%%%%%%%%%%%%%%%%%%%%%%%%%%%%%%%%%%%%%

The WSSSPE breakout group plans to join efforts related to citing software with the FORCE11 Software Citation Working Group (FORCE11-SCWG)\footnote{FORCE11 Software Citation Working Group, \url{https://www.force11.org/group/software-citation-working-group}}; Kyle Niemeyer formally requested the merging of these groups following the meeting.
However, some future plans of the WSSSPE group fall outside the scope of FORCE11-SCWG, which covers software citation practices.
These activities include working with indexers such as Web of Science and Scopus to index software citations archived on, e.g., Zenodo\footnote{Zenodo, \url{https://zenodo.org}}, and pursuing the development of an open indexing service; such plans will be pursued either separately or through the formation of follow-on FORCE11 working groups.

The group will primarily communicate electronically, with Kyle Niemeyer responsible for ensuring regular progress.

%%%%%%%%%%%%%%%%%%%%%%%%%%%%%%%%%%%%%%%%%%%%%%%%%%%%%%%%%%%%
\subsection{What Else is Needed?}
%%%%%%%%%%%%%%%%%%%%%%%%%%%%%%%%%%%%%%%%%%%%%%%%%%%%%%%%%%%%

The near-term actions of the group, focused mainly on software citation, do not require any additional resources.
However, connections with publishers and indexers will be needed to pursue related activities, although the FORCE11-SCWG may satisfy this need; in addition, some members of the group already reached out to relevant contacts.
Funding may be needed to organize meetings or for group members to attend relevant meetings, as discussed further below.

%%%%%%%%%%%%%%%%%%%%%%%%%%%%%%%%%%%%%%%%%%%%%%%%%%%%%%%%%%%%
\subsection{Key Milestones and Responsible Parties}
%%%%%%%%%%%%%%%%%%%%%%%%%%%%%%%%%%%%%%%%%%%%%%%%%%%%%%%%%%%%

Following the meeting, Kyle Niemeyer formally requested the merging of software citation activities with FORCE11-SCWG.
Within a month of the meeting, Kyle will organize a virtual meeting of the group and lead the division of responsibilities for compiling existing practices of software citation and including software\slash products in promotion and tenure dossiers.
Building off of these efforts, the next major milestone is drafting the Software Citation Principles document in collaboration with the SCWG, targeted for April 2016.
While the existing directors of the SCWG, Arfon Smith and Dan Katz, lead the efforts of that group towards the Software Citation Principles document, Kyle will help coordinate contributions from the WSSSPE group members.


%%%%%%%%%%%%%%%%%%%%%%%%%%%%%%%%%%%%%%%%%%%%%%%%%%%%%%%%%%%%
\subsection{Description of Funding Needed}
%%%%%%%%%%%%%%%%%%%%%%%%%%%%%%%%%%%%%%%%%%%%%%%%%%%%%%%%%%%%

Some funding would be useful to support primarily travel to conferences for group meetings (e.g., FORCE2016\footnote{FORCE2016, \url{https://www.force11.org/meetings/force2016}}), and to hold meetings to bring together both group members and key stakeholders (e.g., journals, publishers, professional societies, indexers).
In addition, funding would be desired to support group members' time to perform work towards the key steps described previously.

%%%%%%%%%%%%%%%%%%%%%%%%%%%%%%%%%%%%%%%%%%%%%%%%%%%%%%%%%%%%
\section{Publishing Software Working Group Discussion}
\label{sec:appendix_publishing_SW}
%%%%%%%%%%%%%%%%%%%%%%%%%%%%%%%%%%%%%%%%%%%%%%%%%%%%%%%%%%%%

\subsection{Group Members}
{\small
\begin{longtable}{ll}
   Steven R. Brandt & Louisiana State University
\\ Daniel Gunter    & LBNL
\\ Yuhan Ding       & Illinois Institute of Technology
\\ Neil Chue Hong   & Software Sustainability Institute
\end{longtable}
}

\subsection{Summary of Discussion}

This group explored the value of executable papers (papers whose content includes
the code needed to produce their own results), and other forms of publishing which
include dynamic electronic content. Transitioning to this type of publication offers
possibilities of addressing, or partially addressing sustainability concerns 
such as reproducibility (the paper contains all the artifacts needed to verify its
results), transitive credit (modules an executable paper depends on must be explicitly
loaded, making it more feasible to identify them), and improving documentation (an executable
paper must explain what its code does).

While the group did not feel it had a way to influence what is currently happening
in these experimental publishing venues, it felt that creating and curating a list of
such efforts would have value by attracting interest to these activities.
The Software Sustainability Institute agreed to host this list.

A tentative first cut at the list contains the following:
\begin{itemize}
\item ACM Transactions on Mathematical Software (TOMS) - which provides the extra step
 of having reviewers validate the code which was submitted with the publication.
\item The Mathematica Journal - which publishes Mathematica notebooks (with equations,
figures, etc.) directly.
\item O'Reily Media has announced that it plans to make IPython Notebooks a first-class
 authoring environment for their publishing program alongside their existing mechanisms.
\item Nature is offering a list of notebooks published alongside more traditional articles,
 and is looking at ways to make these documents more official. There are, in fact, a
 number of journals that offer "electronic supplements" to the more traditionally published
 static articles.
\item There is also a list of "reproducible academic publications" maintained here:
  \url{https://github.com/ipython/ipython/wiki/A-gallery-of-interesting-IPython-Notebooks#reproducible-academic-publications}
\end{itemize}

\subsection{Description of Opportunity, Challenges, and Obstacles}

The opportunity is to collect a list of current executable papers and
shine a light on the experiments and development efforts currently underway.

The only obstacle to this is the difficulty in finding and identifying such
publications. The Software Sustainability Institute was able to do something similar
for publications about software by making a public page containing a catalog
of these publications and enlisting the help of the community to grow the list.

\subsection{Key Next Steps}

Create the first version of the web page to be displayed on the Software Sustainability
Institute's website.

\subsection{Plan for Future Organization}

None at this time.

\subsection{What Else is Needed?}

Nothing else at this time.

\subsection{Key Milestones and Responsible Parties}

Steven R. Brandt will create a first version of the page within a week or so of the WSSSPE3 conference.

Neil Chue Hong will take responsibility for the page once it is up.

\subsection{Description of Funding Needed}

None.

%%%%%%%%%%%%%%%%%%%%%%%%%%%%%%%%%%%%%%%%%%%%%%%%%%%%%%%%%%%%
\section{User Community Working Group Discussion}
\label{sec:appendix_user_community}
%%%%%%%%%%%%%%%%%%%%%%%%%%%%%%%%%%%%%%%%%%%%%%%%%%%%%%%%%%%%

\todo{add POC here}

\subsection{Group Members}

\begin{itemize}
\item Ethan Davis -- UCAR Unidata
\item Dan Gunter -- Lawrence Berkeley National Lab
\item Liz Jessup -- University of Colorado
\item Mark Miller -- University of California, San Diego
\item Lindsey Powers -- The HDF Group
\item Daniel Ziskin -- NCAR Atmospheric Chemistry Observations and Modeling (ACOM) Laboratory
\end{itemize}

\subsection{Summary of Discussion}

\subsection{Description of Opportunity, Challenges, and Obstacles}

The main opportunity is to increase awareness among scientific
software developers and project managers of the importance of
developing a community around their project.
While this message is fairly well understood in the open source
community, the scientific community can be more focused on the
science a software project is supporting rather than the software
project itself.

As with many of the issues relevant to the sustainability of science
software, the main challenge here will be changing the culture and
expectations around scientific software.

\subsection{Key Next Steps}

\begin{itemize}
\item Survey successful science software projects
\item Survey community members from the surveyed projects
\item Distill the survey results and document best practices around community engagement
\item Look for ways to raise awareness
\end{itemize}

\subsection{Plan for Future Organization}


\subsection{What Else is Needed?}


\subsection{Key Milestones and Responsible Parties}


\subsection{Description of Funding Needed}


\bibliographystyle{vancouver}

\bibliography{wssspe}
\end{document}

