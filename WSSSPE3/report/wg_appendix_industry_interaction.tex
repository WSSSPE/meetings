%%%%%%%%%%%%%%%%%%%%%%%%%%%%%%%%%%%%%%%%%%%%%%%%%%%%%%%%%%%%
\section{Transition Pathways to Sustainable Software: Industry \& Academic Collaboration Working Group Discussion}
\label{sec:appendix_industry_interaction}
%%%%%%%%%%%%%%%%%%%%%%%%%%%%%%%%%%%%%%%%%%%%%%%%%%%%%%%%%%%%

Nic Weber\footnote{email: \href{mailto:nmweber@uw.edu}{nmweber@uw.edu}} will serve as the point of contact for this working group.

\subsection{Group Members}

\begin{itemize}
\item Nic Weber -- University of Washington
\item Suresh Marru -- Indiana University
\item Jeffrey Carver -- University of Alabama
\item Davide DelVento -- NCAR/CISL
\item Steven Brandt -- Louisiana State University
\end{itemize} 

\subsection{Summary of Discussion}

Our initial broad question was, "What makes for successful transitions of scientific software from academia to industry?" There are a number of potential funding transitions that may occur:  
%
\begin{itemize}

\item A project could be \textbf{refunded}, and development or maintenance of
the software continue as planned.

\item A project might locate a \textbf{new source of funding} in which case the
software may be further developed or simply maintained as before.

\item The project could transition to a \textbf{community supported model}
whereby ownership, maintenance, and stewardship of the software become similar
to peer-production models in open-source (e.g., see Howison~\cite{howison_sustaining_2015}).

\item The project could receive some form of industry sponsorship in which case
ownership of the intellectual property, licensing, maintenance activities,
hosting, etc.\ may change significantly.

\item The project could gain attention from a industry use case who would potentially make in-kind
contributions by having paid staff contribute to the software.

\end{itemize}

We characterized each of these potential changes in funding as ``transition
pathways'' to sustainable software (see similar work by Geels and
Schot~\cite{Geels:2007}).

Our work at WSSSPE3 included the following activities (described in detail
below): (1) brainstorming goals for this type of research, (2) imagining
potential outcomes of completing a set of case studies on this topic, and (3)
generating a set of working definitions for some of the broad concepts we are
describing.

First, we discussed the \textbf{goals} of this research, attempting to answer the 
question \emph{What is the goal of doing research on transition pathways?}
A number of research questions arose:  Can we identify collaborations that have 
occurred and try to understand which were successful, which were unsuccessful, 
and what factors contributed to these successes/failures? Can we determine what 
each partner wants to get out of such a collaboration? For example, why would 
industry be interested in collaborating with academia? Or why would academia 
be interested in collaborating with industry? How could we design a study that 
focused on the impact of the software in undergoing this type of transition?

Next, we imagined \textbf{potential outcomes} of research on this topic, involving 
a set of case studies that look at successful and unsuccessful
transitions of researchers between academia and industry. This might address 
each of the transition types (described below). Successful transitions are
described as those that lead to either weak or strong sustainability (also
defined below). In addition, the results from this research might help create a 
generalizable framework that might allow for the study of different transition 
pathways (other than academia to industry).

Finally, we created some \textbf{general definitions} for these concepts; we 
characterize transitions in the following ways:
\begin{itemize}

\item Handoff model: academia initially writes the software, industry (for-profit 
or nonprofit) then takes over the project.

\item Co-Production Model: industry and academia interact throughout development
of the project.

\item Sponsorship Model: academia writes and maintains the software; 
industry contributes funding for the development\slash maintenance of software.
In this example, industry is also likely a user of the software.

\item Spinoff model: transition to a for-profit or non-profit company owned by or in
collaboration with original developers.

\end{itemize}

We characterized sustainability in the following ways:
\begin{itemize}

\item Weak Sustainability: Software continues to be accessible, useful, and
usable.

\item Strong Sustainability: Software meets criteria above, but is also able to
be reused for further innovation (i.e., issued non-restrictive open-source
license).

\end{itemize}
We refer readers to Becker et al.~\cite{Becker:2014} for an extended discussion of weak versus strong
sustainability. 

\subsection{Description of Opportunity, Challenges, and Obstacles}
The opportunity is to create a catalog of success/failure for current and future software projects to be prepared for transitions and achieve sustainability of the software. 

The obstacle is more superficial, in finding a champion to gather such information. It will be a challenge to keep this information and surveys updated. With changing rapidly changing industry landscapes, an obsolete survey could be of less or no use. 

\subsection{Key Next Steps}

Identify projects that are collaborative, perhaps by reviewing funded projects from programs specifically geared towards industry academic collaborations.

Develop a systematic process for conducting case studies (what kind of data are being gathered about each case)

\subsection{Plan for Future Organization}
No concrete plans were made at this point. If community can rally behind, some momentum could be built. If you are interested post at \url{https://github.com/WSSSPE/meetings/issues/46} 

\subsection{What Else is Needed?}
Nothing at the moment. 

\subsection{Key Milestones and Responsible Parties}
A key portion of this effort will require focused surveys of projects which have succeeded and failed in transition. Both these categories will yield good learning on what works and what does not work. The group has identified what needs to be studied further, but has not identified responsible parties to conduct them.

Community could help in gathering data by means of Interviews; Historical documents / documentation and Surveys

An example data collection will be: 
\begin{itemize}
\item Origin: Where did project start? 
\item People involved: How many people in original project were involved in transition/collaboration? 
\item Specs on software
\item Language
\item Size
\item Hardness (age)
\item Lead-up to Transition: How long was project in development before it began transition?
\item Motivation for Transition: Why was transition initiated? By whom? 
\end{itemize}

\subsection{Description of Funding Needed}
Concrete funding needs were not discussed in this working group but a general impression was a seed funding will motivate members of this group or others in community to launch a survey effort. 