%%%%%%%%%%%%%%%%%%%%%%%%%%%%%%%%%%%%%%%%%%%%%%%%%%%%%%%%%%%%
\subsection{Software engineering processes tailored for research software}
\label{sec:soft-eng}
%%%%%%%%%%%%%%%%%%%%%%%%%%%%%%%%%%%%%%%%%%%%%%%%%%%%%%%%%%%%

\note{Anshu to write this}

%Introduction to group here, including the overall objective of work in this area.

\subsubsection{Participants}
\begin{itemize}
\item Mark Abraham
\item Anshu Dubey - is a computer scientist at the Mathematics and
  Computer Science Division at Argonne National Laboratory. She has
  been deeply involved with the development of multiphysics scientific
  software in multiple domains which has lead to an interest in, and
  understanding of, software engineering concerns in scientific
  computing. 
\item Hans Fangohr
\item Dominic Kempf
\item Eric Seidel
\end{itemize}
\subsubsection{Working group objective}
Scientific computing software lags behind commercial software in
adoption of software engineering practices. Causes range from lack of
exposure to the practices to distrust of adopting practices because
they do not meet the needs of the teams developing such software. This
gap is particularly acute in the area of software testing,
verification and validation where the standard practices are
simulataneously inadequate and over-onerous. The objective of this
working group is to (1) conduct literature survey to guage the extent
of awareness of issue in general, (2) generate content useful for
the community where needed, and (3) curate the collected and added
content for the use of the community.

\subsubsection{Gap or challenge}

%What is the gap or challenge being addressed?

\subsubsection{Relevant people and resources}

What people, groups, or resources are needed.

\subsubsection{Plans}

What tasks will the working group undertake

\subsubsection{SMART steps}

What are the first SMART steps proposed?

\subsubsection{More information \& joining instructions}

How could a reader get more information or get more involved?