%%%%%%%%%%%%%%%%%%%%%%%%%%%%%%%%%%%%%%%%%%%%%%%%%%%%%%%%%%%%
\subsection{Open research index}
\label{sec:open-research-index}
%%%%%%%%%%%%%%%%%%%%%%%%%%%%%%%%%%%%%%%%%%%%%%%%%%%%%%%%%%%%

\note{Dan to write this}

The aim of this group is to investigate the building of an index of research products in an open sustainable manner.  Our goal is not to eliminate commercial products, but to build on what?s there and provide data and services that are missing.

The Open Research Index should take in all research products (papers, software, datasets, workflows, etc.) from their publishers and recorders (journals, societies, domain repositories, government [open access] repositories, preprint servers, general repositories [e.g., figshare, zenodo]) and other services (CrossRef, ORCID).
Each product should list authors and citations and allow people to search the resulting network.
Users should also be able to interact with their own record and edit it, like Google Scholar allows.

\subsubsection{Participants}

The members of this group at WSSSPE4 were:
Gabrielle Allen,
Bruce Childers,
Robert Haines,
Caroline Jay,
Dan Katz,
Daniel Mosse,
Kyle Niemeyer,
and
Robert McDonald.

\subsubsection{Working group objective}

The working group's plans are relatively simple to express, though quite complex to undertake:

\begin{enumerate}
\item Determine a plan to build an open research index that allows various stakeholders to satisfy their needs
\item Then determine if the plan is feasible
\item If so, then obtain resources
\item If successful, then build the index
\end{enumerate}

\subsubsection{Gap or challenge}

Google and others provide some services now.
But these services (and the underlying data) could be removed at any time.
And the community cannot build new services.

\subsubsection{Relevant people and resources}

Names of people willing to contribute (or expertise needed), and how they?ll contribute
- Could bring in someone from ORCID, CrossRef, DataCite, ImpactStory/depsy, altmetrics.org, Plum Analytics, GitHub, Open AIRE, CHORUS, FORCE11, COS, SHARE, CASRAI, Portico, softwareheritage.org, DBLP, eSTEP (CWO)
- Work with some initial publishers who don?t have competing services (domain societies), PubMed, 
- Need to discuss with potential funders
   - What data/services would they like to have
      - Are they willing to invest in this?
      - E.g. current Arnold Foundation Open Science call
- Who would actually lead and coordinate this activity?

What external resources are needed, and how they can be brought in?
- Money
- Time


\subsubsection{Plans}

Currently unclear if any of us have the time/energy to pursue this

If we do, plan could be:
- Obtain funding for conversations
- Conversations with potential partners (publishers, orgs, funders) (could be combined with meeting below)
- Conversation with Google Scholar developer for better understanding of what they did and are doing
- Funding for a meeting (perhaps from NSF) (Dagstuhl meeting?)
- Meeting to define a plan
- Funding to implement the plan

Could also build a mailing list for us and discuss further, depending on how receptive others are to this idea


Possible Plans

Look at curriculum lattes (Brazil) and/or Researchfish (UK).  It has some problems that people don?t like, could check what features it has and what the issues are

Look at what CrossRef data is

What research products should we track?

What model should be used?
- Crawling and obtaining data
- Consortium model
- Need analysis and decision

Need agreements with publishers, repositories, services
- Some of these will have costs, though some may be free
- Need to determine costs
- Perhaps publishers would pay us to be listed eventually, as we gain power
- To some extent, we would be competing with some of the publishers? products

If crawling model:
- How to crawl all of these products?
- What data (metadata) should be the output of the crawls?
    - What happens when the products being crawled and/or their metadata change?
- How to store all of this data (metadata)?

What services to provide on this data (metadata)?

Who will create these services?

What services will the publishers allow to be provided?

Need to develop a plan that shows incremental progress and successes

Consider applying to Google project call to get Google interest and support?


\subsubsection{SMART steps}

What are the first SMART steps proposed?

Community mailing list/forum (Kyle, today)
done - wg-open-research-idx on wssspe slack channel

Find a PI (we suggest Neil Chue Hong)

Apply for initial funding
- For the US, consider Arnold Foundation Open Science award as a planning activity (who, due Dec 15)
- For the UK, 

Determine project name, initial vision, initial mission, logo (who, when)

Identify initial use cases (who, when)

Identify potential stakeholders and partners (who, when)

Discuss this idea with potential partners (who, when)
- Get their feedback \& buy-in
- While doing so, seek suggestions for the right project manager to coordinate this

Build up an advisory board (who, when)

Build up a leadership committee and determine a leader (who, when)

Hire a project manager (who, when)

Iterate use case with stakeholders (who, when)
- E.g. funders, Researchfish, researchers, ?

Iterate vision \& mission with leadership committee and advisory board (who, when)

Develop initial development plan (who, when)


\subsubsection{More information \& joining instructions}

How could a reader get more information or get more involved?

add into about joining WSSSPE slack and then this channel.