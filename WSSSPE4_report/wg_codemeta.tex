%%%%%%%%%%%%%%%%%%%%%%%%%%%%%%%%%%%%%%%%%%%%%%%%%%%%%%%%%%%%
\subsection{CodeMeta}
\label{sec:CodeMeta}
%%%%%%%%%%%%%%%%%%%%%%%%%%%%%%%%%%%%%%%%%%%%%%%%%%%%%%%%%%%%

\note{Alice to write this}

"Define a standard for metadata to be used in current repos including authors, dependencies, ... repo url" had been listed as a possible project for a working group; as several attendees have been participating in a project, CodeMeta,\footnote{\url{http://codemeta.github.io/}} that is currently working on software metadata, an enthusiastic group formed to work on advancing the CodeMeta project.

\subsubsection{Participants}

\begin{itemize}
\item Gabrielle Allen <gdallen@illinois.edu>
\item Stephan Druskat <stephan.druskat@hu-berlin.de>
\item Iain Emsley <iain.emsley@oerc.ox.ac.uk>
\item Carole Goble <carole.goble@manchester.ac.uk>
\item Chris Gwilliams <gwilliamsc@cardiff.ac.uk>
\item Rafael Jimenez <rafael.jimenez@elixir-europe.org>
\item Frank L�ffler <knarf@cct.lsu.edu>
\item Kyle Niemeyer <Kyle.Niemeyer@oregonstate.edu>
\item Thomas Robitaille <thomas.robitaille@gmail.com>
\item Rob Welch <py12rw@leeds.ac.uk>
\item Alice Allen <aallen@ascl.net>
\end{itemize}

\subsubsection{Working group objective}

The primary objective of this working group was to find ways to help the CodeMeta project come to fruition. CodeMeta seeks in part to create a "Rosetta Stone" for software metadata to facilitate retaining such metadata between repositories, services, registries, indexers, publishers, citation managers, and other entities that create, ingest, use, and/or store metadata about software. The project also wants to establish a JSON-LD schema as a tool for making metadata machine-readable.\citep{CodeMeta_schema}

\subsubsection{Gap or challenge}

Research software is often not shared; that that is shared may not have much metadata associated with it, and that which does exist often does not travel further than the website on which the software resides. CodeMeta wants to incentivize software developers to release their software, encourage the development of metadata for it, enable credit assignment and citation of research software and increase its discoverability, more easily track dependencies, and enable reuse of software metadata, all goals that WSSSPE attendees have great interest in supporting. 

\subsubsection{Relevant people and resources}

The CodeMeta team members (as of the WSSSPE4 meeting) are listed below; the CodeMeta project leads are shown in bold, and team members who also part of the WSSSPE working group are shown in italics :

\begin{itemize}
\item {\bf Carl Boettiger}, UC Berkeley
\item {\bf Matt Jones}, NCEAS
\item {\bf Arfon Smith}, GitHub
\item {\bf Abby Mayes}, Mozilla Science Lab
\item Yolanda Gil, USC ISI
\item Peter Slaughter, NCEAS
\item Patricia Cruse, DataCite
\item Neil Chue Hong SSI
\item Merc� Crosas, Harvard IQSS
\item Martin Fenner, DataCite
\item Mark Hahnel, figshare
\item Luke Coy, RIT & MSL
\item {\em Kyle Niemeyer}, Oregon State
\item Krzysztof Nowak, Zenodo
\item Daniel Katz, NCSA
\item {\em Carole Goble}, University of Manchester
\item Ashley Sands, UCLA
\item {\em Alice Allen}, ASCL

\end {itemize}

Resources already established by the CodeMeta team can be found online: 

\begin{itemize}
\item Main website and meeting information (\url{http://codemeta.github.io/})
\item Github repository (\url{https://github.com/codemeta/codemeta})
\item List of milestones (\url{https://github.com/codemeta/codemeta/milestones})
\item Gitter discussion site (\url{https://gitter.im/codemeta/codemeta})
\end {itemize}

Other resources to draw on are the software metadata vocabularies of Schema.org (\url{https://schema.org/SoftwareSourceCode}) and DOAP (\url{https://github.com/ewilderj/doap}), and the Force11 Software Citation Principles\citep{Smith:2016sc}. Relevant documents on research software, including the Guidelines for persistently identifying software using DataCite\citep{Gent2016} are available on the UK's Research Data Network's Research Software web page (\url{https://research-data-network.readme.io/docs/research-software}).

\subsubsection{Plans}

Our plans are for the working group to engage with those already working on CodeMeta, to examine the existing CodeMeta crosswalk table to see what improvements and additions might be made, and to determine how to engage the community and provide ongoing social engagement and structure. Further, the group wants to assist in the implementation of the specifications and, by providing an outsider�s view, contribute suggestions for more understandable project documentation. Finally, the WG seeks a better way or ways to present the crosswalk table to make it more easily understood and consumable by research software communities. 

\subsubsection{SMART steps}

With a comparatively large working group, the use of the established CodeMeta Github repository's issue tracking, and the quick responsiveness of CodeMeta lead Matt Jones to a barrage of questions, comments, and logged issues, several of our SMART steps were completed at WSSSPE; these are identified below in italics.

\begin{itemize}
\item {\em Write to the managers of CodeMeta project to start dialog with the two groups} 
\item {\em Post questions generated by our discussion as issues on the CodeMeta repository}
\item Look at the crosswalk table and Schema.org data elements for common elements
\item Identify text in the project documentation that is unclear and suggest changes
\item Define a list of services to be added to the crosswalk table and match terms
\item {\em Identify two roles and information on CodeMeta that would be useful to these people for engaging with the project}
\item Determine a better way or ways to present the crosswalk table
\item Create a mailing list for WSSSPE CodeMeta participants
\end{itemize}


\subsubsection{More information \& joining instructions}

Notes for the working group were taken in real time in a Google document\footnote{\url{https://docs.google.com/document/d/1UxlHIoBRgVWB8NAXYf4Q0yS7PAqa-EYwFfsNuTKSPbs/edit}} and include email replies in response to WG questions from one of the CodeMeta project leads.

Because of the work done at WSSSPE4, the CodeMeta project README file\footnote{\url{https://github.com/codemeta/codemeta/blob/master/README.md}} was greatly expanded to include a description of the project that is geared to those with little or no prior knowledge of the project, a list of contributors, information on how one can get involved, a brief project history and who is managing the project, and links to additional information. Though a Google group mailing list has been established for the working group, the easiest way to engage with the CodeMeta project is through its Github repository\footnote{\url{https://github.com/codemeta/codemeta}}. 

